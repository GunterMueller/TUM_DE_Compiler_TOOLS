\documentstyle[12pt,a4wide,german]{article}
\parindent0pt
\parskip0.5em
\newcommand{\fp}[1]{{\em$\langle\mbox{#1}\rangle$}}
\newcommand{\var}[1]{{\tt\mbox{#1}}}
\newcommand{\mac}[1]{{\bf\mbox{#1}}}

\begin{document}

\begin{titlepage}
\title{Die parametrisierte C--DatenstrukturListe}
\author{Ulrich Vollath\\Institut f"ur Informatik\\TU M"unchen}
\date{\copyright 1991\\\today}
\end{titlepage}
\maketitle

\section{Einbindung}
Die Einbindung der parametrisierten Datenstruktur erfolgt durch
die C--Pr"aprozessoranweisung \#include ''parlist.h''.
Danach kann die Instantiierung von Listentypen 
wie in Abschnitt \ref{instanz} beliebig
oft erfolgen.

\section{\label{instanz}Instantiierung}
Die Instantiierung der Datenstruktur erfolgt mittels des Makro--Aufrufs:

\mac{INSTANZ\_SLIST}(\fp{Listtyp},\fp{Basistyp},\fp{vergleich},
\fp{abgleich},\fp{kopiere},\fp{abbruch})

Dabei haben die formalen Parameter die folgende Bedeutung:

\fp{Listtyp} ist der Name des durch die Instantiierung definierten Datentyps.
\fp{Listtyp} mu"s also ein f"ur eine Typdefinition g"ultiger Bezeichner sein.

\fp{Basistyp} ist der Basistyp, "uber dem der Listentyp definiert wird. Dieser 
Typ muss an der Instantiierungsstelle definiert sein.


\fp{vergleich} vergleicht zwei Variablen \var{ziel} und 
\var{quelle} von Typ \fp{Basistyp} * bez"uglich der zu verwendenden Ordnung.
Das Ergebnis ist:
\begin{itemize}
\item $< 0$, falls $*\var{ziel} < *\var{quelle}$
\item $0$,  falls $*\var{ziel} = *\var{quelle}$
\item $> 0$, falls $*\var{ziel} > *\var{quelle}$
\end{itemize}

\fp{abgleich} wird aufgerufen, wenn beim Eintrag bzw.\ Merge ein bereits
vorhandener Eintrag erneut hinzugef"ugt werden soll.

\fp{kopiere} entspricht der Zuweisung auf der Ebene des Typen \fp{Basistyp}.

\fp{abbruch} wird ausgef"uhrt beim Vorliegen eines fatalen Fehlers beim
Ausf"uhren einer Listenoperation, i.\ allg.\ bei Speichermangel etc.
\fp{abbruch} kann z.\ B.\ zum (Teil--) Programmabbruch mit Fehlermeldung 
"uber exit() oder longjmp() f"uhren. Es ist aber auch eine leere Anweisung
m"oglich, die Listenoperation liefert dann einen Fehlerwert ab.


\section{Verf"ugbare Operationen}

\fp{Listtyp} enter\fp{Listtyp}(\fp{Listtyp} *liste,Basistyp *quelle)

\fp{Listtyp} lookup\fp{Listtyp}(\fp{Listtyp} *liste,Basistyp *quelle)

\fp{Listtyp} merge\fp{Listtyp}(\fp{Listtyp} *zielliste,\fp{Listtyp} quelliste)

void freelis\fp{Listtyp}(\fp{Listtyp} lis)

int init\fp{Listtyp}(void)

\end{document}

