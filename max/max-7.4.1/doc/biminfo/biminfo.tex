
\documentstyle[a4,german,11pt]{article}

\germanTeX

\pagestyle{plain}

\begin{document}

\title{
Fortgeschrittenenpraktikum\\
\vspace*{1em}
{\em maxbrowse} -\\
Ein flexibler Browser f"ur das MAX-System}

\author{
  Andreas Graf
\and
  Boris Reichel
\and
  Stefan Duschl}

\maketitle

{\bf Aufgabenstellung:} F"ur das compilererzeugende System MAX ist
ein graphischer Browser zu entwerfen und zu implementieren, der
es erm"oglicht, attributierte Parseb"aume hypertextartig zu betrachten.\\[0.5cm]


{\bf Problematik:} Die auftretenden Schwierigkeiten sind:
\begin{itemize}
	\item Zur Visualisierung kann nur beschr"ankt auf existierende
		Widget-klassen zur"uckgegriffen werden,
	\item Zur Laufzeit des Zielcompilers werden f"ur die Visualisierung
	     Informationen ben"otigt, die nicht mehr explizit zur Verf"ugung
		stehen,
	\item Attribute eines Knotens k"onnen Terme sein. Diese Terme werden
		wiederum als B"aume dargestellt. Die genaue Struktur eines
		derartigen Baums steht erst nach der Phase der Attributauswertung
		fest.
\end{itemize}\par

{\bf L"osungsansatz:} Die einzelnen Punkte wurden mit folgenden Ans"atzen gel"ost:
\begin{itemize}
	\item Neue Widgetklassen mit den f"ur den Browser n"otigen Eigenschaften
		wurden implementiert.
	\item Bei der Erzeugung des Zielcompilers werden die ben"otigten Informationen
		in den Quellcode des Zielcompilers aufgenommen und stehen somit
		permanent zur Verf"ugung.
	\item Bei Erzeugung des Baumes wird die Struktur der Termattribute mit Hilfe
		eines Parsers analysiert und mit Hilfe der gewonnenen Informationen
		der zu visualisierende Unterbaum erzeugt.
\end{itemize}\par
\end{document}




