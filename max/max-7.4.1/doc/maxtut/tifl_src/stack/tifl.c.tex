\subsection{tifl.c}
\begin{verbatim}
/*------------    main program for TIFL  ----------*/
#include <stdlib.h>
#include <stdio.h>
#include <string.h>
#include "tifl_spec.h"
#include "stack.c"


ELEMENT itos( ELEMENT i ){
   char s[20];
   sprintf(s,"%d", i );
   return  atoe(s);
}

/*---------------------------------------------------------------------------
  predefined functions
  ---------------------------------------------------------------------------*/

  /* boolean ops  */
ELEMENT or( ELEMENT o1, ELEMENT o2 ){   return btoe( etob(o1) || etob(o2) ); } 
ELEMENT and( ELEMENT o1, ELEMENT o2 ){   return btoe( etob(o1) && etob(o2) ); }
ELEMENT not( ELEMENT o1 ){ if(etob(o1))
                             { return btoe( 0 ); }
                           else 
                             { return btoe( 1 ); }
                         }

  /* arithmetic ops   */
ELEMENT add( ELEMENT o1, ELEMENT o2 ){   return itoe(etoi(o1) + etoi(o2)); }
ELEMENT sub( ELEMENT o1, ELEMENT o2 ){   return itoe(etoi(o1) - etoi(o2)); }
ELEMENT mul( ELEMENT o1, ELEMENT o2 ){   return itoe(etoi(o1) * etoi(o2)); }
ELEMENT dvd( ELEMENT o1, ELEMENT o2 ){   
                                      if(!etoi(o2))
                                        { printf("\n!!! Division by zero\n");
                                          return itoe(1);
                                        }
                                      else
                                        { return itoe(etoi(o1) / etoi(o2)); }
                                     }

  /* relations  */
ELEMENT lt( ELEMENT o1, ELEMENT o2 ){   return btoe(o1<o2); }
ELEMENT gt( ELEMENT o1, ELEMENT o2 ){   return btoe(o1>o2); }
ELEMENT le( ELEMENT o1, ELEMENT o2 ){   return btoe(o1<=o2); }
ELEMENT ge( ELEMENT o1, ELEMENT o2 ){   return btoe(o1>=o2); }
ELEMENT equ( ELEMENT o1, ELEMENT o2 ){   return btoe(o1 == o2); }
ELEMENT ne( ELEMENT o1, ELEMENT o2 ){   return btoe(o1!=o2); }


void  blanks( FILE* f, int n ){
   while( n-- > 0 )   fprintf(f,"  ");
}

void  treeprint( FILE* f, ELEMENT n, int indent ){
   ELEMENT  termv = term(n);

   fprintf(f,"%6dN :  ", index(n,_Node) );
   blanks(f,indent);
   switch( sort(termv) ){

   case _Ident:           fprintf(f,"\"%s\"\n", etoa(idtos(termv)) ); break;
   case _Int:             fprintf(f,"value: %d\n", etoi(termv) );break;
   case _Bool:            if( etob(termv) ){
                            fprintf(f,"value: true\n");break;
                          } else {
                                  fprintf(f,"value: false\n");break;
                          }
   case _String:          fprintf(f,"\"%s\"\n", etoa(termv) );break;
   case _Program:         fprintf(f,"Program\n"); break;
   case _LetExp:          fprintf(f,"LetExp\n"); break;
   case _FctDeclList:     fprintf(f,"FctDeclList (%d)\n", numsubterms(termv));break;
   case _FctDecl:         fprintf(f,"FctDecl\n");break;
   case _Boolean:         fprintf(f,"Boolean\n");break;
   case _Integer:         fprintf(f,"Integer\n");break;
   case _ParDeclList:     fprintf(f,"ParDeclList (%d)\n", numsubterms(termv));break;
   case _ParDecl:         fprintf(f,"ParDecl\n");break;
   case _PredeclBody:     fprintf(f,"PredeclBody\n");break;
   case _CondExp:         fprintf(f,"CondExp\n"); break;
   case _FctAppl:         fprintf(f,"FctAppl\n"); break;
   case _ExpList:         fprintf(f,"ExpList (%d)\n", numsubterms(termv)); break;
   case _UsedId:          fprintf(f,"UsedId in line %d\n",etoi(term(son(2,n)))); break;
   case _DeclId:          fprintf(f,"DeclId in line %d\n",etoi(term(son(2,n)))); break;
   default:               fprintf(f,"************\n"); break;
   }      
   n = son(1,n);
   while( n != nil() ){
      treeprint(f,n,indent+1);
      n = rbroth(n);
   }
}

extern  ELEMENT yyread( FILE * );
static  FILE  *out;

int  main( int argc, char **argv ){
   ELEMENT termv;
   ELEMENT eval_result;
   char   *filename = argv[1];
   FILE   *fileptr;
   long np;

   int     leng = strlen(filename);
   long    ix, bound;
   
   if( argc == 1 ){
      fprintf(stdout,"Usage: tifl <file>\n");
      return EXIT_FAILURE;
   }
   if( filename[leng-5] != '.' || filename[leng-4] != 't'  ||  filename[leng-3] != 'i' 
      ||  filename[leng-2] != 'f'  ||  filename[leng-1] != 'l'){
      fprintf(stdout,"tifl: input file does not end with \".tifl\"\n");
      return EXIT_FAILURE;
   }
   fprintf(stdout,"+++ reading\n");
   if( !( fileptr = fopen(filename,"r") ) ){
      fprintf(stdout,"tifl: Can't open file %s for input\n",filename);
      return EXIT_FAILURE;
   }
   termv = yyread( fileptr );

   if( termv == nil() ){
      fprintf(stdout,"tifl: fatal read error: exit\n",filename);
      return EXIT_FAILURE;
   }
   fprintf(stdout,"+++ structure construction\n");

   init_stacks();

   if(  init_tifl( termv )  ){
      fprintf(stdout,"+++ statistics\n");
      fprintf(stdout,"number of nodes:       %d\n", number(_Node) );
      fprintf(stdout,"number of idents:      %d\n", number(_Ident) );
      fprintf(stdout,"number of FctDecls:  %d\n", number(_FctDecl_) );

      filename[leng-5] = '.';
      filename[leng-4] = 't';
      filename[leng-3] = 'r';
      filename[leng-2] = 'e';
      filename[leng-1] = 'e';
      filename[leng] = '\0';
      if( !( out = fopen(filename,"w") ) ){
         fprintf(stdout,"tifl: Can't open file %s for output\n",filename);
         return EXIT_FAILURE;
      }
      treeprint( out, root(), 0 );

      /*---------------------------------------
        check if there are enough stacks available
        ---------------------------------------*/

      if ((np = number(_ParDecl_)) > MAX_NUM_PAR)
         { 
          printf("\n!!! RUNTIME error: too many parameters\n");
          getchar();
          return EXIT_FAILURE;
         }

      printf("\n+++ TIFL-Interpreter started!\n");
      printf("+++ stack version\n");
      printf("\ninput program evaluates to :  ");

      eval_result =  eval( son(1,root()) );
      if( sort(eval_result) == _Int )
        {
              printf("%d\n", etoi(eval_result) );
        }
      else
        {
         if( etob(eval_result) )
           {
            printf("true\n");
           }
         else
           {
            printf("false\n");
           }
        }

   }
   return EXIT_SUCCESS;
}

\end{verbatim}

