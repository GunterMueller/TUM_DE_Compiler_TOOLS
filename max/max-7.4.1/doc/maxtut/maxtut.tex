\documentstyle[12pt,titlepage,psfig]{report}

\evensidemargin0cm
\oddsidemargin0cm
\topmargin0cm

\textwidth16.5cm
\textheight21cm

\sloppy

\begin{document}
\begin{titlepage}

\vspace*{4cm}

\centerline{\huge\bf The MAX System} 

\vspace*{.5cm}

\centerline{\huge\bf A Tutorial Introduction}

\vspace*{2cm}

\centerline{ {\em Arnd Poetzsch-Heffter}\footnote{poetzsch@informatik.tu-muenchen.de} }
\centerline{ {\em Thomas Eisenbarth}\footnote{eisenbar@informatik.tu-muenchen.de} }

\vspace*{1cm}

\centerline{Institut f\"ur Informatik}
\centerline{Technische Universit\"at}
\centerline{M\"unchen}

\vspace*{2cm}

\centerline{\today}

\end{titlepage}


\tableofcontents

\chapter{Introduction}

This tutorial gives an introduction into the first version of the
MAX\footnote{MAX stands for {\bf M}unich {\bf A}ttribution system for
UNI{\bf X} environments.} system, a system to support 
programming language specification and implementation. The MAX 
system is based on an
entirely formal framework that properly generalizes attribute grammar like
frameworks. In addition to other features, the framework
\begin{itemize}
\item
provides free access to the syntax tree, thereby enabling the inspection
of distant attribute occurrences;
\item
allows attributes to have tree nodes as values; so
we can compute and represent global relations between distant parts of
the syntax tree, like the relation between applications and declarations;
\item
enables the formulation of context conditions by first--order predicate
formulae;
\item
provides a simple and purely functional interface between semantic analysis
and later tasks of language processing; e.g.~it provides an excellent basis
for recursively defined interpreters.
\end{itemize}
The notable aspect of the second feature is that it allows to define additional
edges in the syntax tree, which is very useful to represent identification
and type information.

Even though the MAX system was developped for programming
language specification and implementation, MAX is designed in a way that 
it can be used in many other applications. Given a specification,
the MAX system generates an analysor
that takes an abstract syntax term as input and produces a 
rich data structure based on the attributed syntax tree. This data
structure can be accessed through a powerful functional interface so 
that it can be directly used for further processing tasks.


The following chapter gives an introduction into the specification language
of MAX. The specification language allows to define abstract syntax
by order--sorted terms, and attributes, recursive functions, as well as
context conditions.
Chapter 3 explains the interface between MAX specifications and 
C--programs and shows how the MAX system is used in a UNIX environment.
The appendix contains the complete sources of the example specification,
we used throughout the tutorial to illustrate the explanations.


\vspace{1.5cm}

The MAX system is still under development. This tutorial attempts to 
help people in using the version 1.0 of the MAX system. It has to be
understood as an intermediate report. The authors thank in advance
for any questions,
corrections, comments, and suggestions concerning the system or the
tutorial. Thanks for support so far, go to J.~Eickel, W.~Heinrich,
A.~Liebl, S.~Schreiber, W.~Schreiber, U.~Vollath, H.~Wittner. 


\chapter{The MAX Specification Language}
The MAX system provides its own specification language. This chapter gives an introduction to the use and features of this language. It uses TIFL, a TIny Functional Language, as running example. TIFL allows the (nested) declaration of functions and their application (we will describe it in greater detail later on).\\
A specification according to the MAX system consists of two main parts:
\begin{itemize}
\item The definition of the abstract syntax (cf. section 2.1)
\item A sequence of specification elements:

  \begin{itemize}
  \item Attribute definitions (cf. section 2.3.4)\\
(more generally: a set of mutual recursive attribute definitions)
  \item Function and predicate definitions (cf. section 2.3.5)\\
(more generally: a set of mutual recursive function/predicate definitions)  
  \item Context Condition declarations (cf. section 2.3.6)

  \end{itemize}

\end{itemize}
\noindent
A specification element may only use those attributes, functions and predicates being declared before it in the sequence.\\

\section{Context-Free Syntax}

In contrast to many older attribute grammar systems, the MAX system is
based on abstract syntax instead of concrete syntax. The user has to
specify concrete syntax and the transformation from concrete into
abstract syntax by other tools. This modularization allows more 
flexibility (e.g. MAX can be used as well to attribute any trees 
constructed in C programs) and makes it possible to get rid of all
parsing influenced grammar overhead: The abstract syntax can be taylored
to the needs of attribution and later tasks of language processing.
\noindent
The interface between the parser and the MAX system is described by the 
abstract synax; i.e. the parser provides for each program being correct
with respect to the concrete syntax (defined by scanner and parser 
specifications) an abstract syntax term (referred to 
throughout this report as the "program term"). The attribution phase
produced from a specification gets this order-sorted term as input.

\subsection{How to Define Abstract Syntax}

The abstract syntax defines a data type of order-sorted terms (see 
below). In the following, we give the abstract syntax of our 
example language TIFL:
\begin{verbatim}
Program      (  Exp  )
Exp          =  LetExp  |  Int  |  Bool  |  CondExp  |  FctAppl  |  UsedId
LetExp       (  FctDeclList  Exp  )
FctDeclList  *  FctDecl
FctDecl      (  Type  DeclId  ParDeclList  Body  )
Type         =  Boolean  |  Integer
Boolean      ()
Integer      ()
ParDeclList  *  ParDecl
ParDecl      (  Type  DeclId  )
Body         =  Exp  |  PredeclBody
PredeclBody  () 
CondExp      (  Exp  Exp  Exp  )
FctAppl      (  UsedId  ExpList  )
ExpList      *  Exp
UsedId       (  Ident  LineNo  )
DeclId       (  Ident  LineNo  )


Decl         =  FctDecl |  ParDecl
Scope        =  FctDecl |  LetExp
LineNo       =  Int
\end{verbatim}
  
\noindent
Three kinds of productions are used:
\begin{itemize}
\item tuple  ( denoted by  "(" ... ")" )
\item list  ( denoted by  "* ..." )
\item class   ( denoted by  " ... $\mid$ ... " ) 
\end{itemize}
productions.
Each production defines one sort, denoted by the name on the left-hand side of the production. Tuple productions are denoted by enclosing the right-hand side in parentheses. What might be perhaps a bit unusual, is that we use the same name for the tuple sort and the corresponding constructor function. List productions are denoted by an $\ast$ - symbol. Lists are constructed using empty list constructors (e.g. the empty formal parameters declaration list is denoted by ParDeclList()) and predefined polymorphic functions (cf. section 3.2). The class productions can be understood as defining a sort that is the "union" of the subsorts on the left-hand side. The sorts Int, Bool, Ident are predefined.\\
A TIFL program is just an expression. An expression is either a LetExp ( declaration(s) of one or more functions followed by an expression, in which these functions can be used ), a constant,
a conditional expression (IF ... THEN ... ELSE ... FI), a function application or simply the value of a parameter. There are two types in TIFL, "Integer"
and "Boolean". The following example computes the sum of 1 and 2:
\begin{verbatim}
/* declaration of function "sum"
   taking two integer parameters and yielding an integer */

funct sum = (integer a, integer b) : integer

      a + b
in

/* "in" indicates the end of the declarations;
   now sum is called with parameters "1","2"  */

sum(1,2)
\end{verbatim}
The "+" operation and some other arithmetic and boolean operations are 
predefined.

\subsection{How to Handle MAX Terms}

In addition to the sorts of the abstract syntax, there can be auxiliary sort definitions (e.g the sort "Scope" or the sort to describe
environments as shown in section 3.4.1). In general, we call the
elements of the user--defined sorts {\em MAX terms} throughout this 
report.
\noindent
Mainly, there are two kinds of functions to manipulate MAX terms:
the constructors for user-defined tuple and list terms and the 
predefined functions such as the polymorphic list handling or the various
conversion functions constituting the interface of MAX to C. These 
two families of functions together with the MAX terms form the data 
type of order-sorted terms. In the following, we first give a few 
lines from the 
parser specification (in YACC-style) and then the abstract syntax
term of 
the "sum" example to illustrate the handling of MAX terms. 
\noindent
The YACC--example shows how terms
are constructed using the constructors \verb/FctAppl/, \verb/UsedId/,
the empty list constructor \verb/ExpList/, and the some predefined
functions:

\begin{verbatim}
funcappl  :
      ident oparsy cparsy
      {  $$  =  FctAppl( UsedId( $1.val,  $1.ln ),  ExpList() ); }
;
funcappl  :
      ident oparsy actparlist cparsy
      {  $$  =  FctAppl( UsedId( $1.val,  $1.ln ),  $3 ); }
;
actparlist :
      expr
      {  $$  =  appback( ExpList(),  $1 ); }
;
actparlist : 
      actparlist comma expr
      {  $$  =  appback( $1,  $3 ); }
;
relation :
      sexpr equ sexpr
      {  $$  =  FctAppl( UsedId( stoid(atoe( "equ" )),  itoe( line )),
                          appback( appback( ExpList(),  $1 ),  $3 ) ); }

\end{verbatim}
The above lines are taken from the YACC-specification for TIFL and show how the
abstract syntax term for a function application is constructed.
If the actual parameter list is empty, the constructor FctAppl is 
applied to the UsedId (the identifier and line number is returned by 
the scanner) and an empty expression list, denoted by the empty list
constructor of sort ExprList. The following rules show how the actual
parameter list is constructed in case it is not empty. Since most of the 
used functions are self-explanatory, we give only 
their functionality and a short description:


\begin{verbatim} 
appback(List,Elem):List      /* appends Elem at the end of List */
itoe(long):Int               /* converts C-int to MAX-Int */
atoe(char*):String           /* converts C-string to MAX-String */
stoid(String):Ident          /* converts String to Ident (within MAX) */

\end{verbatim}	      

\pagebreak
\noindent
The abstract syntax term of the "sum" example looks like this:
\begin{verbatim}
               LetExp(
                      FctDeclList(
                           FctDecl( Integer(),
                                    DeclId("sum",1),
                                    ParDeclList(
                                         ParDecl( Integer(),
                                                  DeclId("a",1)
                                                ),
                                         ParDecl( Integer(),
                                                  DeclId("b",1)
                                                ) 
                                               ),
                                    FctAppl( UsedId("add",2),
                                             ExpList( 
                                                UsedId("a",2),
                                                UsedId("b",2)
                                                    )
                                           ) 
                                  )
                                 ), 
                          FctAppl( UsedId("sum",4),
                                   ExpList(
                                      1,
                                      2
                                          )
                                 )
                      )

\end{verbatim}
To get the program term, we embed the terms corresponding to 
user programs into a standard environment that contains definitions
for the predeclared operations of TIFL such as "+".

\section{From Terms to Trees}
\medskip
If we only have the order-sorted term representation of programs, we cannot express global relations between distant parts of a program, as e.g. the function that yields for each occurrence of an identifier the corresponding declaration. To overcome this problem, we enrich each program term by the set of its occurrences or, as we call it, its nodes. The union of the set of the MAX terms and the set of the occurrences together with an extra element "nil" (to make functions total) forms the {\em "MAX universe"}. The MAX system maps a given program in term representation into the following syntax tree representation:
\begin{itemize}
\item the set of its nodes;
\item functions yielding the father of a node (fath), the ith son (son), the left and right brother (lbroth,rbroth), and a constant yielding the root; in cases where these functions are not defined they yield nil();
\item a function term yielding for each node the corresponding order-sorted term; in particular, term applied to a leaf node provides the terminal value of the leaf (Ident,Int,...).
\end{itemize}
In the following, the term "syntax tree" always refers to this representation. The sorting on nodes can be imported from the corresponding terms: for example, let n be a node such that the corresponding term is of sort FctDeclList; then we say n is of node sort FctDeclList@.\\
As the construction of the described syntax tree representation does not need any information except the abstract syntax, it has no counterpart in the specification.
\\

\psfig{figure=treepic.eps}

%-----------------------------------------------------------------------------
\section{Attributes, Functions and Context Conditions}

So far, we have the syntax tree representation of a given input program. To perform tasks like semantical analysis (e.g. checking context conditions) and the preparation for code generation, the MAX system provides an applicative specification language. \\  
This section gives a general introduction to this specification language. We start by presenting the basic concepts of formulae, expressions and patterns and then proceed with attributes, functions (and predicates) and context conditions.
\subsection{Formulae}
A formula consists of either of the following:
\begin{itemize}
\item predicate application
\item formula "or" formula (denoted "\&\&")
\item formula "and" formula (denoted "$\mid$$\mid$")
\item formula "$\Rightarrow$" formula (denoted "-$\rangle$")
\item "not" formula (denoted "!")
\end{itemize}

\subsection{Expressions}
An expression consists of either of the following:
\begin{itemize}
\item constant
\item name
\item function application
\item If-expression: the skeleton:\\
                IF \{formula/pattern : expression\} *\\
                ELSE expression 
\item Let-expression (names the value of expression\_1 by "name", then evaluates expression\_2):\\
                     LET name == expression\_1 :\\
                     expression\_2
\item String-expression (concatenate a given list of strings)
\end{itemize}


\subsection{Patterns}

Patterns are used for two different tasks: in context conditions, they
allow to quantify over variables matching a pattern of the syntax tree;
in conditional expressions, they express the condition that a given
set of identifiers matches a pattern. In addition to that,
they provide a binding mechanism for identifiers to node occurrences. 
A pattern can consist of three classes of items:
\begin{itemize}
\item identifiers (each binding a node occurrence to a name by its position)
\item asterisks "$\ast$" (each representing zero or more node occurrences at its position)
\item underscores "\_" (each representing exactly one node occurrence at its position)
\end{itemize}

\noindent
Of course, a pattern can contain subpatterns to bind e.g. several node occurrences of the first son of the bound node.
%---------------------------------------------------------------------------------------------------------------------
\subsection{Attributes}
The skeleton of an attribute definition looks like this:
\begin{verbatim}
ATT name ( parameter ) result_type :
    expression
\end{verbatim}
where there is exactly one parameter of a node sort.
\noindent
From an abstract point of view, an attribute is just a unary function having a node sort as domain. But, as node sorts are finite, we can use specialized implementation techniques to handle attributes, namely compute all the attribute values once and store them for later use. Besides this, we keep the differences between attributes and functions as small as possible. This is as well documented by the syntax: the head of an attribute declaration looks like the head of a function declaration (especially, inherited and synthesized attributes are not distinguished), and an attribute application is denoted like a function application (but can be denoted using the usual dotted notation, too : e.g. "ID.decl" instead of "decl(ID)").\\
Here are three attribute declarations; the first one provides for each node the enclosing scope by stepping upwards in the syntax tree ($\rightarrow$ encl\_scope(fath(.))) until either a function declaration (FctDecl) node or a LetExp node (list of declarations followed by an expression) is reached. The second attribute yields for an applied occurrence of an identifier (here: functions or their parameters) the corresponding declaration node by issuing a global lookup starting in the enclosing scope of the used identifier (lookup described in 2.3.5), giving a slight impression of how to use node valued attributes to get information from distant parts of the syntax tree; in the type attribute we have an example for a pattern used as condition: assuming that the expression node E is of sort LetExp@, the pattern "$\langle$\_,EX$\rangle$" binds EX to the second son of the LetExp@, which is the expression to be evaluated with the declaration(s) (first son of E, represented by "\_") being valid.
 In the cases where E is a constant, the type is simply Integer() or Boolean(); for conditional expressions, the type of the then - branch is returned. If E happens to be a function application, the type is determined out of the UsedId (the name of the function); for an UsedId in a TIFL program there are two possibilities: it can be the applied occurrence of a function (the result type of the corresponding function declaration node is returned) or of a parameter (taking the type of the corresponding parameter declaration node): 
\begin{verbatim}

ATT encl_scope( Node N ) Scope@ :
   IF  is[ N, _FctDecl@ ]            : N
   |   is[ N, _LetExp@ ]             : N 
   ELSE                     encl_scope( fath( N ) )

ATT decl( UsedId@ UID ) DeclId@ :
   lookup( id(UID), encl_scope(UID) )

ATT type( Exp@ E ) Type :
   IF LetExp@<_,EX> E   :  type( EX )
   |  Int@ E            :  Integer()
   |  Bool@ E   	:  Boolean()
   |  CondExp@<_,E2,_> E:  type( E2 ) 
   |  FctAppl@<UID,_> E :  type( UID ) 
   |  UsedId@ E         :  term( son( 1, fath(decl(E)) ) )
   ELSE nil()

\end{verbatim}
%--------------------------------------------------------
\subsection{Functions and Predicates}
%-------------------------------------------------------

The skeleton of an function definition looks like this:
\begin{verbatim}
FCT name ( parameters ) result_type :
    expression
\end{verbatim}
Here is an example with three function declarations (explanations below):
\begin{verbatim}

FCT first_decl( Scope@ S ) Decl@:
   IF  LetExp@<<FCTD,*>,_> S     :   FCTD
   |   FctDecl@<_,_,<PARD,*>,_> S:   PARD
   ELSE nil()

FCT lookup_list( Ident ID, Decl@ D ) DeclId@ :
   IF  ID = id( son(2,D) )  :  son(2,D)   ELSE  lookup_list( ID, rbroth(D) )

FCT lookup( Ident ID, Scope@ S ) DeclId@ :
   LET  DID == lookup_list(ID,first_decl(S)) :
   IF   DID # nil() :  DID  ELSE  lookup( ID, encl_scope(fath(S)) )

\end{verbatim}
first\_decl takes yields the first declaration node in a given Scope@ S, which can
be a function declaration (FctDecl@) if S happens to be a LetExp@, or a parameter
declaration (ParDecl@), if S is a FctDecl@.\\
lookup\_list examines a list of declarations (either FctDeclList or ParDeclList) for the occurrence of a given identifier. It makes use of the similar structure (concerning Type and DeclId) of the elements in both kinds of declaration lists.\\
lookup defines a global lookup given an Ident and the directly enclosing scope of an applied occurrence of it by stepping upwards in the syntax tree from enclosing scope to enclosing scope examining the corresponding kind of declaration list (by calling lookup\_list()) for the occurrence of the given identifier. The explicit nil()-checks (termination cases) in all functions have been omitted, since functions are strict with respect to nil().\\
\noindent
Predicates are functions with boolean result being not strict with respect to nil(). The skeleton of a predicate definition looks like this:
\begin{verbatim}
PRD name [ parameters ]:
    formula
\end{verbatim}
\noindent
Parameters are enclosed in "[" "]" brackets and precedence is indicated by "\{" "\}".
The following example gives an impression what a predicate looks like:
\begin{verbatim}

PRD par_type_check[ ParDecl@ P, Exp@ E ]:
   term(son(1,P)) = type(E)
 && { rbroth(P) # nil()  ->  par_type_check[ rbroth(P), rbroth(E) ] }

\end{verbatim}
par\_type\_check is "true", if the types of the actual parameters of a function application match the types of its formal parameters. There is no "bottom"(nil()) result for predicates, they yield either "true" or "false". In par\_type\_check it must be for sure, that the number of actual/formal parameters matches!

%----------------------------------------------------------
\subsection{Context Conditions}
The skeleton of a context condition looks like this:
\begin{verbatim}
CND pattern : formula
| expression
\end{verbatim}
where expression has to be a string-expression to print an error message if the condition is violated.\\ 
Our framework allows to formulate context conditions in a very natural, elegant and convenient way based on predicate logic. Especially during language design time, such high-level executable specifications of context conditions proved to be very useful. To show what context conditions look like, we give a few examples below:
\begin{verbatim}

CND UsedId@ UID         : decl(UID) # nil()
| `"### LINE " ln(UID) ": identifier \"" idtos(id(UID)) "\" not declared\n"'

CND ParDeclList@<*,<_,ID1>,*,<_,ID2>,*>  : id(ID1) # id(ID2)
| `"### LINE(S) " ln(ID1) "/" ln(ID2) ": parameter \"" idtos(id(ID1))
  "\" doubly declared\n"'

CND FctAppl@<UID,EL> :  numsons(EL) = numsons(son(3,fath(decl(UID))))
| `"### LINE " ln(UID) ": incorrect number of parameters in call of"
  " function "idtos(id(UID))"\n"'

\end{verbatim}
In context conditions patterns are used as universal quantifications, so we can read the context conditions as first-order formulae: For all applied occurrences of identifiers the declaration attribute must be defined (different from nil()); two different parameter declarations in the same declaration list must have distinct identifiers; for all function applications the number of actual parameters must match the number of formal parameters.\\
In the MAX system, the user can attach an error message to each context condition that may refer to the variables in the pattern. The string expression
(here: concatenation operators " ` " and " ' ") following the "$\mid$" is evaluated if the context condition is violated.




   	



\chapter{Using the MAX System}

This chapter explains how to use the MAX system. After showing how to call
the generated evaluator, we describe the interface between MAX and C.
Finally, we give the specification of an interpreter for TIFL in order to
illustrate how MAX can interact with C-programs and thus how it can
be used to stepwise refine language specifications.


\section{Getting started}
         
This section introduces to the application of the MAX system; first, we present two figures to illustrate the handling of MAX (explanations below):
\\

%------  hier usepic.eps
\psfig{figure=usepic.eps}

\noindent
To create a compiler for a specific language the following steps have to be taken:
\begin{itemize}
\item The first task consists of designing the abstract syntax of the language ($\langle$language\_spec.m$\rangle$ contains the specification);
\item the second step is providing a parser for the language (e.g. using a tool like YACC);
\item third: a main program to control the phases (parsing, semantical analysis etc.) has to be written ($\langle$language.c$\rangle$).
\end{itemize}
\noindent
The skeleton of the main program looks like this:
\begin{verbatim}
#include "<language_spec.h>"   /*  establishing the interface MAX <-> C  */
#include "..."

/* function definitions  */

int main(int argc, char **argv)
{
  ELEMENT termvar;

   ...

  /*  call of parser  */

  termvar = <program term>;

  /*  structure construction  */

  init_<language>(termvar);

  /*  evaluates the ``struc'' specification */  

}
\end{verbatim} 
\noindent
The call of init\_$\langle$language$\rangle$ performs the construction of the syntax tree, the attribution and checking of context conditions according to the specification in $\langle$language\_spec.m$\rangle$. 
\noindent
MAX generates two files from a specification: $\langle$language\_spec.h$\rangle$ and $\langle$language\_spec.c$\rangle$. Together with the parser module, the main program and max\_std.o (contains predefined constants and functions) these files form the input to the C-compiler, which generates the executable compiler (named $\langle$language$\rangle$).\\

  
\section{Standard Functions}
This section presents the predefined standard functions of the MAX system interface to C.\\
For each function we give the functionality and a short description; if their obvious meaning is not defined, they return nil().\\
"ELEMENT" is the C-type for all elements of the MAX universe (cf. section 2.2) (including nil()). These functions can be used from within C by including the $\langle$language\_spec.h$\rangle$ generated by the MAX system (cf. Appendix). This file also contains the definition of the sort constants; in addition to
 the predefined sorts (like Ident,String,Int,Bool,...) there are sort constants for each sort defined in the 
abstract syntax. These constants can be used e.g. to determine, whether a given ELEMENT is of a particular sort
( $\rightarrow$ is(ELEMENT,Sort), where Sort is a sort constant). The sorting on nodes is imported from the corresponding terms (cf. section 2.2).\\
%--------------conversion  C -> MAX
\subsection{Conversion C $\rightarrow$ MAX}
The functions below convert the C-types long,char etc. to ELEMENT.
\begin{verbatim}
ELEMENT  btoe(long)         /* converts boolean value(long) to ELEMENT */       
ELEMENT  ctoe(char)         /* converts char to ELEMENT  */
ELEMENT  itoe(long)         /* converts long int to ELEMENT */
ELEMENT  atoe(char*)        /* converts string(char*) to ELEMENT */
\end{verbatim}
       
%--------------conversion  MAX -> C
\subsection{Conversion MAX $\rightarrow$ C}
The functions below convert ELEMENT to C-types.
\begin{verbatim}
long     etob(ELEMENT)      /* converts ELEMENT to boolean value (long)*/
char     etoc(ELEMENT)      /* converts ELEMENT to char */
long     etoi(ELEMENT)      /* converts ELEMENT to long int */ 
char*    etoa(ELEMENT)      /* converts ELEMENT to string (char*) */
\end{verbatim}
       
%--------------conversion  MAX -> MAX
\subsection{Conversion within MAX}
There are two functions for conversion within the MAX system:
\begin{verbatim}
Ident    stoid(String)       /* converts string to ident */       
String   idtos(Ident)        /* converts ident to string */
\end{verbatim}

%--------------index functions
\subsection{Index Functions}
\begin{verbatim}
ELEMENT  index(ELEMENT,Sort) /* returns index of ELEMENT in Sort */
ELEMENT  element(index,Sort) /* returns ELEMENT with index in Sort */
ELEMENT  number(Sort)        /* returns number of nodes of Sort */
\end{verbatim}

%--------------tree functions
\subsection{Tree Functions}
\begin{verbatim}
ELEMENT  fath(Node)          /* returns father of Node */
ELEMENT  lbroth(Node)        /* returns left brother of Node */
ELEMENT  rbroth(Node)        /* returns right brother of Node */
ELEMENT  son(ith,Node)       /* returns ith son of Node */
ELEMENT  numsons(Node)       /* returns number of sons of Node */
\end{verbatim}

%--------------general term functions
\subsection{General Term Functions}
\begin{verbatim}
ELEMENT  subterm(ith,Term)   /* returns ith subterm of Term */
ELEMENT  numsubterms(Term)   /* returns number of subterms of Term */
\end{verbatim}
%--------------polymorphic list handling (special term functions -> lists)
\subsection{Polymorphic List Handling Functions}
\begin{verbatim}
ELEMENT  front(List)             /* returns first ELEMENT of List */
ELEMENT  back(List)              /* returns List without first ELEMENT */
ELEMENT  appfront(ELEMENT,List)  /* appends ELEMENT in front of List */
ELEMENT  appback(List,ELEMENT)   /* appends ELEMENT at the end of List */
ELEMENT  conc(List,List)         /* concatenates two Lists */
\end{verbatim}
%--------------additional functions
\subsection{Additional Functions}
\begin{verbatim}
ELEMENT  term(Node)    /* returns the corresponding order-sorted term */
ELEMENT  sort(ELEMENT)           /* returns sort of ELEMENT */

long     eq(ELEMENT,ELEMENT)     /* equality predicate */    
long     is(ELEMENT,Sort)        /* sort predicate */
long     desc(Node,Node)         /* descendant predicate */            
\end{verbatim}

\section{External Functions}
For some purposes, it is helpful or necessary to implement functions used in the specification in C. This can be done by declaring the function in the specification and defining it in some C-file. We give an example for this technique below:
\begin{verbatim}
in the specification (tifl_spec.m):

FCT add( Int, Int ) Int

in the main program (tifl.c):

ELEMENT add( ELEMENT o1, ELEMENT o2 ){ return itoe(etoi(o1)+etoi(o2)); }

\end{verbatim}

\section{Implementation of an Interpreter for TIFL}
This section deals with the specification and implementation of an interpreter for TIFL. We present two different implementations, one using the environment technique, the other operating with a stack mechanism (implemented in C).
\subsection{The Interpreter using Environments}
First, we give the additional sort declarations in the abstract syntax and the specification of the interpreter itsself (explanations below, complete sources in the appendix):
\begin{verbatim}
Env          *  Pair
Pair         (  ParDecl@  Value  )
Value        =  Bool | Int


FCT  env_lookup( ParDecl@ PN, Env E ) Value:
   IF  subterm(1,subterm(1,E)) = PN  :  subterm(2,subterm(1,E))
                                  ELSE  env_lookup( PN, back(E) )

FCT  enter_pars( ParDecl@ PN, Exp@ EN, Env E, Env EVALENV ) Env:
   LET  V == eval( EN, EVALENV )   :
   IF   rbroth(PN) = nil()   :	appfront( Pair(PN,V), E )
   ELSE   enter_pars( rbroth(PN), rbroth(EN), 
                        appfront( Pair(PN,V), E ), EVALENV )
\end{verbatim}
\pagebreak
\begin{verbatim}

FCT  eval ( Exp@ X, Env E ) Value:
   IF  LetExp@<_,BD>        X:  eval( BD, E )
    |  Int@                 X:  term(X)
    |  Bool@                X:  term(X)
    |  CondExp@<E1,E2,E3>   X:  IF eval(E1,E) = true() : eval(E2,E)
                                ELSE                     eval(E3,E) 
    |  UsedId@              X:  env_lookup( fath(decl(X)), E )
    |  FctAppl@<UID,<>>     X:  eval( son(4, UID.decl.fath ), E )
    |  FctAppl@<UID,<E1,*>> X:  LET  FCTDCL ==  UID.decl.fath  :
	IF  FctDecl@<_,<IDN,_>,_,PredeclBody@> FCTDCL:  
               eval_predecl( term(IDN), E1, E )
	ELSE   eval( son(4,FCTDCL), enter_pars(son(1,son(3,FCTDCL)),E1,E,E)  )
   ELSE nil()

\end{verbatim}
Environments consist of a list of pairs, each constructed of a parameter (its declaration occurrence) and the corresponding actual value (which is either a Bool or Int).
eval yields the value of the expression in the given environment.It works together with three auxiliary functions:
\begin{itemize}
\item env\_lookup() : returns the value of given parameter in a given environment;
\item eval\_predecl() : evaluates the application of predeclared functions,e.g. "+" (cf. appendix);
\item enter\_pars() : inserts the actual values of the formal parameters of a given function in a given environment.
\end{itemize}
\noindent
The interpreter is fully specified within the MAX system; it is called by the main program:
\begin{verbatim}

ELEMENT eval_result;            
eval_result =  eval( son(1,root()), Env() );
                
\end{verbatim}

\subsection{The Interpreter using the Stack Mechanism}
In the following we give the necessary sort declarations, the specification of the interpreter using the stack mechanism and the related C-implementation (explanations below, complete sources in the appendix):
\begin{verbatim}

Value        =  Bool | Int

FCT  par_lookup( ParDecl@ ) Value

FCT  eval_userfunc( ParDecl@, Exp@, Exp@ ) Value

FCT  eval ( Exp@ X ) Value:
   IF  LetExp@<_,BD>        X:  eval( BD )
    |  Int@                 X:  term(X)
    |  Bool@                X:  term(X)
    |  CondExp@<E1,E2,E3>   X:  IF eval(E1) = true() :  eval(E2)  
                                                  ELSE  eval(E3) 
    |  UsedId@              X:  par_lookup( fath(decl(X)) )
    |  FctAppl@<UID,<>>     X:  eval( son(4, UID.decl.fath ) )
    |  FctAppl@<UID,<E1,*>> X:  LET  FCTDCL ==  UID.decl.fath  :
	IF  FctDecl@<_,<IDN,_>,_,PredeclBody@> FCTDCL:  
               eval_predecl( term(IDN), E1 )
	ELSE   eval_userfunc( son(1,son(3,FCTDCL)), E1, son(4,FCTDCL) )
   ELSE nil()

\end{verbatim}
The stack implementation in C looks like this:
\begin{verbatim}

/*----------------------------------------------------------------------

  stack.c           
                   
  --------------------------------------------------------------------*/

#define MAX_NUM_PAR 50
#define MAX_DEPTH 100      

ELEMENT stacks[MAX_NUM_PAR][MAX_DEPTH];
int sp[MAX_NUM_PAR];


void pop_params( ELEMENT Parnode )
{
  while( Parnode != nil() )
       {
        sp[index(Parnode,_ParDecl_)]--;
        Parnode = rbroth(Parnode);
       }
}


void push_params( ELEMENT Parnode, ELEMENT Expnode )
{
  long indx;
  ELEMENT n;
  ELEMENT to_push[MAX_NUM_PAR];

  /* first, evaluate all actual parameter expressions;
     then push them on the stacks
  */

  n = Parnode;
  while( n != nil() )
       {
        to_push[index( n, _ParDecl_ )] = eval( Expnode );
        Expnode = rbroth( Expnode );
        n = rbroth( n );
       }


  while( Parnode != nil() )
       {
        indx = index(Parnode, _ParDecl_);

        if( sp[ indx ] > MAX_DEPTH )
          {
           printf("\n!!! RUNTIME error: stack overflow\n");
           return;
          }

         sp[ indx ]++;
         stacks[ indx ][ sp[indx] ] = to_push[ index(Parnode,_ParDecl_) ];
         Parnode = rbroth(Parnode);
       }
}
\end{verbatim}
\pagebreak
\begin{verbatim}
ELEMENT par_lookup( ELEMENT Parnode )
{
  long indx = index(Parnode, _ParDecl_);
  return stacks[ indx ][ sp[ indx ] ];
}


ELEMENT eval_userfunc( ELEMENT Parnode, ELEMENT Expnode, ELEMENT Body )
{
   ELEMENT result;

   push_params( Parnode, Expnode );
   result = eval(Body);    
   pop_params( Parnode );
   return result; 
}

\end{verbatim}
Since we operate with stacks now, we do not need any environments. The functions used are:
\begin{itemize}
\item par\_lookup() : returns the actual value of a formal parameter (implemented in C);
\item eval\_userfunc() : evaluates a given user-defined function with given actual parameters (implemented in C);
\item eval\_predecl() : same as above (besides the fact, that environments are no longer needed);
\item eval() : same as above (besides the fact, that environments are no longer needed).
\end{itemize}
The stack mechanism itself is implemented in a straightforward way. Each parameter has its own stack which is accessed by the following functions:
\begin{itemize}
\item eval\_userfunc() : pushes the actual parameters on the corresponding stacks($\rightarrow$ push\_params())($\rightarrow$ push\_params()), evaluates the body of the function, pops the parameters($\rightarrow$ pop\_params()) and returns the result of the evaluation;
\item par\_lookup() : returns top of stack of given parameter.
\end{itemize}
The application in the main program looks like this:
\pagebreak
\begin{verbatim}

ELEMENT eval_result;

/*---------------------------------------
check if there are enough stacks available
---------------------------------------*/
if ((np = number(_ParDecl_)) > MAX_NUM_PAR)
   { 
    printf("\n!!! RUNTIME error: too many parameters\n");
    return EXIT_FAILURE;
   }

eval_result =  eval( son(1,root()) );

\end{verbatim}



\chapter{Conclusions}

The turotial presented an introduction to MAX and showed how the
system can be used to specify programming languages. The main advantages 
of the system compared to systems like MUG or GAG \cite{GAG} are (1) the
more general attribution constructs and (2) the flexible
functional back--end that can be used for further tasks of language
processing or refinement of specifications. Besides being a very
useful tool in its own rights, the system is meant to
serve as a kernel for even higher--level specification facilities (see 
e.g.~\cite{PH:diss}, \cite{PH:identif}) and for systems allowing
the specification of dynamic semantics and logic of programming 
languages (work in this area is under way; contact the first author)
in order to generate interpreters, debuggers, and programming environments
that enable queries about program properties.
Readers interested in details about the undelying formal framework are
refered to \cite{PH:maxsec}.

\vspace*{1cm}

\small
\begin{thebibliography}{KHZ82}

\bibitem[KHZ82]{GAG}
U.~Kastens, B.~Hutt, E.~Zimmermann.
\newblock {GAG}: A practical compiler generator.
\newblock Lecture Notes in Computer Science 141, 1982.

\bibitem[PH91]{PH:diss}
A.~Poetzsch-Heffter.
\newblock {\em Context--dependent syntax of programming languages: A new
  specification method and its application}.
\newblock PhD thesis (in German); Technische Universit\"at M\"unchen, 1991.

\bibitem[PH92a]{PH:identif}
A.~Poetzsch-Heffter.
\newblock Identifcation as Programming Language Principle.
\newblock Technical Report TUM--I9223, July 1992;
\newblock Technische Universit\"at M\"unchen, 1992.

\bibitem[PH92b]{PH:maxsec}
A.~Poetzsch-Heffter.
\newblock Programming Language Specification and Prototyping Using the 
MAX System.
\newblock Internal Report, September 1992 (submitted for publication);
\newblock Technische Universit\"at M\"unchen, 1992.

\end{thebibliography}




\appendix
\chapter{The MAX System and its Environment}
The following two figures show the MAX system and its environment:
\\

\psfig{figure=maxpic.eps}
\noindent
We used LEX and YACC to generate the scanner and the parser, but the user is free to choose other tools for these tasks or to code them himself.\\
The MAX system takes the specification (tifl\_spec.m) as input and generates an attribution phase from it (in tifl\_spec.h, tifl\_spec.c).\\
There is a main program to control the phases of the compiler (tifl.c). It scans and parses the input file (according to the specification in tifl\_scan.l and tifl\_pars.y) and then starts the attribution phase ("init\_tifl()"); max\_std.o contains the predefined functions and constants. From these input files
the executable interpreter (tifl) is generated.

%------------------------------------------------------------
% Hier finden sich alle Quellen!
\chapter{The TIFL--Sources}
\section{Predeclared constants and functions (tifl\_spec.h)}
\begin{verbatim}

/*---------------------------------------------------------------------
  predefined functions 
  ---------------------------------------------------------------------*/
/*  The C-type for all elements of the MAX-universe including  nil :       */
#define  ELEMENT            mxi_ELEMENT

/*  The following functions convert C-values to corresponding MAX-elements */
/*  in case of an error, the functions set the global variable conv_error  */
/*  to 0 and yield  nil.                                                   */
#define  conv_error          mxi_conv_error        
#define  btoe(BOOLVAL)       mxi_btoe(__FILE__,__LINE__,BOOLVAL)
#define  ctoe(CHARVAL)       mxi_ctoe(__FILE__,__LINE__,CHARVAL)
#define  itoe(INTVAL)        mxi_itoe(__FILE__,__LINE__,INTVAL)
#define  atoe(STRVAL)        mxi_atoe(__FILE__,__LINE__,STRVAL)
#define  ptoe(POINTVAL)      mxi_ptoe(__FILE__,__LINE__,POINTVAL)

/*  The following functions convert MAX-elements to corresponding C-values */
/*  in case of an error, the functions set the global variable conv_error  */
/*  to 0 and yield  0, '\0', 0, the empty string or NULL.                 */
#define  etob(BOOLELEM)      mxi_etob(__FILE__,__LINE__,BOOLELEM)
#define  etoc(CHARELEM)      mxi_etoc(__FILE__,__LINE__,CHARELEM)
#define  etoi(INTELEM)       mxi_etoi(__FILE__,__LINE__,INTELEM)
#define  etoa(STRELEM)       mxi_etoa(__FILE__,__LINE__,STRELEM)
#define  etop(REFELEM)       mxi_etop(__FILE__,__LINE__,REFELEM)

/*  stoid  enters a string into sort  Ident  containing a finite set of    */
/*  hash-coded strings;  idtos  returns the string of an identifier        */
#define  stoid(STRING)       mxi_stoid(__FILE__,__LINE__,STRING)
#define  idtos(IDENT)        mxi_idtos(__FILE__,__LINE__,IDENT)

/*  sort  returns the sort constant of a defined element; otherwise exits. */
#define  sort(ELEM)          mxi_sort(__FILE__,__LINE__,ELEM)
#define  index(ELEM,SORT)    mxi_index(__FILE__,__LINE__,ELEM,SORT)
#define  element(INDEX,SORT) mxi_element(__FILE__,__LINE__,INDEX,SORT)
#define  number(SORT)        mxi_number(__FILE__,__LINE__,SORT)

/*  The following functions provide the inspection of the tree structure; */
/*  if their obvious meaning is not defined, they return  nil :           */
#define  fath(NODE)          mxi_fath(__FILE__,__LINE__,NODE)
#define  lbroth(NODE)        mxi_lbroth(__FILE__,__LINE__,NODE)
#define  rbroth(NODE)        mxi_rbroth(__FILE__,__LINE__,NODE)
#define  son(ITH,NODE)       mxi_son(__FILE__,__LINE__,ITH,NODE)
#define  numsons(NODE)       mxi_numsons(__FILE__,__LINE__,NODE)

/*  subterm(i,t)  returns the ith subterm of term t;  numsubterms(t)      */
/*  returns the number of subterms t; both functions exit on failure      */
#define  subterm(ITH,TERM)   mxi_subterm(__FILE__,__LINE__,ITH,TERM)
#define  numsubterms(TERM)   mxi_numsubterms(__FILE__,__LINE__,TERM)

/*  The following functions provide the polymorphic handling of lists;    */
/*  they exit on failure :                                                */
#define  front(LIST)         mxi_front(__FILE__,__LINE__,LIST)
#define  back(LIST)          mxi_back(__FILE__,__LINE__,LIST)
#define  appfront(ELEM,LIST) mxi_appfront(__FILE__,__LINE__,ELEM,LIST)
#define  appback(ELEM,LIST)  mxi_appback(__FILE__,__LINE__,ELEM,LIST)
#define  conc(LIST1,LIST2)   mxi_conc(__FILE__,__LINE__,LIST1,LIST2)

/*  The following functions relate nodes to the corresponding terms and   */
/*  points; they exit on failure :                                        */
#define  term(NODE)          mxi_term(__FILE__,__LINE__,NODE)
#define  before(NODE)        mxi_before(__FILE__,__LINE__,NODE)
#define  after(NODE)         mxi_after(__FILE__,__LINE__,NODE)
#define  node(POINT)         mxi_node(__FILE__,__LINE__,POINT)

/*  eq  is the equality predicate;  is  tests whether an element is in    */
/*  the given sort;  desc tests whether node1 is the descendant of node2; */
/*  bef  and  aft  establish the linear order on  points :                */
#define  eq(ELEM1,ELEM2)     mxi_eq(__FILE__,__LINE__,ELEM1,ELEM2)
#define  is(ELEM,SORT)       mxi_is(__FILE__,__LINE__,ELEM,SORT)
#define  desc(N1,N2)         (before(N1)>before(N2) && after(N1)<after(N2))
#define  bef(P1,P2)          (is(P1,_Point) && is(P2,_Point) && P1<P2)
#define  aft(P1,P2)          (is(P1,_Point) && is(P2,_Point) && P1>P2)

/*  The predefined constants are :                                        */
#define  minimalInt()   -16777216L
#define  maximalInt()   16777215L

#define  root()         -939524095L
#define  true()         -268435455L
#define  false()        -268435456L
#define  nil()          -536805377L
/*  The sort constants :                                                  */
/*---------------------------------------------------------------------
  sort constants defined by the abstract syntax
  ---------------------------------------------------------------------*/
#define  _FctDeclList 		 -469696538L	/* -26 */
#define  _FctDeclList_		 -469762022L	/* 26 */
#define  _ParDeclList 		 -469696539L	/* -27 */
#define  _ParDeclList_		 -469762021L	/* 27 */
#define  _ExpList 		 -469696540L	/* -28 */
#define  _ExpList_		 -469762020L	/* 28 */
#define  _Env 		 -469696541L	/* -29 */
#define  _Env_		 -469762019L	/* 29 */
#define  _Program 		 -469696526L	/* -14 */
#define  _Program_		 -469762034L	/* 14 */
#define  _LetExp 		 -469696527L	/* -15 */
#define  _LetExp_		 -469762033L	/* 15 */
#define  _FctDecl 		 -469696528L	/* -16 */
#define  _FctDecl_		 -469762032L	/* 16 */
#define  _Boolean 		 -469696529L	/* -17 */
#define  _Boolean_		 -469762031L	/* 17 */
#define  _Integer 		 -469696530L	/* -18 */
#define  _Integer_		 -469762030L	/* 18 */
#define  _ParDecl 		 -469696531L	/* -19 */
#define  _ParDecl_		 -469762029L	/* 19 */
#define  _PredeclBody 		 -469696532L	/* -20 */
#define  _PredeclBody_		 -469762028L	/* 20 */
#define  _CondExp 		 -469696533L	/* -21 */
#define  _CondExp_		 -469762027L	/* 21 */
#define  _FctAppl 		 -469696534L	/* -22 */
#define  _FctAppl_		 -469762026L	/* 22 */
#define  _UsedId 		 -469696535L	/* -23 */
#define  _UsedId_		 -469762025L	/* 23 */
#define  _DeclId 		 -469696536L	/* -24 */
#define  _DeclId_		 -469762024L	/* 24 */
#define  _Pair 		 -469696537L	/* -25 */
#define  _Pair_		 -469762023L	/* 25 */
#define  _Element 		 -469696542L	/* -30 */
#define  _Element_		 -469762018L	/* 30 */
#define  _Exp 		 -469696543L	/* -31 */
#define  _Exp_		 -469762017L	/* 31 */
#define  _Type 		 -469696544L	/* -32 */
#define  _Type_		 -469762016L	/* 32 */
#define  _Body 		 -469696545L	/* -33 */
#define  _Body_		 -469762015L	/* 33 */
#define  _Decl 		 -469696546L	/* -34 */
#define  _Decl_		 -469762014L	/* 34 */
#define  _Scope 		 -469696547L	/* -35 */
#define  _Scope_		 -469762013L	/* 35 */
#define  _LineNo 		 -469696548L	/* -36 */
#define  _LineNo_		 -469762012L	/* 36 */
#define  _Value 		 -469696549L	/* -37 */
#define  _Value_		 -469762011L	/* 37 */
/*---------------------------------------------------------------------
  predefined sort constants
  ---------------------------------------------------------------------*/
#define  _Point 		 -469762047L	/* 1 */
#define  _Node 		 -469762046L	/* 2 */
#define  _nil 		 -469696513L	/* -1 */
#define  _Term 		 -469696514L	/* -2 */
#define  _NodeSort 		 -469696515L	/* -3 */
#define  _NodeSort_		 -469762045L	/* 3 */
#define  _PredeclSort 		 -469696516L	/* -4 */
#define  _PredeclSort_		 -469762044L	/* 4 */
#define  _ClassSort 		 -469696517L	/* -5 */
#define  _ClassSort_		 -469762043L	/* 5 */
#define  _TupelSort 		 -469696518L	/* -6 */
#define  _TupelSort_		 -469762042L	/* 6 */
#define  _ListSort 		 -469696519L	/* -7 */
#define  _ListSort_		 -469762041L	/* 7 */
#define  _Ident 		 -469696520L	/* -8 */
#define  _Ident_		 -469762040L	/* 8 */
#define  _Bool 		 -469696521L	/* -9 */
#define  _Bool_		 -469762039L	/* 9 */
#define  _Char 		 -469696522L	/* -10 */
#define  _Char_		 -469762038L	/* 10 */
#define  _Int 		 -469696523L	/* -11 */
#define  _Int_		 -469762037L	/* 11 */
#define  _String 		 -469696524L	/* -12 */
#define  _String_		 -469762036L	/* 12 */
#define  _Reference 		 -469696525L	/* -13 */
#define  _Reference_		 -469762035L	/* 13 */

\end{verbatim}


%----  env version   -------
\section{Stack version}
\subsection{Makefile}
\begin{verbatim}
# Makefile for generation of MAX analysers 

# language dependent variables
LANG=tifl

# compiler dependent variables
INCLUDES=
CFLAGS= $(INCLUDES)
CC=cc -Aa 

$(LANG) : y.tab.o  max_std.o   $(LANG)_spec.o  $(LANG).o 
	$(CC) $(CFLAGS) -o $(LANG) y.tab.o  max_std.o \
			   $(LANG)_spec.o $(LANG).o 
	cp $(LANG) ../../bsp

y.tab.o: y.tab.c  lex.yy.c 
	$(CC) -g -c  y.tab.c

y.tab.c: $(LANG)_pars.y  $(LANG)_spec.h
	yacc $(LANG)_pars.y	

lex.yy.c: $(LANG)_scan.l 
	lex $(LANG)_scan.l 	

$(LANG)_spec.o :  $(LANG)_spec.c
	$(CC) $(CFLAGS) -c $(LANG)_spec.c

$(LANG)_spec.c $(LANG)_spec.h :  $(LANG)_spec.m ../../../MAX/src.5/max
	../../../MAX/src.5/max $(LANG)_spec.m

$(LANG).o :  $(LANG).c  $(LANG)_spec.h stack.c 
	$(CC) $(CFLAGS) -c $(LANG).c

\end{verbatim}


\subsection{tifl\_scan.l}
\begin{verbatim}
%{
#define TABLENGTH 3

static int   line   =  1 ;
static int   column =  0 ;

%}
%%

[ ]*               { column += yyleng; }
\n                 { line++; column = 0; }
\t                 { column += TABLENGTH; }
"funct"            { column += yyleng;  return(funcsy); }
"FUNCT"            { column += yyleng;  return(funcsy); }
"if"               { column += yyleng;  yylval.l = itoe( line );return(ifsy); }
"IF"               { column += yyleng;  yylval.l = itoe( line );return(ifsy); }
"then"             { column += yyleng;  return(thensy); }
"THEN"             { column += yyleng;  return(thensy); }
"else"             { column += yyleng;  return(elsesy); }
"ELSE"             { column += yyleng;  return(elsesy); }
"fi"               { column += yyleng;  return(fisy); }
"FI"               { column += yyleng;  return(fisy); }
"in"               { column += yyleng;  return(insy); }
"IN"               { column += yyleng;  return(insy); }
"bool"             { column += yyleng;  return(booltyp); }
"BOOL"             { column += yyleng;  return(booltyp); }
"boolean"          { column += yyleng;  return(booltyp); }
"BOOLEAN"          { column += yyleng;  return(booltyp); }
"int"              { column += yyleng;  return(inttyp); }
"INT"              { column += yyleng;  return(inttyp); }
"integer"          { column += yyleng;  return(inttyp); }
"INTEGER"          { column += yyleng;  return(inttyp); }

","                { column += yyleng;  return(comma); }
":"                { column += yyleng;  return(colon); }
"("                { column += yyleng;  return(oparsy); }
")"                { column += yyleng;  return(cparsy); }
"="                { column += yyleng;  return(equal); }
"&&"               { column += yyleng;  return(and); }
"and"              { column += yyleng;  return(and); }
"||"               { column += yyleng;  return(or); }
"or"               { column += yyleng;  return(or); }
"!"                { column += yyleng;  return(not); }
"not"              { column += yyleng;  return(not); }
"=="               { column += yyleng;  return(equ); }
"equ"              { column += yyleng;  return(equ); }
"#"                { column += yyleng;  return(ne); }
"!="               { column += yyleng;  return(ne); }
"ne"               { column += yyleng;  return(ne); }
"+"                { column += yyleng;  return(add); }
"add"              { column += yyleng;  return(add); }
"-"                { column += yyleng;  return(sub); }
"sub"              { column += yyleng;  return(sub); }
"*"                { column += yyleng;  return(mult); }
"mult"             { column += yyleng;  return(mult); }
"/"                { column += yyleng;  return(dvd); }
"div"              { column += yyleng;  return(dvd); }
"<"                { column += yyleng;  return(lt); }
"lt"               { column += yyleng;  return(lt); }
">"                { column += yyleng;  return(gt); }
"gt"               { column += yyleng;  return(gt); }
"<="               { column += yyleng;  return(le); }
"le"               { column += yyleng;  return(le); }
">="               { column += yyleng;  return(ge); }
"ge"               { column += yyleng;  return(ge); }
"true"             { column += yyleng;  yylval.l = btoe(1);  return(boolconst); }
"TRUE"             { column += yyleng;  yylval.l = btoe(1);  return(boolconst); }
"false"            { column += yyleng;  yylval.l = btoe(0);  return(boolconst); }
"FALSE"            { column += yyleng;  yylval.l = btoe(0);  return(boolconst); }
[0-9][0-9]*        { column += yyleng;  yylval.l = itoe(atoi((char*)yytext));\
                     return(intconst); }
[a-zA-Z](([a-zA-Z0-9_]))*     { yylval.a.val = stoid(atoe((char*)yytext));
                                yylval.a.ln  = itoe(line);
                                column += yyleng;  
                                return(ident); }
"{"           {   char c1;
                  column += yyleng;
                  while( 1 ){
                       c1 = input();
                       if( c1 == '\0' ){
                         unput(c1);
                         fprintf(stderr,"input file ends with unterminated comment\n");
                         break;
                       }
                       if( c1 == '}' ){
                         column++;
                         break;
                       }
                       if( c1 == '\n' ){
                         line++; column = 0;
                       } else {
                               column++;
                         }
                  }
              }

\end{verbatim}


\subsection{tifl\_pars.y}
\begin{verbatim}
%{
#include "tifl_spec.h"

static ELEMENT  syntaxtree = nil();

%}

%union {
    ELEMENT t;
    ELEMENT l;
    struct {
       ELEMENT val;
       ELEMENT ln;
                }a;
       }

%token<a>   ident 
%token<l>   lt    gt    le   ge    equ   ne 
%token<l>   and  or  not 
%token<l>   equal  
%token<l>   add   sub   mult  dvd 
%token<l>   ifsy  intconst  boolconst 

%token      thensy  elsesy  fisy  oparsy  cparsy  
%token      colon   comma   funcsy  insy  inttyp  booltyp 

%type<t>   prog  expr  relation  
%type<t>   sexpr  dterm  uterm  fact object  constant  funcappl  actparlist
%type<t>   decllist  decl  formparlist  type  body

/*

  or, + , -              sexpr   (simple)

  and, *, /              dterm   (dyadic)

  not, -                 uterm   (unary)

*/


%start prog

%%


prog :
           expr   
      {
       ELEMENT e;

/*--------------------------------------------------------------------------
  boolean operations 
  --------------------------------------------------------------------------*/

       e = appback(FctDeclList(), FctDecl(Boolean(),DeclId(stoid(atoe("or")),\
itoe(0)), appback(appback(ParDeclList(),ParDecl(Boolean(),DeclId(stoid(atoe("i")),\
itoe(0)))),ParDecl(Boolean(),DeclId(stoid(atoe("j")),itoe(0)))),PredeclBody()));
       e = appback(e, FctDecl(Boolean(),DeclId(stoid(atoe("and")),itoe(0)),\
 appback(appback(ParDeclList(),ParDecl(Boolean(),DeclId(stoid(atoe("i")),itoe(0)))),\
ParDecl(Boolean(),DeclId(stoid(atoe("j")),itoe(0)))),PredeclBody()));
       e = appback(e, FctDecl(Boolean(),DeclId(stoid(atoe("not")),itoe(0)),\
appback(ParDeclList(),ParDecl(Boolean(),DeclId(stoid(atoe("i")),itoe(0)))),\
PredeclBody()));


/*--------------------------------------------------------------------------
  arithmetic operations 
  --------------------------------------------------------------------------*/

       e = appback(e, FctDecl(Integer(),DeclId(stoid(atoe("add")),itoe(0)),\
 appback(appback(ParDeclList(),ParDecl(Integer(),DeclId(stoid(atoe("i")),itoe(0)))),\
ParDecl(Integer(),DeclId(stoid(atoe("j")),itoe(0)))),PredeclBody()));
       e = appback(e, FctDecl(Integer(),DeclId(stoid(atoe("sub")),itoe(0)),\
 appback(appback(ParDeclList(),ParDecl(Integer(),DeclId(stoid(atoe("i")),itoe(0)))),
ParDecl(Integer(),DeclId(stoid(atoe("j")),itoe(0)))),PredeclBody()));
       e = appback(e, FctDecl(Integer(),DeclId(stoid(atoe("mult")),itoe(0)),\
 appback(appback(ParDeclList(),ParDecl(Integer(),DeclId(stoid(atoe("i")),itoe(0)))),\
ParDecl(Integer(),DeclId(stoid(atoe("j")),itoe(0)))),PredeclBody()));
       e = appback(e, FctDecl(Integer(),DeclId(stoid(atoe("dvd")),itoe(0)),\
 appback(appback(ParDeclList(),ParDecl(Integer(),DeclId(stoid(atoe("i")),itoe(0)))),\
ParDecl(Integer(),DeclId(stoid(atoe("j")),itoe(0)))),PredeclBody()));


/*--------------------------------------------------------------------------
  relations 
  --------------------------------------------------------------------------*/

       e = appback(e, FctDecl(Boolean(),DeclId(stoid(atoe("lt")),itoe(0)),\
 appback(appback(ParDeclList(),ParDecl(Integer(),DeclId(stoid(atoe("i")),itoe(0)))),\
ParDecl(Integer(),DeclId(stoid(atoe("j")),itoe(0)))),PredeclBody()));
       e = appback(e, FctDecl(Boolean(),DeclId(stoid(atoe("gt")),itoe(0)),\
 appback(appback(ParDeclList(),ParDecl(Integer(),DeclId(stoid(atoe("i")),itoe(0)))),\
ParDecl(Integer(),DeclId(stoid(atoe("j")),itoe(0)))),PredeclBody()));
       e = appback(e, FctDecl(Boolean(),DeclId(stoid(atoe("le")),itoe(0)),\
 appback(appback(ParDeclList(),ParDecl(Integer(),DeclId(stoid(atoe("i")),itoe(0)))),\
ParDecl(Integer(),DeclId(stoid(atoe("j")),itoe(0)))),PredeclBody()));
       e = appback(e, FctDecl(Boolean(),DeclId(stoid(atoe("ge")),itoe(0)),\
 appback(appback(ParDeclList(),ParDecl(Integer(),DeclId(stoid(atoe("i")),itoe(0)))),\
ParDecl(Integer(),DeclId(stoid(atoe("j")),itoe(0)))),PredeclBody()));
       e = appback(e, FctDecl(Boolean(),DeclId(stoid(atoe("equ")),itoe(0)),\
 appback(appback(ParDeclList(),ParDecl(Integer(),DeclId(stoid(atoe("i")),itoe(0)))),\
ParDecl(Integer(),DeclId(stoid(atoe("j")),itoe(0)))),PredeclBody()));
       e = appback(e, FctDecl(Boolean(),DeclId(stoid(atoe("ne")),itoe(0)),\
 appback(appback(ParDeclList(),ParDecl(Integer(),DeclId(stoid(atoe("i")),itoe(0)))),\
ParDecl(Integer(),DeclId(stoid(atoe("j")),itoe(0)))),PredeclBody()));


      syntaxtree  = Program( LetExp( e, $1 ) );

      } 
;
expr :
      sexpr
      {  $$  =  $1; }
;
sexpr:
      dterm
      {  $$  =  $1; }
|     sexpr or dterm
      {  $$  =  FctAppl( UsedId( stoid(atoe( "or" )),  itoe( line ) ),\
                   appback( appback( ExpList(),  $1 ),  $3 ) );  }
|     sexpr add dterm
      {  $$  =  FctAppl( UsedId( stoid(atoe( "add" )), itoe( line ) ),\
                   appback( appback( ExpList(),  $1 ),  $3 ) ); } 
|     sexpr sub dterm
      {  $$  =  FctAppl( UsedId( stoid(atoe( "sub" )), itoe( line ) ),\
                   appback( appback( ExpList(),  $1 ),  $3 ) ); } 
;
dterm :
      uterm
      {  $$  =  $1; }
|     dterm and uterm
      {  $$  =  FctAppl( UsedId( stoid(atoe( "and" )),  itoe( line ) ),\
                   appback( appback( ExpList(),  $1 ),  $3 ) );  }
|     dterm mult uterm
      {  $$  =  FctAppl( UsedId( stoid(atoe( "mult" )),  itoe( line ) ),\
                   appback( appback( ExpList(),  $1 ),  $3 ) );  }
|     dterm dvd uterm
      {  $$  =  FctAppl( UsedId( stoid(atoe( "dvd" )),  itoe( line ) ),\
                   appback( appback( ExpList(),  $1 ),  $3 ) );  }
;
uterm :        
      fact
      {  $$  =  $1; }
|     not fact
      {  $$  =  FctAppl( UsedId( stoid(atoe( "not" )),  itoe( line ) ),\
                   appback( ExpList(),  $2 ) ); }
|     sub fact
      {  $$  =  FctAppl( UsedId( stoid(atoe( "sub" )),  itoe( line ) ),\
                   appback( appback( ExpList(), itoe(0) ),  $2 )  ); }
;
expr :
      relation
      {  $$  =  $1; }
;
relation :
      sexpr equ sexpr
      {  $$  =  FctAppl( UsedId( stoid(atoe( "equ" )),  itoe( line ) ),\
                   appback( appback( ExpList(),  $1 ),  $3 ) ); }
|     sexpr ne sexpr
      {  $$  =  FctAppl( UsedId( stoid(atoe( "ne" )),  itoe( line ) ),\
                   appback( appback( ExpList(),  $1 ),  $3 ) ); }
|     sexpr lt sexpr
      {  $$  =  FctAppl( UsedId( stoid(atoe( "lt" )),  itoe( line ) ),\
                   appback( appback( ExpList(),  $1 ),  $3 ) ); }
|     sexpr gt sexpr
      {  $$  =  FctAppl( UsedId( stoid(atoe( "gt" )),  itoe( line ) ),\
                   appback( appback( ExpList(),  $1 ),  $3 ) ); }
|     sexpr le sexpr
      {  $$  =  FctAppl( UsedId( stoid(atoe( "le" )),  itoe( line ) ),\
                   appback( appback( ExpList(),  $1 ),  $3 ) ); }
|     sexpr ge sexpr
      {  $$  =  FctAppl( UsedId( stoid(atoe( "ge" )),  itoe( line ) ),\
                   appback( appback( ExpList(),  $1 ),  $3 ) ); }
;
expr :
      decllist insy expr
      {  $$  =  LetExp( $1, $3 ); }
;
fact :
      constant
      {  $$  =  $1; }
|     oparsy expr cparsy 
      {  $$  =  $2; }
|      ifsy expr thensy expr elsesy expr fisy
      {  $$  =  CondExp( $2,  $4,  $6 ); }
|     object
      {  $$  =  $1; }
|     funcappl
      {  $$  =  $1; }
;
object :
      ident
      {  $$  =  UsedId( $1.val,  $1.ln  ); }
;
constant :
      intconst   
      {  $$  =   $1;  }
|     boolconst
      {  $$  =   $1;  }
;
funcappl  : 
           ident oparsy cparsy 
           {  $$  =  FctAppl( UsedId( $1.val,  $1.ln ),  ExpList() ); }
;
funcappl  :
      ident oparsy actparlist cparsy 
      {  $$  =  FctAppl( UsedId( $1.val,  $1.ln ),  $3 ); }
;
actparlist :
      expr
      {  $$  =  appback( ExpList(),  $1 ); }
;
actparlist : 
      actparlist comma expr
      {  $$  =  appback( $1,  $3 ); }
;
decllist :  
      decl
      {  $$  =  appback( FctDeclList(),  $1 ); }
;
decllist :  
      decllist decl
      {  $$  =  appback( $1,  $2 ); }
;
decl     :  
      funcsy ident equal oparsy formparlist cparsy colon type body 
      {  $$  =  FctDecl( $8, DeclId( $2.val,  $2.ln ),  $5,  $9 ); }
;
formparlist :
                                              
      {  $$  =  ParDeclList(); }
;
formparlist :
      type ident
      {  $$  =  appback( ParDeclList(),  ParDecl( $1,  DeclId( $2.val,  $2.ln ) ) ); }
;
formparlist : 
      formparlist comma type ident
      {  $$  =  appback( $1,  ParDecl( $3,  DeclId( $4.val,  $4.ln ) ) ); }
;
type :  
      inttyp
      {  $$  =  Integer(); }
;
type :   
      booltyp
      {  $$  =  Boolean(); }
;
body :          
      expr
      {  $$  =  $1; }
;


%%

#include "lex.yy.c" 
#include <stdio.h> 

int yyerror( char *s ){
   fprintf( stderr, "**** Line %d: %s\n",line, s );
}
yywrap(){return(1);}


int yyparse();

ELEMENT  yyread( FILE *file ){
   yyin = file;
   if( yyparse() )
      return nil();
   else 
      return syntaxtree;
}
\end{verbatim}


\subsection{tifl\_spec.m}
\begin{verbatim}
////////////////////////////////////////////////////////////////
//   Abstract Syntax of TIFL 
/////////////////////////////////


Program      (  Exp  )
Exp          =  LetExp  |  Int  |  Bool  |  CondExp  |  FctAppl  |  UsedId
LetExp       (  FctDeclList  Exp  )
FctDeclList  *  FctDecl
FctDecl      (  Type  DeclId  ParDeclList  Body  )
Type         =  Boolean  |  Integer
Boolean      ()
Integer      ()
ParDeclList  *  ParDecl
ParDecl      (  Type  DeclId  )
Body         =  Exp  |  PredeclBody
PredeclBody  () 
CondExp      (  Exp  Exp  Exp  )
FctAppl      (  UsedId  ExpList  )
ExpList      *  Exp
UsedId       (  Ident  LineNo  )
DeclId       (  Ident  LineNo  )


Decl         =  FctDecl |  ParDecl
Scope        =  FctDecl |  LetExp
LineNo       =  Int


/////////////////////////////////////////////////////////////
//   some sorts for the evaluator
/////////////////////////////////
Env          *  Pair
Pair         (  ParDecl@  Value  )
Value        =  Bool | Int



//////////////////////////////
//  function declarations  
//  (bodies in tifl.c)
//////////////////////////////

FCT itos( Int ) String

FCT and( Bool, Bool ) Bool
FCT or( Bool, Bool ) Bool
FCT not( Bool ) Bool

FCT add( Int, Int ) Int
FCT sub( Int, Int ) Int
FCT mul( Int, Int ) Int

FCT lt( Int, Int ) Bool
FCT le( Int, Int ) Bool
FCT equ( Int, Int ) Bool


STRUC  tifl  [ Program ]{

FCT id( Node I ) Ident : term( son( 1, I ) )
FCT ln( Node N ) String : itos( term( son( -1, N ) ) )

ATT encl_scope( Node N ) Scope@ :
   IF  is[ N, _FctDecl@ ]            : N
   |   is[ N, _LetExp@ ]             : N 
   ELSE                     encl_scope( fath( N ) )

FCT first_decl( Scope@ S ) Decl@:
   IF  LetExp@<<FCTD,*>,_> S     :   FCTD
   |   FctDecl@<_,_,<PARD,*>,_> S:   PARD
   ELSE nil()

FCT lookup_list( Ident ID, Decl@ D ) DeclId@ :
   IF  ID = id( son(2,D) )  :  son(2,D)   ELSE  lookup_list( ID, rbroth(D) )

FCT lookup( Ident ID, Scope@ S ) DeclId@ :
   LET  DID == lookup_list(ID,first_decl(S)) :
   IF   DID # nil() :  DID  ELSE  lookup( ID, encl_scope(fath(S)) )

ATT decl( UsedId@ UID ) DeclId@ :
   lookup( id(UID), encl_scope(UID) )

ATT type( Exp@ E ) Type :
   IF LetExp@<_,EX> E   :  type( EX )
   |  Int@ E            :  Integer()
   |  Bool@ E   	:  Boolean()
   |  CondExp@<_,E2,_> E:  type( E2 ) 
   |  FctAppl@<UID,_> E :  type( UID ) 
   |  UsedId@ E         :  term( son( 1, fath(decl(E)) ) )
   ELSE nil()


PRD par_type_check[ ParDecl@ P, Exp@ E ]:
   term(son(1,P)) = type(E)
 && { rbroth(P) # nil()  ->  par_type_check[ rbroth(P), rbroth(E) ] }
   

////////////////////////////////////////////////////////

//////////////////////////////////////////////////////////////////
//
//     Context Conditions 
//
//////////////////////////////////////////////////////////////////

CND CondExp@<_,E1,E2>               : type(E1) = type(E2)
| `"### : then/else expr's must be of same type\n"'


CND CondExp@<E1,*>                  : is[type(E1),_Boolean] 
| `"### : conditional expr must be of type Boolean\n"'


CND ParDeclList@<*,<_,ID1>,*,<_,ID2>,*>  : id(ID1) # id(ID2)
| `"### LINE(S) " ln(ID1) "/" ln(ID2) ": parameter \"" idtos(id(ID1))
  "\" doubly declared\n"'


CND FctDeclList@<*,<_,ID1,*>,*,<_,ID2,*>,*>   : id(ID1) # id(ID2)
| `"### LINE(S) " ln(ID1) "/" ln(ID2) ": function \"" idtos(id(ID1))
  "\" doubly declared\n"'


CND UsedId@ UID                   : decl(UID) # nil()
| `"### LINE " ln(UID) ": identifier \"" idtos(id(UID)) "\" not declared\n"'


CND FctAppl@<UID,EL>    : numsons(EL) = numsons( son( 3, fath( decl(UID) ) ) )
| `"### LINE " ln(UID) ": incorrect number of parameters in call of function
  "idtos(id(UID))"\n"'


CND FctAppl@<UID,<E,*>>    : par_type_check[ son(1,son(3,fath(decl(UID)))), E ]
| `"### LINE " ln(UID) ": type mismatch in call of function "idtos(id(UID))"\n"'


CND FctDecl@<T,DID,_,B>                :
 !is[ B , _PredeclBody@] -> term( T ) = type( B )
| `"### LINE " ln(DID) ": declaration of "idtos(id(DID))
  " does not match function result\n"'


CND UsedId@ UID                        :
   rbroth(UID) = nil() -> !is[ fath( decl ( UID ) ), _FctDecl@ ]
| `"### LINE " ln(UID) ": incorrect call of "idtos(id(UID))" \n"'


///////////////////////////////////////////////////////////////////////////
//
//  eval and related functions
//  (environment version)
///////////////////////////////////////////////////////////////////////////
                         

FCT  env_lookup( ParDecl@ PN, Env E ) Value:
   IF  subterm(1,subterm(1,E)) = PN  :  subterm(2,subterm(1,E))
                                  ELSE  env_lookup( PN, back(E) )

FCT  enter_pars( ParDecl@ PN, Exp@ EN, Env E, Env EVALENV ) Env:
   LET  V == eval( EN, EVALENV )   :
   IF   rbroth(PN) = nil()   :	appfront( Pair(PN,V), E )
   ELSE   enter_pars( rbroth(PN), rbroth(EN), appfront( Pair(PN,V), E ), EVALENV )


FCT  eval_predecl( Ident ID, Exp@ EN, Env E ) Value :
   IF  idtos(ID) = "or"      :   or( eval(EN,E), eval(rbroth(EN),E) )
    |  idtos(ID) = "and"     :   and( eval(EN,E), eval(rbroth(EN),E) )
    |  idtos(ID) = "not"     :   not( eval(EN,E) )
    |  idtos(ID) = "add"     :   add( eval(EN,E), eval(rbroth(EN),E) )
    |  idtos(ID) = "sub"     :   sub( eval(EN,E), eval(rbroth(EN),E) )
    |  idtos(ID) = "mul"     :   mul( eval(EN,E), eval(rbroth(EN),E) )
    |  idtos(ID) = "lt"      :   lt( eval(EN,E), eval(rbroth(EN),E) )
    |  idtos(ID) = "le"      :   le( eval(EN,E), eval(rbroth(EN),E) )
    |  idtos(ID) = "equ"     :   equ( eval(EN,E), eval(rbroth(EN),E) )
   ELSE  nil()

FCT  eval ( Exp@ X, Env E ) Value:
   IF  LetExp@<_,BD>        X:  eval( BD, E )
    |  Int@                 X:  term(X)
    |  Bool@                X:  term(X)
    |  CondExp@<E1,E2,E3>   X:  IF eval(E1,E) = true() : eval(E2,E)  ELSE  eval(E3,E) 
    |  UsedId@              X:  env_lookup( fath(decl(X)), E )
    |  FctAppl@<UID,<>>     X:  eval( son(4, UID.decl.fath ), E )
    |  FctAppl@<UID,<E1,*>> X:  LET  FCTDCL ==  UID.decl.fath  :
	IF  FctDecl@<_,<IDN,_>,_,PredeclBody@> FCTDCL:  eval_predecl(term(IDN), E1, E)
	ELSE   eval( son(4,FCTDCL), enter_pars(son(1,son(3,FCTDCL)),E1,E,E)  )
   ELSE nil()

}
\end{verbatim}


\subsection{tifl.c}
\begin{verbatim}
/*------------    main program for TIFL  ----------*/
#include <stdlib.h>
#include <stdio.h>
#include <string.h>
#include "tifl_spec.h"
#include "stack.c"


ELEMENT itos( ELEMENT i ){
   char s[20];
   sprintf(s,"%d", i );
   return  atoe(s);
}

/*---------------------------------------------------------------------------
  predefined functions
  ---------------------------------------------------------------------------*/

  /* boolean ops  */
ELEMENT or( ELEMENT o1, ELEMENT o2 ){   return btoe( etob(o1) || etob(o2) ); } 
ELEMENT and( ELEMENT o1, ELEMENT o2 ){   return btoe( etob(o1) && etob(o2) ); }
ELEMENT not( ELEMENT o1 ){ if(etob(o1))
                             { return btoe( 0 ); }
                           else 
                             { return btoe( 1 ); }
                         }

  /* arithmetic ops   */
ELEMENT add( ELEMENT o1, ELEMENT o2 ){   return itoe(etoi(o1) + etoi(o2)); }
ELEMENT sub( ELEMENT o1, ELEMENT o2 ){   return itoe(etoi(o1) - etoi(o2)); }
ELEMENT mul( ELEMENT o1, ELEMENT o2 ){   return itoe(etoi(o1) * etoi(o2)); }
ELEMENT dvd( ELEMENT o1, ELEMENT o2 ){   
                                      if(!etoi(o2))
                                        { printf("\n!!! Division by zero\n");
                                          return itoe(1);
                                        }
                                      else
                                        { return itoe(etoi(o1) / etoi(o2)); }
                                     }

  /* relations  */
ELEMENT lt( ELEMENT o1, ELEMENT o2 ){   return btoe(o1<o2); }
ELEMENT gt( ELEMENT o1, ELEMENT o2 ){   return btoe(o1>o2); }
ELEMENT le( ELEMENT o1, ELEMENT o2 ){   return btoe(o1<=o2); }
ELEMENT ge( ELEMENT o1, ELEMENT o2 ){   return btoe(o1>=o2); }
ELEMENT equ( ELEMENT o1, ELEMENT o2 ){   return btoe(o1 == o2); }
ELEMENT ne( ELEMENT o1, ELEMENT o2 ){   return btoe(o1!=o2); }


void  blanks( FILE* f, int n ){
   while( n-- > 0 )   fprintf(f,"  ");
}

void  treeprint( FILE* f, ELEMENT n, int indent ){
   ELEMENT  termv = term(n);

   fprintf(f,"%6dN :  ", index(n,_Node) );
   blanks(f,indent);
   switch( sort(termv) ){

   case _Ident:           fprintf(f,"\"%s\"\n", etoa(idtos(termv)) ); break;
   case _Int:             fprintf(f,"value: %d\n", etoi(termv) );break;
   case _Bool:            if( etob(termv) ){
                            fprintf(f,"value: true\n");break;
                          } else {
                                  fprintf(f,"value: false\n");break;
                          }
   case _String:          fprintf(f,"\"%s\"\n", etoa(termv) );break;
   case _Program:         fprintf(f,"Program\n"); break;
   case _LetExp:          fprintf(f,"LetExp\n"); break;
   case _FctDeclList:     fprintf(f,"FctDeclList (%d)\n", numsubterms(termv));break;
   case _FctDecl:         fprintf(f,"FctDecl\n");break;
   case _Boolean:         fprintf(f,"Boolean\n");break;
   case _Integer:         fprintf(f,"Integer\n");break;
   case _ParDeclList:     fprintf(f,"ParDeclList (%d)\n", numsubterms(termv));break;
   case _ParDecl:         fprintf(f,"ParDecl\n");break;
   case _PredeclBody:     fprintf(f,"PredeclBody\n");break;
   case _CondExp:         fprintf(f,"CondExp\n"); break;
   case _FctAppl:         fprintf(f,"FctAppl\n"); break;
   case _ExpList:         fprintf(f,"ExpList (%d)\n", numsubterms(termv)); break;
   case _UsedId:          fprintf(f,"UsedId in line %d\n",etoi(term(son(2,n)))); break;
   case _DeclId:          fprintf(f,"DeclId in line %d\n",etoi(term(son(2,n)))); break;
   default:               fprintf(f,"************\n"); break;
   }      
   n = son(1,n);
   while( n != nil() ){
      treeprint(f,n,indent+1);
      n = rbroth(n);
   }
}

extern  ELEMENT yyread( FILE * );
static  FILE  *out;

int  main( int argc, char **argv ){
   ELEMENT termv;
   ELEMENT eval_result;
   char   *filename = argv[1];
   FILE   *fileptr;
   long np;

   int     leng = strlen(filename);
   long    ix, bound;
   
   if( argc == 1 ){
      fprintf(stdout,"Usage: tifl <file>\n");
      return EXIT_FAILURE;
   }
   if( filename[leng-5] != '.' || filename[leng-4] != 't'  ||  filename[leng-3] != 'i' 
      ||  filename[leng-2] != 'f'  ||  filename[leng-1] != 'l'){
      fprintf(stdout,"tifl: input file does not end with \".tifl\"\n");
      return EXIT_FAILURE;
   }
   fprintf(stdout,"+++ reading\n");
   if( !( fileptr = fopen(filename,"r") ) ){
      fprintf(stdout,"tifl: Can't open file %s for input\n",filename);
      return EXIT_FAILURE;
   }
   termv = yyread( fileptr );

   if( termv == nil() ){
      fprintf(stdout,"tifl: fatal read error: exit\n",filename);
      return EXIT_FAILURE;
   }
   fprintf(stdout,"+++ structure construction\n");

   init_stacks();

   if(  init_tifl( termv )  ){
      fprintf(stdout,"+++ statistics\n");
      fprintf(stdout,"number of nodes:       %d\n", number(_Node) );
      fprintf(stdout,"number of idents:      %d\n", number(_Ident) );
      fprintf(stdout,"number of FctDecls:  %d\n", number(_FctDecl_) );

      filename[leng-5] = '.';
      filename[leng-4] = 't';
      filename[leng-3] = 'r';
      filename[leng-2] = 'e';
      filename[leng-1] = 'e';
      filename[leng] = '\0';
      if( !( out = fopen(filename,"w") ) ){
         fprintf(stdout,"tifl: Can't open file %s for output\n",filename);
         return EXIT_FAILURE;
      }
      treeprint( out, root(), 0 );

      /*---------------------------------------
        check if there are enough stacks available
        ---------------------------------------*/

      if ((np = number(_ParDecl_)) > MAX_NUM_PAR)
         { 
          printf("\n!!! RUNTIME error: too many parameters\n");
          getchar();
          return EXIT_FAILURE;
         }

      printf("\n+++ TIFL-Interpreter started!\n");
      printf("+++ stack version\n");
      printf("\ninput program evaluates to :  ");

      eval_result =  eval( son(1,root()) );
      if( sort(eval_result) == _Int )
        {
              printf("%d\n", etoi(eval_result) );
        }
      else
        {
         if( etob(eval_result) )
           {
            printf("true\n");
           }
         else
           {
            printf("false\n");
           }
        }

   }
   return EXIT_SUCCESS;
}

\end{verbatim}


%----  stack version  ------
\section{Stack version}
\subsection{Makefile}
\begin{verbatim}
# Makefile for generation of MAX analysers 

# language dependent variables
LANG=tifl

# compiler dependent variables
INCLUDES=
CFLAGS= $(INCLUDES)
CC=cc -Aa 

$(LANG) : y.tab.o  max_std.o   $(LANG)_spec.o  $(LANG).o 
	$(CC) $(CFLAGS) -o $(LANG) y.tab.o  max_std.o \
			   $(LANG)_spec.o $(LANG).o 
	cp $(LANG) ../../bsp

y.tab.o: y.tab.c  lex.yy.c 
	$(CC) -g -c  y.tab.c

y.tab.c: $(LANG)_pars.y  $(LANG)_spec.h
	yacc $(LANG)_pars.y	

lex.yy.c: $(LANG)_scan.l 
	lex $(LANG)_scan.l 	

$(LANG)_spec.o :  $(LANG)_spec.c
	$(CC) $(CFLAGS) -c $(LANG)_spec.c

$(LANG)_spec.c $(LANG)_spec.h :  $(LANG)_spec.m ../../../MAX/src.5/max
	../../../MAX/src.5/max $(LANG)_spec.m

$(LANG).o :  $(LANG).c  $(LANG)_spec.h stack.c 
	$(CC) $(CFLAGS) -c $(LANG).c

\end{verbatim}


\subsection{tifl\_spec.m}
\begin{verbatim}
////////////////////////////////////////////////////////////////
//   Abstract Syntax of TIFL 
/////////////////////////////////


Program      (  Exp  )
Exp          =  LetExp  |  Int  |  Bool  |  CondExp  |  FctAppl  |  UsedId
LetExp       (  FctDeclList  Exp  )
FctDeclList  *  FctDecl
FctDecl      (  Type  DeclId  ParDeclList  Body  )
Type         =  Boolean  |  Integer
Boolean      ()
Integer      ()
ParDeclList  *  ParDecl
ParDecl      (  Type  DeclId  )
Body         =  Exp  |  PredeclBody
PredeclBody  () 
CondExp      (  Exp  Exp  Exp  )
FctAppl      (  UsedId  ExpList  )
ExpList      *  Exp
UsedId       (  Ident  LineNo  )
DeclId       (  Ident  LineNo  )


Decl         =  FctDecl |  ParDecl
Scope        =  FctDecl |  LetExp
LineNo       =  Int


/////////////////////////////////////////////////////////////
//   some sorts for the evaluator
/////////////////////////////////
Env          *  Pair
Pair         (  ParDecl@  Value  )
Value        =  Bool | Int



//////////////////////////////
//  function declarations  
//  (bodies in tifl.c)
//////////////////////////////

FCT itos( Int ) String

FCT and( Bool, Bool ) Bool
FCT or( Bool, Bool ) Bool
FCT not( Bool ) Bool

FCT add( Int, Int ) Int
FCT sub( Int, Int ) Int
FCT mul( Int, Int ) Int

FCT lt( Int, Int ) Bool
FCT le( Int, Int ) Bool
FCT equ( Int, Int ) Bool


STRUC  tifl  [ Program ]{

FCT id( Node I ) Ident : term( son( 1, I ) )
FCT ln( Node N ) String : itos( term( son( -1, N ) ) )

ATT encl_scope( Node N ) Scope@ :
   IF  is[ N, _FctDecl@ ]            : N
   |   is[ N, _LetExp@ ]             : N 
   ELSE                     encl_scope( fath( N ) )

FCT first_decl( Scope@ S ) Decl@:
   IF  LetExp@<<FCTD,*>,_> S     :   FCTD
   |   FctDecl@<_,_,<PARD,*>,_> S:   PARD
   ELSE nil()

FCT lookup_list( Ident ID, Decl@ D ) DeclId@ :
   IF  ID = id( son(2,D) )  :  son(2,D)   ELSE  lookup_list( ID, rbroth(D) )

FCT lookup( Ident ID, Scope@ S ) DeclId@ :
   LET  DID == lookup_list(ID,first_decl(S)) :
   IF   DID # nil() :  DID  ELSE  lookup( ID, encl_scope(fath(S)) )

ATT decl( UsedId@ UID ) DeclId@ :
   lookup( id(UID), encl_scope(UID) )

ATT type( Exp@ E ) Type :
   IF LetExp@<_,EX> E   :  type( EX )
   |  Int@ E            :  Integer()
   |  Bool@ E   	:  Boolean()
   |  CondExp@<_,E2,_> E:  type( E2 ) 
   |  FctAppl@<UID,_> E :  type( UID ) 
   |  UsedId@ E         :  term( son( 1, fath(decl(E)) ) )
   ELSE nil()


PRD par_type_check[ ParDecl@ P, Exp@ E ]:
   term(son(1,P)) = type(E)
 && { rbroth(P) # nil()  ->  par_type_check[ rbroth(P), rbroth(E) ] }
   

////////////////////////////////////////////////////////

//////////////////////////////////////////////////////////////////
//
//     Context Conditions 
//
//////////////////////////////////////////////////////////////////

CND CondExp@<_,E1,E2>               : type(E1) = type(E2)
| `"### : then/else expr's must be of same type\n"'


CND CondExp@<E1,*>                  : is[type(E1),_Boolean] 
| `"### : conditional expr must be of type Boolean\n"'


CND ParDeclList@<*,<_,ID1>,*,<_,ID2>,*>  : id(ID1) # id(ID2)
| `"### LINE(S) " ln(ID1) "/" ln(ID2) ": parameter \"" idtos(id(ID1))
  "\" doubly declared\n"'


CND FctDeclList@<*,<_,ID1,*>,*,<_,ID2,*>,*>   : id(ID1) # id(ID2)
| `"### LINE(S) " ln(ID1) "/" ln(ID2) ": function \"" idtos(id(ID1))
  "\" doubly declared\n"'


CND UsedId@ UID                   : decl(UID) # nil()
| `"### LINE " ln(UID) ": identifier \"" idtos(id(UID)) "\" not declared\n"'


CND FctAppl@<UID,EL>    : numsons(EL) = numsons( son( 3, fath( decl(UID) ) ) )
| `"### LINE " ln(UID) ": incorrect number of parameters in call of function
  "idtos(id(UID))"\n"'


CND FctAppl@<UID,<E,*>>    : par_type_check[ son(1,son(3,fath(decl(UID)))), E ]
| `"### LINE " ln(UID) ": type mismatch in call of function "idtos(id(UID))"\n"'


CND FctDecl@<T,DID,_,B>                :
 !is[ B , _PredeclBody@] -> term( T ) = type( B )
| `"### LINE " ln(DID) ": declaration of "idtos(id(DID))
  " does not match function result\n"'


CND UsedId@ UID                        :
   rbroth(UID) = nil() -> !is[ fath( decl ( UID ) ), _FctDecl@ ]
| `"### LINE " ln(UID) ": incorrect call of "idtos(id(UID))" \n"'


///////////////////////////////////////////////////////////////////////////
//
//  eval and related functions
//  (environment version)
///////////////////////////////////////////////////////////////////////////
                         

FCT  env_lookup( ParDecl@ PN, Env E ) Value:
   IF  subterm(1,subterm(1,E)) = PN  :  subterm(2,subterm(1,E))
                                  ELSE  env_lookup( PN, back(E) )

FCT  enter_pars( ParDecl@ PN, Exp@ EN, Env E, Env EVALENV ) Env:
   LET  V == eval( EN, EVALENV )   :
   IF   rbroth(PN) = nil()   :	appfront( Pair(PN,V), E )
   ELSE   enter_pars( rbroth(PN), rbroth(EN), appfront( Pair(PN,V), E ), EVALENV )


FCT  eval_predecl( Ident ID, Exp@ EN, Env E ) Value :
   IF  idtos(ID) = "or"      :   or( eval(EN,E), eval(rbroth(EN),E) )
    |  idtos(ID) = "and"     :   and( eval(EN,E), eval(rbroth(EN),E) )
    |  idtos(ID) = "not"     :   not( eval(EN,E) )
    |  idtos(ID) = "add"     :   add( eval(EN,E), eval(rbroth(EN),E) )
    |  idtos(ID) = "sub"     :   sub( eval(EN,E), eval(rbroth(EN),E) )
    |  idtos(ID) = "mul"     :   mul( eval(EN,E), eval(rbroth(EN),E) )
    |  idtos(ID) = "lt"      :   lt( eval(EN,E), eval(rbroth(EN),E) )
    |  idtos(ID) = "le"      :   le( eval(EN,E), eval(rbroth(EN),E) )
    |  idtos(ID) = "equ"     :   equ( eval(EN,E), eval(rbroth(EN),E) )
   ELSE  nil()

FCT  eval ( Exp@ X, Env E ) Value:
   IF  LetExp@<_,BD>        X:  eval( BD, E )
    |  Int@                 X:  term(X)
    |  Bool@                X:  term(X)
    |  CondExp@<E1,E2,E3>   X:  IF eval(E1,E) = true() : eval(E2,E)  ELSE  eval(E3,E) 
    |  UsedId@              X:  env_lookup( fath(decl(X)), E )
    |  FctAppl@<UID,<>>     X:  eval( son(4, UID.decl.fath ), E )
    |  FctAppl@<UID,<E1,*>> X:  LET  FCTDCL ==  UID.decl.fath  :
	IF  FctDecl@<_,<IDN,_>,_,PredeclBody@> FCTDCL:  eval_predecl(term(IDN), E1, E)
	ELSE   eval( son(4,FCTDCL), enter_pars(son(1,son(3,FCTDCL)),E1,E,E)  )
   ELSE nil()

}
\end{verbatim}


\subsection{tifl.c}
\begin{verbatim}
/*------------    main program for TIFL  ----------*/
#include <stdlib.h>
#include <stdio.h>
#include <string.h>
#include "tifl_spec.h"
#include "stack.c"


ELEMENT itos( ELEMENT i ){
   char s[20];
   sprintf(s,"%d", i );
   return  atoe(s);
}

/*---------------------------------------------------------------------------
  predefined functions
  ---------------------------------------------------------------------------*/

  /* boolean ops  */
ELEMENT or( ELEMENT o1, ELEMENT o2 ){   return btoe( etob(o1) || etob(o2) ); } 
ELEMENT and( ELEMENT o1, ELEMENT o2 ){   return btoe( etob(o1) && etob(o2) ); }
ELEMENT not( ELEMENT o1 ){ if(etob(o1))
                             { return btoe( 0 ); }
                           else 
                             { return btoe( 1 ); }
                         }

  /* arithmetic ops   */
ELEMENT add( ELEMENT o1, ELEMENT o2 ){   return itoe(etoi(o1) + etoi(o2)); }
ELEMENT sub( ELEMENT o1, ELEMENT o2 ){   return itoe(etoi(o1) - etoi(o2)); }
ELEMENT mul( ELEMENT o1, ELEMENT o2 ){   return itoe(etoi(o1) * etoi(o2)); }
ELEMENT dvd( ELEMENT o1, ELEMENT o2 ){   
                                      if(!etoi(o2))
                                        { printf("\n!!! Division by zero\n");
                                          return itoe(1);
                                        }
                                      else
                                        { return itoe(etoi(o1) / etoi(o2)); }
                                     }

  /* relations  */
ELEMENT lt( ELEMENT o1, ELEMENT o2 ){   return btoe(o1<o2); }
ELEMENT gt( ELEMENT o1, ELEMENT o2 ){   return btoe(o1>o2); }
ELEMENT le( ELEMENT o1, ELEMENT o2 ){   return btoe(o1<=o2); }
ELEMENT ge( ELEMENT o1, ELEMENT o2 ){   return btoe(o1>=o2); }
ELEMENT equ( ELEMENT o1, ELEMENT o2 ){   return btoe(o1 == o2); }
ELEMENT ne( ELEMENT o1, ELEMENT o2 ){   return btoe(o1!=o2); }


void  blanks( FILE* f, int n ){
   while( n-- > 0 )   fprintf(f,"  ");
}

void  treeprint( FILE* f, ELEMENT n, int indent ){
   ELEMENT  termv = term(n);

   fprintf(f,"%6dN :  ", index(n,_Node) );
   blanks(f,indent);
   switch( sort(termv) ){

   case _Ident:           fprintf(f,"\"%s\"\n", etoa(idtos(termv)) ); break;
   case _Int:             fprintf(f,"value: %d\n", etoi(termv) );break;
   case _Bool:            if( etob(termv) ){
                            fprintf(f,"value: true\n");break;
                          } else {
                                  fprintf(f,"value: false\n");break;
                          }
   case _String:          fprintf(f,"\"%s\"\n", etoa(termv) );break;
   case _Program:         fprintf(f,"Program\n"); break;
   case _LetExp:          fprintf(f,"LetExp\n"); break;
   case _FctDeclList:     fprintf(f,"FctDeclList (%d)\n", numsubterms(termv));break;
   case _FctDecl:         fprintf(f,"FctDecl\n");break;
   case _Boolean:         fprintf(f,"Boolean\n");break;
   case _Integer:         fprintf(f,"Integer\n");break;
   case _ParDeclList:     fprintf(f,"ParDeclList (%d)\n", numsubterms(termv));break;
   case _ParDecl:         fprintf(f,"ParDecl\n");break;
   case _PredeclBody:     fprintf(f,"PredeclBody\n");break;
   case _CondExp:         fprintf(f,"CondExp\n"); break;
   case _FctAppl:         fprintf(f,"FctAppl\n"); break;
   case _ExpList:         fprintf(f,"ExpList (%d)\n", numsubterms(termv)); break;
   case _UsedId:          fprintf(f,"UsedId in line %d\n",etoi(term(son(2,n)))); break;
   case _DeclId:          fprintf(f,"DeclId in line %d\n",etoi(term(son(2,n)))); break;
   default:               fprintf(f,"************\n"); break;
   }      
   n = son(1,n);
   while( n != nil() ){
      treeprint(f,n,indent+1);
      n = rbroth(n);
   }
}

extern  ELEMENT yyread( FILE * );
static  FILE  *out;

int  main( int argc, char **argv ){
   ELEMENT termv;
   ELEMENT eval_result;
   char   *filename = argv[1];
   FILE   *fileptr;
   long np;

   int     leng = strlen(filename);
   long    ix, bound;
   
   if( argc == 1 ){
      fprintf(stdout,"Usage: tifl <file>\n");
      return EXIT_FAILURE;
   }
   if( filename[leng-5] != '.' || filename[leng-4] != 't'  ||  filename[leng-3] != 'i' 
      ||  filename[leng-2] != 'f'  ||  filename[leng-1] != 'l'){
      fprintf(stdout,"tifl: input file does not end with \".tifl\"\n");
      return EXIT_FAILURE;
   }
   fprintf(stdout,"+++ reading\n");
   if( !( fileptr = fopen(filename,"r") ) ){
      fprintf(stdout,"tifl: Can't open file %s for input\n",filename);
      return EXIT_FAILURE;
   }
   termv = yyread( fileptr );

   if( termv == nil() ){
      fprintf(stdout,"tifl: fatal read error: exit\n",filename);
      return EXIT_FAILURE;
   }
   fprintf(stdout,"+++ structure construction\n");

   init_stacks();

   if(  init_tifl( termv )  ){
      fprintf(stdout,"+++ statistics\n");
      fprintf(stdout,"number of nodes:       %d\n", number(_Node) );
      fprintf(stdout,"number of idents:      %d\n", number(_Ident) );
      fprintf(stdout,"number of FctDecls:  %d\n", number(_FctDecl_) );

      filename[leng-5] = '.';
      filename[leng-4] = 't';
      filename[leng-3] = 'r';
      filename[leng-2] = 'e';
      filename[leng-1] = 'e';
      filename[leng] = '\0';
      if( !( out = fopen(filename,"w") ) ){
         fprintf(stdout,"tifl: Can't open file %s for output\n",filename);
         return EXIT_FAILURE;
      }
      treeprint( out, root(), 0 );

      /*---------------------------------------
        check if there are enough stacks available
        ---------------------------------------*/

      if ((np = number(_ParDecl_)) > MAX_NUM_PAR)
         { 
          printf("\n!!! RUNTIME error: too many parameters\n");
          getchar();
          return EXIT_FAILURE;
         }

      printf("\n+++ TIFL-Interpreter started!\n");
      printf("+++ stack version\n");
      printf("\ninput program evaluates to :  ");

      eval_result =  eval( son(1,root()) );
      if( sort(eval_result) == _Int )
        {
              printf("%d\n", etoi(eval_result) );
        }
      else
        {
         if( etob(eval_result) )
           {
            printf("true\n");
           }
         else
           {
            printf("false\n");
           }
        }

   }
   return EXIT_SUCCESS;
}

\end{verbatim}


\subsection{stack.c}
\begin{verbatim}

/*---------------------------------------------------------------------------

  stack.c           
                  
  ---------------------------------------------------------------------------*/

#define MAX_NUM_PAR 50
#define MAX_DEPTH 28
#define SP_INIT 0 

ELEMENT stacks[MAX_NUM_PAR][MAX_DEPTH];
int sp[MAX_NUM_PAR];

/*  can be omitted, since ANSI C */
void init_stacks()
{
 unsigned i;

 for( i=0 ; i<= MAX_NUM_PAR; i++)
    sp[i] = SP_INIT;

}


void pop_params( ELEMENT Parnode )
{
  while( Parnode != nil() )
       {
        sp[index(Parnode,_ParDecl_)]--;
        Parnode = rbroth(Parnode);
       }
}


void push_params( ELEMENT Parnode, ELEMENT Expnode )
{
  long indx;
  ELEMENT n;
  ELEMENT to_push[MAX_NUM_PAR];

  n = Parnode;
  while( n != nil() )
       {
        to_push[index( n, _ParDecl_ )] = eval( Expnode );
        Expnode = rbroth( Expnode );
        n = rbroth( n );
       }

  while( Parnode != nil() )
       {
        indx = index(Parnode, _ParDecl_);

        if( sp[ indx ] > MAX_DEPTH )
          {
           printf("\n!!! RUNTIME error: stack overflow\n");
           getchar();
           return;
          }

         sp[ indx ]++;
         stacks[ indx ][ sp[ indx ] ] = to_push[ index( Parnode, _ParDecl_ ) ];
         Parnode = rbroth(Parnode);
       }
}

   
ELEMENT par_lookup( ELEMENT Parnode )
{
  long indx = index(Parnode, _ParDecl_);
  return stacks[ indx ][ sp[ indx ] ];
}


ELEMENT eval_userfunc( ELEMENT Parnode, ELEMENT Expnode, ELEMENT Body )
{
   ELEMENT result;

   push_params( Parnode, Expnode );
   result = eval(Body);
   pop_params( Parnode );

   return result; 
}

\end{verbatim}



\end{document}

