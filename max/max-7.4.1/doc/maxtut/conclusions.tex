
\chapter{Conclusions}

The turotial presented an introduction to MAX and showed how the
system can be used to specify programming languages. The main advantages 
of the system compared to systems like MUG or GAG \cite{GAG} are (1) the
more general attribution constructs and (2) the flexible
functional back--end that can be used for further tasks of language
processing or refinement of specifications. Besides being a very
useful tool in its own rights, the system is meant to
serve as a kernel for even higher--level specification facilities (see 
e.g.~\cite{PH:diss}, \cite{PH:identif}) and for systems allowing
the specification of dynamic semantics and logic of programming 
languages (work in this area is under way; contact the first author)
in order to generate interpreters, debuggers, and programming environments
that enable queries about program properties.
Readers interested in details about the undelying formal framework are
refered to \cite{PH:maxsec}.

\vspace*{1cm}

\small
\begin{thebibliography}{KHZ82}

\bibitem[KHZ82]{GAG}
U.~Kastens, B.~Hutt, E.~Zimmermann.
\newblock {GAG}: A practical compiler generator.
\newblock Lecture Notes in Computer Science 141, 1982.

\bibitem[PH91]{PH:diss}
A.~Poetzsch-Heffter.
\newblock {\em Context--dependent syntax of programming languages: A new
  specification method and its application}.
\newblock PhD thesis (in German); Technische Universit\"at M\"unchen, 1991.

\bibitem[PH92a]{PH:identif}
A.~Poetzsch-Heffter.
\newblock Identifcation as Programming Language Principle.
\newblock Technical Report TUM--I9223, July 1992;
\newblock Technische Universit\"at M\"unchen, 1992.

\bibitem[PH92b]{PH:maxsec}
A.~Poetzsch-Heffter.
\newblock Programming Language Specification and Prototyping Using the 
MAX System.
\newblock Internal Report, September 1992 (submitted for publication);
\newblock Technische Universit\"at M\"unchen, 1992.

\end{thebibliography}
