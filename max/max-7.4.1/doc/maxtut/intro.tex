\chapter{Introduction}

This tutorial gives an introduction into the first version of the
MAX\footnote{MAX stands for {\bf M}unich {\bf A}ttribution system for
UNI{\bf X} environments.} system, a system to support 
programming language specification and implementation. The MAX 
system is based on an
entirely formal framework that properly generalizes attribute grammar like
frameworks. In addition to other features, the framework
\begin{itemize}
\item
provides free access to the syntax tree, thereby enabling the inspection
of distant attribute occurrences;
\item
allows attributes to have tree nodes as values; so
we can compute and represent global relations between distant parts of
the syntax tree, like the relation between applications and declarations;
\item
enables the formulation of context conditions by first--order predicate
formulae;
\item
provides a simple and purely functional interface between semantic analysis
and later tasks of language processing; e.g.~it provides an excellent basis
for recursively defined interpreters.
\end{itemize}
The notable aspect of the second feature is that it allows to define additional
edges in the syntax tree, which is very useful to represent identification
and type information.

Even though the MAX system was developped for programming
language specification and implementation, MAX is designed in a way that 
it can be used in many other applications. Given a specification,
the MAX system generates an analysor
that takes an abstract syntax term as input and produces a 
rich data structure based on the attributed syntax tree. This data
structure can be accessed through a powerful functional interface so 
that it can be directly used for further processing tasks.


The following chapter gives an introduction into the specification language
of MAX. The specification language allows to define abstract syntax
by order--sorted terms, and attributes, recursive functions, as well as
context conditions.
Chapter 3 explains the interface between MAX specifications and 
C--programs and shows how the MAX system is used in a UNIX environment.
The appendix contains the complete sources of the example specification,
we used throughout the tutorial to illustrate the explanations.


\vspace{1.5cm}

The MAX system is still under development. This tutorial attempts to 
help people in using the version 1.0 of the MAX system. It has to be
understood as an intermediate report. The authors thank in advance
for any questions,
corrections, comments, and suggestions concerning the system or the
tutorial. Thanks for support so far, go to J.~Eickel, W.~Heinrich,
A.~Liebl, S.~Schreiber, W.~Schreiber, U.~Vollath, H.~Wittner. 
