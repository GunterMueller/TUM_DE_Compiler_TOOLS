
\section{Die Widgethierarchie}

In Folgendem Abschnitt wird der von der Applikation aufgebaute 
Widgetbaum dargestellt.

Die eckigen Knoten enthalten den Widgetnamen und die Widgetklasse.
Ein Widget der Klasse {\tt TopLevelShell} das den Namen {\tt Shell}
hat, wird zum Beispiel durch den Knoteninhalt
{\tt ''Shell'' : topLevelShellWidgetClass } dargestellt.

Der Widgetname {\tt <Sortenname>} steht f"ur einen beliebigen
im MAX-Baum auftretenden Sortennamen.

Die runden Knoten sind Verweise auf weitere Widgethierarchieb"aume.
Sie wurden zur besseren "Ubersicht eingef"uhrt.

\begin{figure}[h]
   \begin{center}
      \leavevmode
      \psfig{figure=figures/widgeth.eps}
   \end{center}
   \caption{Widgethierarchie}
\end{figure}

\begin{figure}[h]
   \begin{center}
      \leavevmode
      \psfig{figure=figures/AttributList.eps}
   \end{center}
   \caption{AttributList}
\end{figure}

\begin{figure}[h]
   \begin{center}
      \leavevmode
      \psfig{figure=figures/GlobalMenu.eps}
   \end{center}
   \caption{GlobalMenu}
\end{figure}

\begin{figure}[h]
   \begin{center}
      \leavevmode
      \psfig{figure=figures/NodeMenu.eps}
   \end{center}
   \caption{NodeMenu}
\end{figure}

\begin{figure}[h]
   \begin{center}
      \leavevmode
      \psfig{figure=figures/MaskWindow.eps}
   \end{center}
   \caption{MaskWindow}
\end{figure}

\begin{figure}[h]
   \begin{center}
      \leavevmode
      \psfig{figure=figures/SortListWindow.eps}
   \end{center}
   \caption{SortListWindow}
\end{figure}

