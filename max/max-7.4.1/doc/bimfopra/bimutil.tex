\section{Funktionsbeschreibung des Moduls {\tt bim\_util.c}}

\begin{description}

\item[\tt Bim\_IsOfSort
]{\bf :\\}
Stellt fest, ob ein Browserknoten von einer bestimmten Sorte
 ist. Dazu wird der Sortenname des Knotens mit dem Parameter verglichen.
 Stimmen diese nicht "uberein, so werden die Sortennamen der Eltern
 verglichen.
 \\
IN:
\begin{itemize}
   \item MAX\_NODE * : Browserknoten, dessen Sorte "uberpr"uft werden soll.
char * name: Sortenname, auf den "uberpr"uft werden soll.

\end{itemize}
OUT:
\begin{itemize}
   \item int:	1 - Ist von dieser Sorte  \\
0 - Nicht von dieser Sorte

\end{itemize}
\item[\tt Bim\_CopyMaxTree\_Do
]{\bf :\\}
Kopiert einen Browserbaum.
  \\
IN:
\begin{itemize}
   \item MAX\_NODE * tree: Baum der kopiert werden soll.
\item MAX\_NODE * parent: Knoten, den der neue Baum als Vater erhalten
soll.

\end{itemize}
OUT:
\begin{itemize}
   \item MAX\_NODE * : Kopie des Baumes unter * tree, mit * parent als
Vater.

\end{itemize}

\item[\tt sort\_sort\_names
]{\bf :\\}
Sortiert eine Liste von Strings alphabetisch. Die Liste muss
 mit einem NULL-Pointer abgeschlossen sein.
  \\
IN:
\begin{itemize}
   \item char * list\_of\_sorts[ ] : Liste von Strings

\end{itemize}
\item[\tt Bim\_NumberOfSorts
]{\bf :\\}
Ermittelt die Anzahl der Sorten in der Grammatik
  \\
\item[\tt Bim\_NumberOfAttributes
]{\bf :\\}
Ermittelt die Anzahl der Attribute in der Grammatik
  \\
OUT:
\begin{itemize}
   \item int : Anzahl der Attribute

\end{itemize}

\item[\tt Bim\_ListOfAttributes
]{\bf :\\}
Gibt die sortierte Liste aller Attribute zur"uck.
  \\
OUT:
\begin{itemize}
   \item char **: Liste der Attributnamen

\end{itemize}

\item[\tt Bim\_ListOfSorts
]{\bf :\\}
Gibt die sortierte Liste aller Sortennamen zur"uck.
  \\
OUT:
\begin{itemize}
   \item char **: Liste der Sortennamen

\end{itemize}

\item[\tt Bim\_UpdateAttributes
]{\bf :\\}
Stellt beim ersten Erzeugen des Browserbaums die richtigen
 Knotenverweise her. Dazu wird die Tabelle der MAX-Knoten/Browser-Knoten
 Abbildungen herangezogen.
  \\
IN:
\begin{itemize}
   \item MAX\_NODE * tree: Zu durchlaufender Baum
\item MAX\_NODE * root: obsolet

\end{itemize}
OUT:
\begin{itemize}
   \item MAX\_NODE * : Wie Eingabe * tree.

\end{itemize}

\item[\tt ClearOriginNode
]{\bf :\\}
Setzt in einem Baum den Wert von origin auf NULL. Dies ist
 vor dem Anpassen der Attribute nach Kopieren des Baums
 notwendig.
  \\
IN:
\begin{itemize}
   \item MAX\_NODE * tree: Baum

\end{itemize}
\item[\tt Bim\_UpdateAttributesOnCopy
]{\bf :\\}
Nach dem Kopieren die Knotenverweise des Baums anpassen. Dazu
 wird das Element {\tt origin} ben"otigt, da\3 beim Kopieren gesetzt
 wurde.
  \\
IN:
\begin{itemize}
   \item MAX\_NODE * tree: Baum

\end{itemize}
OUT:
\begin{itemize}
   \item MAX\_NODE *: Wie Eingabe

\end{itemize}

\item[\tt Bim\_CopyMaxTree
]{\bf :\\}
Kopiere einen Browserbaum. Setze dazu im Ursprungsbaum alle
 {\tt orogin\_node} auf NULL, kopiere und korrigiere die Attributverweise
  \\
IN:
\begin{itemize}
   \item MAX\_NODE * tree: Zeiger auf Baum

\end{itemize}
OUT:
\begin{itemize}
   \item MAX\_NODE *: neuer Baum.

\end{itemize}
\end{description}
