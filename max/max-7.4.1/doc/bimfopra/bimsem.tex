\section{Funktionsbeschreibung des Moduls {\tt bimsem.c}}

Dieses Modul ist f"ur das Einlesen der Datei {\tt .g} zust"andig.
Es enth"alt die semantischen Aktionen, die von der Bison-Grammatik
aufgerufen werden.

\begin{description}
\item[\tt add\_production
]{\bf :\\}
F"ugt zur Liste der Sortenproduktionen eine neue
 Produktion hinzu.
 \\
IN:
\begin{itemize}
   \item int type: Gibt Art der Produktion an (Liste,Tupel,Klasse)
\item char * left: Zeiger auf linke Seite der Produktion
\item char * right: Zeiger auf rechte Seite

\end{itemize}
\item[\tt sem\_sublist
]{\bf :\\}
Hilft beim Aufbau der Subsorten einer Sorte.
 H"angt an den Buffer eine neue Sorte an, und gibt einen
 Zeiger zurueck, der hinter diese Sorte zeigt.
 \\
IN:
\begin{itemize}
   \item char * dest: Ziel
\item char * source: Ausgangssorte

\end{itemize}
OUT:
\begin{itemize}
   \item char *: Zeiger auf das Ende des Buffers, hier kann die
n"achste Sorte eingeh"angt werden.

\end{itemize}

\item[\tt add\_sort
]{\bf :\\}
F"ugt eine Sorte zur Liste der Sorten hinzu.
 Eine Sonderbehandlung f"ur Sorten vom Typ ''Node'' ist n"otig,
 da diese von nichts abgeleitet sein sollen und f"ur unsere
 Darstellung die Wurzel des Baumes sind.
 \\
IN:
\begin{itemize}
   \item char * name: Name der Sorte
\item struct SORT * parent: Zeiger auf Vatersorte. Ist dies
Null, so wird ''Node'' als Vatersorte angenommen

\end{itemize}
\item[\tt add\_attr
]{\bf :\\}
F"ugt ein Attribut zur Liste der Attribute hinzu.
 \\
IN:
\begin{itemize}
   \item char * asortname: Sorte, f"ur die dieses Attribut
definiert ist
\item char * aname: Attributname
\item char * rsortname: Ergebnissorte

\end{itemize}
\item[\tt add\_classsort
]{\bf :\\}
F"ugt eine Sorte, die eine Klasse ist, zur Liste
 der Sorten hinzu.
 \\
IN:
\begin{itemize}
   \item char * name: Sortenname

\end{itemize}
\item[\tt dump\_info
]{\bf :\\}
Gibt Informationen "uber die internen Tabellen aus.
 Wird nur zum Debuggen ben"otigt.
 \\
\item[\tt find\_sort
]{\bf :\\}
Sucht in der Tabelle der Sorteneintr"age nach einer
 Sorte mit angegebenem Namen.
 \\
IN:
\begin{itemize}
   \item char * name: Sortenname

\end{itemize}
OUT:
\begin{itemize}
   \item struct SORT *: Zeiger auf den Sorteneintrag oder NULL,
falls Eintrag nicht gefunden.

\end{itemize}
\end{description}
