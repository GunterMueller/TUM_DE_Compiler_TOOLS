\section{Funktionsbeschreibung des Moduls {\tt bimcode.c}}

Dieses Modul ist f"ur die Erzeugung des Files {\tt 42.c} zust"andig.

\begin{description}

\item[\tt semval\_typ
]{\bf :\\}
Ermittelt zu einem Typnamen den dazugeh"origen Typ des semantischen
 Werts. Gibt fuer Typen, die einen semantischen Wert haben, den Typ dieses
 Wertes als C-String zurueck. Typen, denen kein semantischer Wert zugeordnet
 ist, ergeben als Resultat einen NULL-Zeiger.
 \\
IN:
\begin{itemize}
   \item char * typ :	 Name des Typen

\end{itemize}
OUT:
\begin{itemize}
   \item char * : Name des Typen des semantischen Werts

\end{itemize}

\item[\tt actual\_attributes
]{\bf :\\}
Z"ahlt durch, wieviele Attribute eine Sorte wirklich hat, dazu
 summiert die Funktion die Anzahl die Attribute dieses Knoten und aller
 seiner Vorfahren.
 \\
IN:
\begin{itemize}
   \item struct SORT * p : Zeiger auf Sortendefinition

\end{itemize}
OUT:
\begin{itemize}
   \item int : Anzahl der Attribute

\end{itemize}

\item[\tt actual\_node\_attributes
]{\bf :\\}
Z"ahlt durch, wieviele Attribute eine Sorte wirklich hat, wenn
 es sich bei der Instanz um einen Knoten handelt.
 Die Funktion sumimert die Anzahl der Knoten-Attribute dieses Knoten und
 aller seiner Vorfahren.
 \\
IN:
\begin{itemize}
   \item struct SORT * p : Zeiger auf Sortendefinition

\end{itemize}
OUT:
\begin{itemize}
   \item int : Anzahl der Attribute

\end{itemize}

\item[\tt actual\_term\_attributes
]{\bf :\\}
Z"ahlt durch, wieviele Attribute eine Sorte wirklich hat, wenn
 es sich bei der Instanz {\bf nicht} um einen Knoten handelt.
 \\
IN:
\begin{itemize}
   \item struct SORT * p : Zeiger auf Sortendefinition

\end{itemize}
OUT:
\begin{itemize}
   \item int : Anzahl der Attribute

\end{itemize}

\item[\tt bimcode
]{\bf :\\}
Gibt den zu erzeugenden Code in die Datei 42.c aus.
 \\
IN:
\begin{itemize}
   \item char * bn: Name der Sprache

\end{itemize}
\item[\tt replace\_suffix
]{\bf :\\}
Ersetzt ein Dateisuffix durch ein anderes.
 \\
\item[\tt emit\_header
]{\bf :\\}
Gibt den Header der Zieldatei aus.
 \\
\item[\tt emit\_bim\_sort\_name
]{\bf :\\}
Gibt die Funktion aus, die zu jeder Sorte deren Namen als String
 zur"uckgibt
 \\
\item[\tt emit\_sort\_dependency
]{\bf :\\}
Gibt die Liste der Sortenabh"angigkeiten aus
 \\
\item[\tt emit\_bim\_nodeinfo
]{\bf :\\}
Gibt die Routine get\_nodeinfo() aus.
 \\
\item[\tt emit\_bim\_number\_of\_attributes
]{\bf :\\}
Gibt f"ur jede Sorte die Anzahl ihrer Attribute zur"uck.
 \\
\item[\tt emit\_sort\_dependencies
]{\bf :\\}
Gibt die Sortenabh"angigkeiten aus.
 \\
\item[\tt emit\_sort\_attributes
]{\bf :\\}
Gibt die Funktionen aus, die die Sorten berechnen sollen.
 \\
\item[\tt emit\_attr\_list
]{\bf :\\}
Gibt f"ur ein MAX-Element die Liste der Knoten zur"uck
 \\
\item[\tt emit\_list\_of\_sorts\_array
]{\bf :\\}
Gibt die Liste aller vorkommenden Sorten aus.
 \\
\item[\tt emit\_term\_parse\_table
]{\bf :\\}
Gibt die Parsertabelle aus, die zum Erstellen der Darstellung von
 Termattributen n"otig ist.
 \\
\item[\tt emit\_standard\_code
]{\bf :\\}
Kopiert den Code aus {\tt bim\_std.c} her"uber
 \\
\end{description}
