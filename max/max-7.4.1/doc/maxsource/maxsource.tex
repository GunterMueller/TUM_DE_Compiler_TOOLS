
\documentstyle[german,12pt,a4]{article}
\pagestyle{headings}

\begin{document}
\parskip 0.5em
\parindent 0pt

{\LARGE\bf\centerline{Kurze Beschreibung der MAX-Quellen}}

In Abschnitt~1 werden die neuen Attribute in der MAX-Spezifikation einschlie"slich
der in C implementierten externen Funktionen und Attribute beschrieben.
Abschnitt~2 behandelt die Codegenerierung der dynamischen Semantik.
In Abschnitt~3 wird die Realisierung der Frontendgenerierung beschrieben.
Schlie"slich beschreibt Abschnitt~4 die Ver"anderungen, die an der Codegenerierung
durchgef"uhrt wurden.


\bigskip (Roland Merk, 01.07.94)
\tableofcontents

\newpage
\section{Die MAX-Spezifikation in max\_spec.m}


\subsection{Neue MAX-Sorten}

{\tt FuncList} und {\tt AttrList} wurden zu\\
{\tt FuAtList * FctOrAtt\\
     FctOrAtt = Function | Attribute}\\
zusammengefa"st.


{\tt Expr} wurde um {\tt DynUndef} erg"anzt (das entspricht den "`??"').
{\tt ApplyEntity} wurde um {\tt DynFunc} und {\tt DynSemInput} erg"anzt.


\subsubsection{Neue Sorten f"ur die Abh"angigkeitsanalyse}

\begin{verbatim}
Value    = OrValue | FList | FClosure | FValue
OrValue  * Value
FList    * Value
FClosure ( Value )

FValue   = DefId@ | TWfath | TWson | TWrbroth | TWlbroth | RelevAcc
TWson    ( Int )
TWfath   ()
TWrbroth ()
TWlbroth ()
RelevAcc ( Int )


FctOrAtt          = Function | Attribute
FctOrAttDecl      = Function | Attribute | FuncDecl | AttrDecl
OrderTup          = Int

ApplList          * FuncAppl@
BinOp             = Or | And | Impl

FuAtListInfo      * FuAtNodeList
FuAtNodeList      * FctOrAtt@

AllNodes          ()
DefIdNodeInfo     = DefIdNodeList | AllNodes

IntResExpr        = ComposedExpr | FuncAppl | ApplyEntity | Condition
IntResOrName      = IntResExpr | Name
IntResList        * IntResOrName@
IntList           * Int
RefList           * Reference
\end{verbatim}

\subsubsection{Neue Sorten f"ur die Frontendgenerierung}
\begin{verbatim}
ProdNodeList * Production@
IdentList    * Ident
CharList     * Char

Rside        = SortId | String | Parsnr | Parsline | Parscol | Parspos
Parsnr       ( Int )
Parsline     ()
Parscol      ()
Parspos      ()
RsideInfo    * Rside
RsideList    * RsideInfo

Lside        = Parsrule | Parsaddrule | Parsno | Parsaux |
               Parsleft | Parsright | Parsprio | Int
LsideList    * Lside

Parsprio     ()
Parsrule     ( StringList )
Parsaddrule  ( StringList )
Parsno       ()
Parsleft     ()
Parsright    ()
Parsaux      ()

LPinfoListlist * LPinfolist
LPinfolist     * LPinfo
LPinfo         = ParsStern | ParsPlus
ParsStern      ( String )
ParsPlus       ( String )

IdentTripel    ( Ident Ident Ident )
StringList     * String
CPinfo         ( StringList SortId StringList )
CPinfoList     * CPinfo
CPinfoListlist * CPinfoList
\end{verbatim}

\subsubsection{Neue Sorten f"ur die dynamische Semantik}

\begin{verbatim}
DynFunc        ( DefId  SortIdList  NameList  SortId  Expr )

RuleList       * Rule
Rule           = IfRule | CaseRule
IfRule         ( Formula  UpdateList  UpdateList )

CaseRule       ( Expr  CaseIsList  UpdateList )
CaseIsList     * CaseIs
CaseIs         ( CaseConst UpdateList )
CaseConst      = Char | Int | SortConst

UpdateList     * UpdateOrRule
UpdateOrRule   = Update  |  Rule
Update         ( FuncAppl  Expr )

UListNodeList  * UpdateList@

DynUndef       ()
DynSemInput    ( DefId  SortIdList  SortId )

CaseTripel     ( DefIdNodeList DefIdNodeList UpdateList@ )
CaseTripelList * CaseTripel
\end{verbatim}

\subsection{Neue Attribute f"ur die Frontendgenerierung}

{\tt
\begin{itemize}
\item ATT parscode     ( Production@ ) Lside
\item ATT LPsepsymbols ( ListProd@   ) LPinfolist
\item ATT TPconcrSyntax( TupelProd@  ) RsideList
\item ATT CPrsideinfo  ( ClassProd@  ) CPinfo
\item {\rm dazwischen Behandlung der Deklarationen (i.w.\ unver"andert)}
\item ATT ExchangeTripel( TupelProd@ ) IdentTripel {\rm (braucht "`decl"')}
\item ATT ExtraRule( Production@ ) Element  (Ident {\rm oder} ClassProd@)
\end{itemize}
}

Die Implementierung dieser Attribute befindet sich in der Datei {\tt max\_ygen.c}.
Beschreibung siehe dort (Abschnitt \ref{frontendgen}).

\subsection{Diverse Funktionen und Attribute der alten Version}

Hier folgt die Attributierung f"ur Sammeln der Grundsorten bei Klassenproduktionen,
Bereichs"uberpr"ufung der angewandten Bezeichner und Bestimmung ihrer Bindungsstellen
sowie die gesamte Behandlung der Patterns.

\subsection{Funktionen zum Aufbau der Value-Terme}

{\tt
\begin{itemize}
\item FCT TW\_identabb() Value
\item FCT TW\_emptyset() Value
\item FCT TW\_or( Value, Value ) Value
\item FCT TW\_stern( Value ) Value
\item FCT TW\_conc( Value, Value ) Value
\end{itemize}
}

\subsection{Aufsammeln der FuncAppl@ in den R"umpfen von Funktionen und Attributen}

{\tt
\begin{itemize}
\item {\rm diverse Hilfsfunktionen f"ur die bottom-up--Aufsammlung}
\item ATT appl\_list( FctOrAtt@ ) ApplList
\end{itemize}
}

"`appl\_list"' wird in "`eval\_groupnr"' (in der Hilfsfunktion "`enter\_dep\_matrix"')
bei der Einteilung in Auswertungsgruppen verwendet.


\subsection{Analyse der zyklischen Abh"angigkeiten der Funktionen und Attribute}

Die folgenden Attribute sind alle in der Datei "`max\_extrn.c"' in C implementiert.
Jede Liste von Funktionen und Attributen wird in m"oglichst kleine Gruppen
partitioniert und sortiert.

\begin{itemize}
\item {\tt ATT eval\_groupnr ( FctOrAtt@ ) Int}\\
gibt an, in welcher Auswertungsgruppe eine Funktion oder ein Attribut ist.
\item {\tt ATT eval\_prev\_grnum( FuAtList@ ) Int}\\
gibt die Nummer der letzten Auswertungsgruppe der letzten {\tt FuAtList} an.

\item {\tt ATT group\_index   ( FctOrAtt@ ) Int}\\
gibt die lfd.\ Nummer innerhalb einer Auswertungsgruppe an.

\item {\tt ATT eval\_group    ( FctOrAtt@ ) FuAtNodeList}\\
gibt die gesamte Auswertungsgruppe (incl.\ enthaltener Funktionen) an.

\item {\tt ATT attr\_list\_groups( FuAtList@ ) FuAtListInfo}\\
gibt die Liste aller Auswertungsgruppen der "`FuAtList"' an.

\end{itemize}

Die Funktion "`init\_eval\_groupnr"' initialisiert
"`eval\_groupnr"' und "`eval\_prev\_grnum"'.

\paragraph{Vorgehensweise}

Der Aufrufgraph innerhalb der Gruppe und die H"ullenbildung
wird mit einer Adjazenzmatrix und dem Warshall-Algorithmus (f"ur short-Objekte)
implementiert.
Die Hilfsfunktion "`enter\_dep\_matrix"' tr"agt alle Aufrufe im Rumpf einer
Funktion in die Matrix ein. Die Hilfsfunktion "`calc\_order"' teilt die
Gruppennummern zu.


Die Funktion "`init\_group\_index"' initialisiert unter Verwendung obiger Information
die Attribute "`group\_index"', "`eval\_group"' und "`attr\_list\_groups"'.



\subsection{Aufbau der Value-Terme}

\begin{itemize}
\item diverse Hilfsfunktionen
\item {\tt ATT appl\_value( FuncAppl@ ) Value}\\
liefert den Beitrag eines Funktionsaufrufs zum Value-Term
\item {\tt ATT nodepar\_expr( FuncAppl@ ) Expr@}\\
liefert den ersten knotenwertigen Parameter
\item {\tt ATT valueterm( Expr@ ) Value}\\
gesamter Value-Term einer Expression
\end{itemize}


\subsection{Elimination der Rekursion}

Hier wird mit Value-Termen innerhalb der Auswertungsgruppen gerechnet
(vgl.\ Direct Dependence Automaton).

\begin{itemize}
\item {\tt FCT init\_matrix( Int ) Reference}\\
Speicherreservierung f"ur eine Matrix der doppelten (!) Gr"o"se wie angegeben.
Die Eintr"age werden mit {\tt TW\_emptyset()} vorbesetzt.

\item {\tt FCT enter\_attribute( Int, FctOrAtt@, Reference, Int ) Reference}\\
"`f"ullt"' die Matrix gem"a"s des Rumpfes der Funktion oder des Attributs
mit Value-Termen, die nur noch aus elementaren Baumlauffunktionen bestehen.

\item {\tt FCT warshall\_value( Reference, Int ) Reference}\\
Warshall-Algorithmus f"ur die H"ullenbildung im Automaten. Die letzten
drei Funktionen sind in {\tt max\_extrn.c} implementiert.

\item {\tt FCT enter\_attrgroup( Int, FuAtNodeList, Reference ) Reference}
\item {\tt FCT enter\_attrgroup\_list( Int, FuAtListInfo, RefList ) RefList}
\item {\tt ATT attr\_list\_matlist( FuAtList@ ) RefList}\\

Diese drei Hilfsfunktionen k"ummern sich um das Eintragen der Value-Terme der
Funktions- und Attributr"umpfe in die Matrix und um das Entrekursivieren mit dem
Warshall-Algorithmus (f"ur jede Auswertungsgruppe).

\item {\tt FCT mat\_lookup( Reference, Int, Int, FctOrAtt@ ) Value}\\
externe Hilfsfunktion f"ur den Matrixzugriff.

\item {\tt ATT result\_value( FctOrAtt@ N ) Value}\\
liefert den abgesch"atzten Wert des Rumpfes der Funktion
oder des Attributs. Es kommen nur noch elementare Baumlauffunktionen vor.
\end{itemize}



\subsection{Vereinfachung der Value-Terme auf eine Normalform}

Soweit es sich mit relativ wenig Aufwand machen l"a"st, werden die Value-Terme
vereinfacht. F"ur die Abh"angigkeitsanalyse werden die Laufzeitwerte der
{\tt FuncAppl@} ben"otigt, die durch Zugriff auf {\tt result\_value} hier eingehen.

\paragraph{Hilfsfunktionen}

{\tt
\begin{itemize}
\item FCT simplify\_closure( Value ) Value
\item PRD simplification[ Value, FValue ]
\item FCT normal\_tw( FList, Value, FValue ) FList
\item FCT normal\_stern( FList, FClosure ) Value
\item FCT normalform( FList, Value, FuAtList@ ) Value
\end{itemize}
}

\subsubsection*{Laufzeitwert des ersten knotenwertigen Parameters eines Funktionsaufrufs}

Dieser Wert wird durch folgenden Aufruf erhalten (wird sp"ater beim Attribut
{\tt appl\_order} benutzt). Dabei soll N der {\tt FuncAppl@} des Aufrufs sein.

Bei einem Attributzugriff gibt dieser Wert an, auf welche Baumknoten zugegriffen
werden kann.

\begin{verbatim}
normalform( FList(),
            valueterm(nodepar_expr(N)),
            fath(encl_fctoratt(N))      )
\end{verbatim}


\subsection{Abbildung der Value-Terme auf Baumordnungen}

\subsubsection*{Ordnungen f"ur elementare Funktionen}

{\tt
\begin{tabular}{lllll}
$\bullet$ & FCT Ord\_fath()    & OrderTup : & 35  & // AA BB = 0x23\\
$\bullet$ & FCT Ord\_son()     & OrderTup : & 50  & // BB AA = 0x32\\
$\bullet$ & FCT Ord\_rbroth()  & OrderTup : & 16  & // AB 00 = 0x10\\
$\bullet$ & FCT Ord\_lbroth()  & OrderTup : &  1  & // 00 AB = 0x01\\
$\bullet$ & FCT Ord\_ident()   & OrderTup : &204  & // ID ID = 0xCC\\
$\bullet$ & FCT Ord\_neutral() & OrderTup : & 17  & // AB AB = 0x11\\
$\bullet$ & FCT Ord\_nothing() & OrderTup : &136  & // NO NO = 0x88\\
\end{tabular}
}

\paragraph{Bedeutung der {\tt OrderTup}}

\begin{itemize}
\item "`Sechzehnerstelle"' (hexadezimal) :\\
Strategie, falls die {\it Points} vorw"arts durchlaufen werden
\item "`Einerstelle"' :\\
Strategie, falls die {\it Points} r"uckw"arts durchlaufen werden
\item F"ur beide F"alle gilt : ist "`8"' gesetzt (d.h.\ {\tt 0x80} bzw.\ {\tt 0x08}),
dann ist evtl.\ ein Zugriff am selben Baumknoten m"oglich.
\end{itemize}

\begin{itemize}
\item {\tt NO = 0}: keine Strategie m"oglich
\item {\tt AB = 1}
\item {\tt AA = 2}
\item {\tt BB = 3}
\item {\tt ID = 4}
\item {\tt BA = 5} : geben die M"oglichkeiten an, welchen {\it Points}
das gerade berechnete bzw.\ dazu ben"otigte Attribut f"ur einen erfolgreichen
Durchlauf zugeordnet werden k"onnen.
\end{itemize}

\paragraph{Verkn"upfungen der Baumordnungen}
{\tt
\begin{itemize}
\item FCT OrdOp\_or( OrderTup, OrderTup ) OrderTup
\item FCT OrdOp\_conc( OrderTup, OrderTup ) OrderTup
\item FCT OrdOp\_closure( OrderTup ) OrderTup
\end{itemize}
}

\paragraph{Restliche Funktionen}

\begin{itemize}
\item {\tt FCT val\_order( Value ) OrderTup}\\
Hauptfunktion, die die Abbildung {\tt Value} $\to$ {\tt OrderTup} induktiv
"uber den Termaufbau jedes Value-Terms in Normalform durchf"uhrt.

\item {\tt FCT encl\_fctoratt( Node ) FctOrAtt@}\\
liefert den umschlie"senden {\tt FctOrAtt@}

\item {\tt PRD is\_relev\_att\_access[ FuncAppl@ ]}\\
testet, ob ein Attributzugriff in derselben Auswertungsgruppe
erfolgen kann (evtl.\ indirekt "uber andere Funktionen).


\item {\tt ATT appl\_order( FuncAppl@ ) OrderTup}\\
liefert die m"oglichen Baumdurchlaufordnungen, bei denen jeder Attributzugriff
immer gutgeht, d.h.\ jede ben"otigte Instanz wurde schon fr"uher w"ahrend
des Durchlaufs berechnet.

\item {\tt FCT filter\_relev\_acc( ApplList ) ApplList}\\
filtert diejenigen {\tt FuncAppl@} heraus, die die m"oglichen
Baumdurchlaufstrategien echt einschr"anken.

\item {\tt ATT appl\_relacc\_list( FctOrAtt@ ) ApplList}\\
liefert zum jedem Attribut und zu jeder Funktion eine Liste von {\tt FuncAppl@},
die die m"oglichen Baumdurchlaufstrategien echt einschr"anken.
\end{itemize}




\subsection{Auswahl der Strategien f"ur die Auswertungsgruppen}

Bis auf {\tt ATT maybe\_wait} sind alle Attribute und Pr"adikate
dieses Abschnitts in {\tt max\_extrn.c} implementiert.

\paragraph{Beachte}: Alle folgenden Attribute werden f"ur Funktionen und Attribute
ausgewertet. Das liegt daran, da"s die rekursive Verschr"ankung von Funktionen
und Attribute erlaubt ist. Dabei wird bei jeder Funktion der erste knotenwertige
Parameter ausgezeichnet.

Bei der Strategieberechnung wird jede Funktion wie ein Attribut behandelt.
Erst sp"ater bei der Codegenerierung der Attributauswertung werden die Funktionen
wieder unterdr"uckt.

\begin{itemize}
\item {\tt ATT eval\_finegroup( FctOrAtt@ ) Int}\\
gibt die Nummer der Feingruppe innerhalb der aktuellen Auswertungsgruppe an.

\item {\tt ATT eval\_aftbef   ( FctOrAtt@ ) Int}\\
gibt an, welchem {\it Point} das Attribut zugeordnet wird.

\item {\tt ATT eval\_strategy ( FctOrAtt@ ) Int}\\
gibt die ausgew"ahlte Durchlaufstrategie an (siehe unten).

\item {\tt ATT eval\_sortlist ( FctOrAtt@ ) DefIdNodeInfo}\\
gibt an, aus welchen Grundsorten der Definitionsbereich des Attributes besteht.

\item {\tt ATT attr\_list\_info( FuAtList@  ) FuAtListInfo}\\
enth"alt eine Liste der Attributauswertungsgruppen.

\item {\tt PRD maybe\_same\_node[ OrderTup ]}
\item {\tt PRD no\_strategy\_found[ FctOrAtt@ ]}\\
geben an, ob ein Zugriff am selben Knoten m"oglich ist bzw.\ ob keine Durchlaufstrategie
gefunden wurde.

\item {\tt ATT maybe\_wait( FuncAppl@ ) Bool}\\
gibt an, ob beim Zugriff auf eine Attributinstanz ein Warten auf
die Wertberechnung m"oglich ist.

\end{itemize}

\subsubsection{Strategieberechnung}

Jede Auswertungsgruppe bekommt eine Auswertungsstrategie zugeordnet.
Bei der Initialisierung von {\tt ATT  attr\_list\_info}
wird als Seiteneffekt die Strategie berechnet.

Im allgemeinen ist die Strategie innerhalb jeder Gruppe gleich,
jedoch kann es Ausnahmen geben,
falls die Gruppe in weitere Feingruppen unterteilt ist,
die beide Auswertungsvarianten (sowohl mit als auch ohne Wartelisten) verlangen.

Die Strategie wird als {\tt Int} implementiert, wo die einzelnen Bits folgende
Bedeutung haben :

\begin{itemize}

\item 0 = keine Strategie gefunden
\item 1 = Aufz"ahlung mit Heap (schlie"st 2 aus)
\item 2 = direkte Aufz"ahlung (schlie"st 1 aus)

\item 4 = partielle Auswertung

\item 8 = Aufz"ahlung der {\it Points} r"uckw"arts

\item 16 = FCT und ATT kommen in der Gruppe gemischt vor

\item 32 = eine Warteliste ist notwendig
\end{itemize}

\subsection{Vorbereitung der Codegenerierung}


\begin{itemize}
\item {\tt FCT aux\_itoa( Int ) String}\\
Umwandlung vom Integer in String (in {\tt max\_extrn.c}).

\item {\tt ATT cgen\_name( IntResOrName@ ) String}\\
Bezeichnername eines Zwischenergebnisses im generierten Code.

\item {\tt PRD is\_intres[ Node ]}\\
gibt an, ob einem Baumknoten ein Zwischenergebnis zugeordnet wird.

\item {\tt FCT  list\_intresexprs( Node ) IntResList}
\item {\tt FCT  intresexprs( Node ) IntResList}\\
Hilfsfunktionen zum Aufsammeln der Zwischenergebnisse, die im generierten
Code im Funktionsterm vorkommen werden.
\end{itemize}



\subsection{Bestimmung der Zwischenergebnisse}

Alle diese Attribute sammeln die Zwischenergebnisse f"ur die {\tt Expr@} auf.

{\tt
\begin{itemize}
\item ATT  sces\_appl  ( ApplyEntity@ ) IntResList 
\item ATT  sces1\_cond ( Condition@   ) IntResList 
\item ATT  sces2\_cond ( Condition@   ) IntResList 
\item ATT  sces1\_let  ( LetExpr@     ) IntResList 
\item ATT  sces2\_let  ( LetExpr@     ) IntResList 
\item ATT  sces\_if    ( IfExpr@      ) IntResList 
\item ATT  sces1\_fcase( FormulaCase@ ) IntResList 
\item ATT  sces2\_fcase( FormulaCase@ ) IntResList 
\item ATT  sces\_pcase ( PatternCase@ ) IntResList 
\item ATT  sces\_attacc( FuncAppl@    ) IntResList 
\end{itemize}
}

\subsection{Aufsammeln der Zwischenergebnisse}


\begin{itemize}
\item {\tt FCT encl\_part\_intres( Node ) Node}\\
liefert das umschlie"sende Zwischenergebnis im Syntaxbaum.

\item {\tt FCT aux\_var\_need( Int, FctOrAtt@ ) IntResList}\\


\item {\tt FCT aux\_intres\_left( IntResExpr@, Int, IntResList ) IntResList}
\item {\tt FCT aux\_intres\_right( Point, Point, Point, IntResList, Int ) IntResList}
\item {\tt ATT lokvar\_need( FctOrAtt@ ) IntResList}\\
Hilfsfunktionen zum Aufsammeln der Zwischenergebnisse, die bei der partiellen
Auswertung gespeichert werden m"ussen.

\item {\tt FCT intres\_union( IntResList, IntResList, IntResList ) IntResList}\\
Vereinigung dreier Mengen von Zwischenergebnissen (in {\tt max\_extrn.c})

\item {\tt ATT intres\_collect( IntResExpr@ ) IntResList}\\
ergibt f"ur jedes Zwischenergebnis die Liste von Zwischenergebnissen,
die f"ur seine Berechnung ben"otigt werden.
Dieses Attribut wird bei der Codegenerierung (Datei {\tt max\_sgen.c}) verwendet.
\end{itemize}


%%%%%%%%%%%%%%%%%%%%%%%%%%%%%%%%%%%%%%%%%%%%%%%%%%%%%%%%%%
\newpage
\section{Dynamische Semantik in max\_dgen.c}

\begin{itemize}
\item {\tt void gen\_dynhash()}\\
erzeugt Code f"ur die gesamte Verwaltung der Hashtabelle.

\item {\tt ELEMENT DNL\_insert( ELEMENT, ELEMENT )}\\
Hilfsfunktion zum Einf"ugen von {\tt DefId@} in eine Liste.

\item {\tt long case\_elem( ELEMENT, ELEMENT )}\\
Hilfsfunktion, die testet, ob ein Element als Subterm
in einem anderen vorkommt.

\item {\tt long contains\_term( ELEMENT )}\\
testet, ob {\tt Term} in einer Liste von {\tt DefId@} vorkommt.

\item {\tt void gen\_guards\_case( ELEMENT, int )}\\
erzeugt das "`case"'-Konstrukt f"ur die Updates.

\item {\tt void gen\_guards\_UR( ELEMENT, int )}\\
erzeugt Code f"ur eine (allgemeine) Update-Regel.

\item {\tt void gen\_guards\_UL( ELEMENT, int )}\\
erzeugt Code f"ur eine Update-Liste.

\item {\tt void gen\_dynadr( ELEMENT )}\\
erzeugt eine Funktion, die die Adresse eines Elements ermittelt (f"ur
die linke Seite der Updates notwendig).

\item {\tt void gen\_dyndata( void )}\\
erzeugt alle globalen Variablen, die f"ur die dynamische Semantik
gebraucht werden (ausgenommen der Hashtabelle).

\item {\tt void aux\_prt\_lokvar( ELEMENT, int )}
\item {\tt int genc\_dynfct\_aux( ELEMENT, int )}\\
erzeugen alle Hilfsvariablen, die im Rumpf der Initialisierung
einer dynamischen Funktion ben"otigt werden.

\item {\tt void gen\_init\_dynsem()}\\
erzeugt Code f"ur die Initialisierung aller dynamischen Gr"o"sen.

\item {\tt void dyn\_defines( FILE * )}\\
erzeugt alle {\tt \#define}'s, die f"ur den Zugriff auf die dynamischen
Gr"o"sen ben"otigt werden.

\item {\tt void dynsem\_gen( FILE * )}\\
steuert die gesamte Ausgabe des Codes f"ur die dynamische Semantik.
\end{itemize}

%%%%%%%%%%%%%%%%%%%%%%%%%%%%%%%%%%%%%%%%%%%%%%%%%%%%%%%%%%%%%%%%%%%%%%%%%
\newpage
\section{Frontendgenerierung in max\_ygen.c}
\label{frontendgen}

{\tt int get\_hosttype()}\\
pr"uft die Environmentvariablen "`{\tt HOST}"' und "`{\tt HOSTTYPE}"'.
Dadurch werden bei der Makefile-generierung, die auch hier eingegliedert ist,
die richtigen Kommandos f"ur den C-Compiler eingesetzt und das
"`SECONDBITSET"'-Symbol bei HP-Rechnern definiert.

\subsection{Notwendige Seiteneffekte beim Parsen}

Die Funktionen
{\tt mxy\_enterlp}, {\tt mxy\_entercp} und {\tt mxy\_entertp}\\
werden beim Baumaufbau verwendet. Sie tragen die notwendige Information
f"ur die konkrete Syntax als Seiteneffekt in untenstehende Listen ein.
Das Ergebnis des Parsens ist immer der abstrakte Syntaxbaum sowie
diese sechs Listen.

\begin{verbatim}
static ELEMENT LPcodes = LsideList(),  /* Init vor dem Parsen */
               TPcodes = LsideList(),
               CPcodes = LsideList();

static ELEMENT LPrside = LPinfoListlist(),
               TPrside = RsideList(),
               CPrside = CPinfoListlist();
\end{verbatim}

\subsection{Bereitstellung der konkreten Syntax in Attributen}

\begin{itemize}
\item {\tt ELEMENT parscode( Production@ )}\\
enth"alt die entsprechenden Subterme von
{\tt LPcodes}, {\tt TPcodes} und {\tt CPcodes}.

\item {\tt ELEMENT LPsepsymbols( ListProd@ )}\\
enth"alt die Trennsymbole der Listenproduktionen ({\tt LPrside}).

\item {\tt ELEMENT TPconcrSyntax( TupelProd@ )}\\
enth"alt die konkrete Syntax von Tupelproduktionen ({\tt TPrside}).

\item {\tt ELEMENT CPrsideinfo( ClassProd@ )}\\
enth"alt die konkrete Syntax von Klassenproduktionen ({\tt CPrside}).

\item {\tt ELEMENT ExchangeTripel( TupelProd@ )}\\
gibt Abweichungen bei {\tt \#prio}-Produktionen an, welche Sorten
beim Parsen durch andere ersetzt werden m"ussen.

\item {\tt ELEMENT ExtraRule( Production@ )}\\
gibt an, ob bei {\tt \#prio}-Produktionen zus"atzliche Kettenproduktionen
ausgegeben werden m"ussen.
\end{itemize}


\subsection{Initialisierung von Sortenmengen}

\begin{itemize}
\item {\tt long elem( ELEMENT, ELEMENT )}\\
testet, ob ein Element als Subterm in einem anderen vorkommt.

\item {\tt void grsym\_in\_parstree( ELEMENT )}\\
sucht alle von einer gegebenen Sorte abh"angigen Sorten (die
tiefer im Syntaxbaum vorkommen k"onnen).

\item {\tt void init\_grammarsymbols( void )}
sucht alle Sorten, die beim Parsen auftreten k"onnen (mit der Hilfsfunktion
"`{\tt grsym\_in\_parstree}"').
\end{itemize}

\subsection{Aufsammeln von Symbolen}

In folgenden globalen Variablen werden die notwendigen Symbole aufgesammelt.
\begin{verbatim}
static ELEMENT SemTokenList, TokenList, CharList,
               PreDeclProdList, ReachableProdList;
\end{verbatim}

\begin{itemize}
\item {\tt SemTokenList}: semantisch relevante Token (mit {\tt \#mark} markiert),
bei denen Zeile und Spalte in den Syntaxbaum aufgenommen werden.
\item {\tt TokenList}: andere Token (L"ange $>1$).
\item {\tt CharList}: Token der L"ange 1.
\item {\tt PreDeclProdList}: vordefinierte Sorten, die im Syntaxbaum vorkommen k"onnen.
\item {\tt ReachableProdList}: andere erreichbare Sorten, die im Syntaxbaum vorkommen k"onnen.
\end{itemize}

\paragraph{Bemerkung:}
f"ur Strings mit einer L"ange $>1$ mu"s im YACC-File ein eigenes
Token reserviert werden, bei L"ange $1$ kann der String als Character
verwaltet werden.

\begin{itemize}
\item {\tt void addstring( char *, int )}\\
f"ugt einen String zur Tokenliste hinzu.

\item {\tt void addtoken( char * )}\\
Bei Produktionen mit {\tt \#yacc} und {\tt \#yaccadd} werden alle in
der YACC-Regel vorkommende Strings (in einfachen Hochkomma) erkannt.

\item {\tt void init\_aux\_lists( void )}\\
k"ummert sich um das Aufsammeln aller notwendigen Symbole (wird von
{\tt parser\_gen} aufgerufen).
\end{itemize}

\subsection{Scannergenerierung}

\begin{itemize}
\item {\tt void lexprintf( char * )}\\
k"ummert sich um die Ausgabe eines Tokens in das LEX-File.
Dabei werden f"uhrende Backslashes herausgefiltert.

Beispiel: \verb|"--"[-]*| mu"s als \verb|"\\\"--\"[-]*"| eingegeben werden.

\item {\tt void mxy\_lexgen( void )}\\
steuert die Ausgabe des gesamten Scanners.
\end{itemize}

\subsection{Parsergenerierung}

\begin{itemize}
\item {\tt void yacctoken\_print( FILE *, int ) }
\item {\tt void yacctoken\_semprint( FILE *, int )}
\item {\tt void yaccnont\_print( FILE *, char *, char *, int, int )}\\
Ausgabe von YACC-Definitionen ({\tt \%token} bzw.\ {\tt \%type})
f"ur Token und Nonterminale.

\item {\tt void mxy\_yaccrule( ELEMENT, FILE * )}\\
Ausgabe der mit {\tt \#yacc} und {\tt \#yaccadd} definierten YACC-Regeln.

\item {\tt int print\_string( ELEMENT )}\\
Hilfsfunktion zur Ausgabe eines Strings als Character bzw.\ Token.

\item {\tt void additional\_rule( ELEMENT )}\\
Ausgabe einer zus"atzlich notwendigen YACC-Regel (bei {\tt \#prio}-Produktionen).

\item {\tt void LPout( ELEMENT )}\\
Ausgabe der Regeln f"ur eine gesamte Listenproduktion.

\item {\tt void CPout( ELEMENT )}\\
Ausgabe der Regeln f"ur eine gesamte Klassenproduktion.

\item {\tt int isnorekpos( int, ELEMENT, ELEMENT, ELEMENT )}\\
pr"uft, ob bei einer {\tt \#prio}-Produktion die Gesamtsorte rekursiv
verwendet wird.

\item {\tt void TPout( ELEMENT )}
Ausgabe der Regeln f"ur eine gesamte Tupelproduktion.

\item {\tt void mxy\_yaccgen( char * )}\\
steuert die Ausgabe des gesamten Parsers.

\item {\tt void parser\_gen( FILE *, FILE *, int, char * )}\\
erzeugt den Scanner und den Parser als LEX- bzw.\ YACC-File.
\end{itemize}

\subsection{Generierung von Rahmenprogramm und Makefile}

\begin{itemize}
\item {\tt void frame\_gen( FILE *, char * )}\\
erzeugt das Hauptprogramm der Anwendung.

\item {\tt void makefile\_gen( FILE *, char * )}\\
erzeugt ein Makefile.

\end{itemize}

%%%%%%%%%%%%%%%%%%%%%%%%%%%%%%%%%%%%%%%%%%%%%%%%
\newpage
\section{Codegenerierung in max\_sgen.c}


\subsection{Besonderheiten bei der Funktion prt\_stdpart}


\paragraph{M"ogliche Attributauswertungsfunktionen}

\begin{enumerate}
\item Auswertungsfunktion und Wartelistenelement bei partieller Auswertung

{\tt typedef void (*EVALFCT)( ELEMENT, ELEMENT, int, ELEMENT * );}

\begin{verbatim}
typedef struct AttInstElem{
  EVALFCT attevalfct;
  ELEMENT  attnode;
  struct AttInstElem *rest;

  int state;
  ELEMENT intres_array[ <maxsize> ];
  /* fehlt, wenn keine Zwischenergebnisse ben"otigt werden */

} *AttInstSq;


static ELEMENT *insAttInst( EVALFCT attevalfct, ELEMENT attnode,
        int state, AttInstSq* undefatt, long undefattix ) { ... }
\end{verbatim}

Falls keine Zwischenergebnisse ben"otigt werden, wird die Funktion als {\tt void}
deklariert (wie ohne partielle Auswertung).

Falls Zwischenergebnisse ben"otigt werden, zeigt der R"uckgabepointer auf
das {\tt intres\_array} im Wartelistenelement, wo die Zwischenergebnisse
eingetragen werden k"onnen.

\item Auswertungsfunktion und Wartelistenelement ohne partieller Auswertung

{\tt typedef void (*EVALFCT1)( ELEMENT );}

\begin{verbatim}
typedef struct AttInstElem{
  EVALFCT attevalfct;
  ELEMENT  attnode;
  struct AttInstElem *rest;
} *AttInstSq;

static void insAttInst( EVALFCT attevalfct, ELEMENT attnode,
	AttInstSq* undefatt, long undefattix ) { ... }
\end{verbatim}

\end{enumerate}

\paragraph{Bemerkung}: in der {\tt struct AttInstElem} steht immer {\tt EVALFCT},
da beide Varianten ({\tt EVALFCT} und {\tt EVALFCT1}) auch gemischt vorkommen k"onnen.
Dies ist der Fall, wenn Funktionen und Attribute gegenseitig rekursiv verwendet werden.

\subsubsection*{Unver"anderte Funktionen}

{\tt prt\_listtoelemtab,
prt\_classtosorttab,
init\_supsorts,\\
prt\_supsorttab {\rm und}
prt\_indexingtabs
}

\paragraph{Die Funktion {\tt prt\_heap\_fcts}} gibt Code f"ur die Heapverwaltung aus.


\paragraph{Die Funktion {\tt prt\_stack\_fcts}} gibt Code f"ur den Kellerautomat aus,
der die Knoten f"ur den R"uckw"artsdurchlauf indiziert.

\paragraph{Funktionen f"ur die Extern Pattern Class:}
{\tt genc\_pat\_<diverse>(...)}

%%%%%%%%%%%%%%%%%%%%%%%%%%%%%%%%%%%%%%%%%%%%%%%%%%%%%%%%%%%%%%%%%%%%%%%%

\subsection{Funktionen f"ur die Codegenerierung der Funktions- und Attributr"umpfe}

\begin{itemize}
\item {\tt static void genc\_addpar( ELEMENT, ELEMENT )}\\
gibt evtl.\ zus"atzliche Parameter aus, wenn Funktionen und Attribute
rekursiv verschr"ankt sind.

\item {\tt void  genc\_list( ELEMENT )}\\
gibt eine Parameterliste aus.

\item {\tt void  genc\_top( ELEMENT )}\\
gibt Code f"ur eine Expression oder Formula aus.

\item {\tt void  genc\_att\_seq( ELEMENT, int )}\\
gibt Code f"ur einen Attributzugriff oder Funktionsaufruf aus.

\item {\tt void  genc\_subces( ELEMENT, int )}
\item {\tt genc\_intres( ELEMENT, int )}\\
geben Code f"ur die Berechnung eines Zwischenergebnisses aus.
\end{itemize}

%%%%%%%%%%%%%%%%%%%%%%%%%%%%%%%%%%%%%%%%%%%%%%%%%%%%%%%%%%%%%%%%%%%%%%%%

\subsection{Codegenerierung f"ur die Conditions}

\begin{itemize}
\item {\tt void genc\_cnd\_itl( ELEMENT, ELEMENT, char*, int )}\\
\item {\tt void genc\_cnd\_pat( ELEMENT, ELEMENT, char*, int )}
\end{itemize}

\subsection{Generierung der lokalen Variablen}

\begin{itemize}
\item {\tt void genc\_cnd\_aux( ELEMENT )}
\item {\tt void genc\_att\_aux( ELEMENT )}
\item {\tt void aux\_prt\_lokvar( ELEMENT, int )}
\item {\tt int  genc\_fct\_aux( ELEMENT, int )}\\
geben alle in den R"umpfen ben"otigten lokalen Variablen aus.

\item {\tt int  genc\_att\_aux\_case( ELEMENT tn, int count )}\\
erzeugt die Verzweigung f"ur das Fortsetzen der partiellen Auswertung.

\item {\tt void  prt\_propagation( char *, int, int ) }\\
erzeugt Code f"ur das Abarbeiten der Warteliste.
\end{itemize}

\subsection{Ausgabe der globalen Deklarationen und R"umpfe}

Funktion : {\tt void  prt\_globdecls( int maxwaitpar )}

Dabei wird folgendes erzeugt :
\begin{itemize}
\item Deklarationen f"ur die externen Funktionen, Attribute und Pr"adikate
\item Deklarationen f"ur die (in MAX definierten) Funktionen und Attribute
\item Zugriffsfunktionen f"ur die Attribute

\item R"umpfe der Funktionen\\
Die Parameter werden noch um ({\tt EVALFCT mxw\_evf,ELEMENT mxw\_par})
erg"anzt, falls Funktionen und Attribute verschr"ankt sind und
eine Warteliste notwendig ist.

\item R"umpfe der Pr"adikate

\item R"umpfe der Attributauswertungsfunktionen\\

\begin{itemize}
\item Ohne Warteliste :\\
die Auswertungsfunktion kann einfach aufgerufen werden.
\item Mit Warteliste, ohne partielle Auswertung :\\
die Auswertungsfunktion kann einfach aufgerufen werden, am Ende wird
jedoch {\tt prt\_propagation} aufgerufen.
\item
Mit Warteliste und partieller Auswertung:\\
Der bisherige Auswertungszustand mu"s wiederhergestellt werden, dann wird die
Auswertung fortgesetzt. Am Ende wird {\tt prt\_propagation} aufgerufen.
\end{itemize}

\item Auswertungsprozeduren der Conditions
\end{itemize}


\subsection{Sonstige Funktionen}

\begin{itemize}
\item {\tt void reserv\_attvalue( ELEMENT, int )}\\
Hilfsfunktion, die den Speicher f"ur die Attributwerte und Wartelisten
"ubernimmt.

\item {\tt ELEMENT Defidnode\_insert( ELEMENT, ELEMENT )} und\\
\item {\tt ELEMENT Defidnodelist\_union( ELEMENT, ELEMENT )}\\
Hilfsfunktionen f"ur die Verwaltung von Mengen von {\tt DefId@}.

\item {\tt void prt\_evalset( ELEMENT, ELEMENT, int )}\\
erzeugt die partielle Auswertung einer Feingruppe (nur am selben Knoten).

\item {\tt long mxz\_elem( ELEMENT, ELEMENT )}\\
Teilmengenfunktion f"ur die Sorte {\tt DefIdNodeInfo}.

\item {\tt void prt\_init\_heap( ELEMENT, int, int, int, char *, char * )} und\\
\item {\tt void prt\_choose\_heap( ELEMENT, ELEMENT, int, int, int, char *, char *, int )}\\
k"ummern sich um die Aufz"ahlung mehrerer Grundsorten mit dem Heap.


\item {\tt void prt\_finegroupeval( ELEMENT, ELEMENT, int )}\\
Hilfsfunktion f"ur {\tt prt\_choose\_heap}, die die Auswertung einer Feingruppe
"ubernimmt.

\item {\tt void prt\_groupeval( ELEMENT, int )}:\\
erzeugt eine Auswertungsstrategie, bei der mehrere Grundsorten zusammengemischt
werden m"ussen (ruft {\tt prt\_init\_heap} und {\tt prt\_choose\_heap} auf).

\end{itemize}

\subsection{Steuerung der gesamten Attributauswertung}

Die Funktion {\tt void prt\_strucconstr( ELEMENT )}
steuert die gesamte Attributauswertung und den Test aller Conditions.

\subsection{Hilfsfunktionen}

\begin{itemize}
\item {\tt long  init\_endofpredef( void )}

\item {\tt ELEMENT intlist\_insert( ELEMENT, int )}
\end{itemize}

\subsection{Aufruf der gesamten Codegenerierung}

Die Funktion {\tt void  source\_gen( FILE *, char * )}
steuert die gesamte Codegenerierung.
Zus"atzlich werden folgende Gr"o"sen berechnet :

\begin{itemize}
\item Berechnung von {\tt maxwaitpar}, das die maximale Anzahl der Zwischenergebnisse
der Auswertungsfunktionen angibt (bei einer Unterbrechung der partiellen Ausw.).

\item Berechnung der Heapgr"o"se f"ur die Aufz"ahlung der Klassensorten
\item Berechnung der maximal notwendigen Stackgr"o"se f"ur den Kellerautomat, der
die Knoten f"ur den R"uckw"artsdurchlauf indiziert.
\end{itemize}



\end{document}
