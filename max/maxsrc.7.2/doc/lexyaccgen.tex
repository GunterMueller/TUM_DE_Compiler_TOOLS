\documentstyle[german,12pt,a4]{article}
\begin{document}

{\LARGE\bf\centerline{Dokumentation der MAX-Optionen}}
\bigskip\noindent
In Abschnitt~1 werden die m"oglichen Optionen der MAX-Version 0.6 beschrieben.
Der Rest dieser Dokumentation (Abschnitte 2 bis 5)
behandelt die Generierung von Scanner, Parser, Rahmenprogramm und Makefile.
Damit ist es m"oglich, nur aus der MAX-Spezifikation ({\tt LANG\_spec.m})
sowie der Bibliotheksdatei {\tt max\_std.o}
eine lauff"ahige MAX-Anwendung zu generieren.

\bigskip\noindent (Roland Merk, 17.12.93)

\vspace{2cm}
\tableofcontents
\newpage
\section{Programmaufruf von MAX}

In diesem Abschnitt werden kurz die Optionen beschrieben, die beim Aufruf des
MAX-Compilers ({\tt max}) erlaubt sind.

\begin{itemize}
\item Zu bearbeitende MAX-Spezifikation (Dateiname "`{\tt <spec\_name>.m}"'):

Bei mehreren Dateinamen wird der letzte ausgew"ahlt.
Vor oder nach dem Dateinamen k"onnen beliebig viele Optionen angegeben
werden.

\item Generierung eines Makefile (Option {\tt -m}):

Ein Makefile mit dem Namen "`{\tt Makefile}"' wird erzeugt, das
eine MAX-Anwendung generieren kann.

\item Generierung eines Rahmenprogramms (Option {\tt -f}):

Es wird ein Rahmenprogramm "`{\tt <name>.c}"' erzeugt, das
das Quellprogramm einliest, parst, die Attribute auswertet
und anschlie"send den abstrakten Syntaxbaum des Quellprogramms
in die Standardausgabe schreibt.

Dabei wird der {\tt <name>} aus der ersten \mbox{\tt STRUC <name> [...]\{\}}
in der MAX-Spezifikation genommen.

\item Generierung von Scanner und Parser f"ur die abstrakte Syntax
(Option {\tt -ya}):

In Abschnitt \ref{abssyn} wird beschrieben, wie die abstrakte
Syntax genau aussehen mu"s.

Der Scanner wird als LEX-Spezifikation in die Datei "`{\tt <spec\_name>.l}"'
geschrieben; der Parser als YACC-Spezifikation in die Datei
"`{\tt <spec\_name>.y}"'.

\item Generierung von Scanner und Parser f"ur die konkrete Syntax
(Option {\tt -yc}):

Zur Spezifikation der konkreten Syntax werden die in die MAX-Spezifikation
eingestreuten Symbole und Kommandos verwendet (Beschreibung in Abschnitt
\ref{konkrsyn}).

Es werden die gleichen Dateien wie beim letzten Punkt erzeugt.

\item Strategie der Attributauswertung (Option {\tt -O[123]}):

Bei der Attributauswertung k"onnen drei verschiedene Optimierungsstufen
gew"ahlt werden (der Defaultwert ist dabei {\tt O3}):

\begin{tabular}{ll}
Option & Vorgehensweise\\
\hline
{\tt -O1}: & Ohne Durchlaufstrategie ohne partielle Auswertung.\\
{\tt -O2}: & Ohne Durchlaufstrategie mit partieller Auswertung.\\
{\tt -O3} (Defaultwert): & Mit Durchlaufstrategie und partieller Auswertung.
\end{tabular}

\noindent
Dabei bedeutet "`mit Durchlaufstrategie"' einen Baumdurchlauf, bei
dem die Attributinstanzen gem"a"s einer topologischen Sortierung
ihres Abh"angigkeitsgraphen besucht werden.

\item Ausgabe der Ergebnisse der Abh"angigkeitsanalyse (Option {\tt -p}):

Das funktionale Verhalten aller (in MAX definierten) Funktionen und Attribute
{\tt (f : Node -> Node)} wird als regul"arer Ausdruck "uber den
elementaren Baumlauffunktionen {\tt fath}, {\tt son}, {\tt rbroth}
und {\tt lbroth} ausgegeben. Falls nach der Analyse keine Aussage m"oglich
ist, wird der Wert {\tt nil()} ausgegeben.

\item Ausgabe des abstrakten Syntaxbaums (Option {\tt -t}).

\item Aufruf des Baumbrowsers zur Visualisierung (Option {\tt -b}).
\end{itemize}

%%%%%%%%%%%%%%%%%%%%%%%%%%%%%%%%%%%%%%%%%%%%%%%%%%%%%%%%%%%%%%%%

\section{Quellprogrammeingabe in der abstrakten Syntax}
\label{abssyn}

\subsection{Terminale}

In MAX sind nur Elemente der f"unf vordefinierten Typen {\tt Int},
{\tt Ident}, {\tt Bool}, {\tt Char} und {\tt String}
als Terminale der abstrakten Syntax zul"assig.

\begin{table}
\begin{center}
\caption{Terminale der abstrakten Syntax} \medskip
\begin{tabular}{|l|ll|}
\hline
Terminalsorte & Syntax & Beispiel \\
\hline
Int    &{\tt [-][0-9]* }& {\tt 123 } \\
Ident  &{\tt [a-zA-Z][a-zA-Z0-9\_]* }& {\tt TypeSpec } \\
Bool   &{\tt true} oder {\tt false }& {\tt true}\\
Char   &{\tt '[a-zA-Z...]' }& {\tt 'k' } \\
String &{\tt \verb@"@[a-zA-Z...]*\verb@"@ }& {\tt \verb@"@Fehler !\verb@"@ } \\
\hline
\end{tabular}
\end{center}
\end{table}

\subsubsection*{Anmerkung :}

Kann {\tt Bool} im Syntaxbaum vorkommen, so werden {\tt true}
und {\tt false} vom generierten Scanner stets als {\tt Bool}-Symbole
interpretiert. Sonst werden sie als {\tt Ident} betrachtet.


\subsection{Nichtterminale}

Alle Sorten, die in der MAX-Spezifikation als Tupel- oder Listenproduktionen
definiert wurden, k"onnen in der folgender Form angegeben werden :

\medskip\noindent
{\bf Schema :} \qquad {\tt Sortenname( Subterme, durch ',' getrennt )}

\medskip\noindent
Die Sorten der Subterme m"ussen mit denen in der Definition der Produktionen
"ubereinstimmen. Dies wird beim Termaufbau des Syntaxbaums "uberpr"uft.
Bei Listenproduktionen kann eine leere Liste in der Form
{\tt Sortenname()} angegeben werden.

\medskip\noindent {\bf Beispiel :}

\medskip\noindent Tupelproduktion in der MAX-Spezifikation :

{\tt Function ( Ident ParList Type )}

\medskip\noindent Abstrakte Syntax im Quellprogramm :

{\tt Function( zeichnen, ParList(...), Type(...) )}

\subsubsection*{Anmerkung :}

Ein Sortenname (im Beispiel "`{\tt Function}"') kann entweder als
Termaufbaufunktion oder als einfacher Bezeichner ({\tt Ident})
angesehen werden. Der generierte Scanner erkennt diesen Unterschied am
Lookahead. Falls eine "offnende Klammer folgt, nimmt er eine
Termaufbaufunktion an.

\subsubsection*{Sorte des gesamten Quellprogramms}

Der Syntaxbaum des gesamten Quellprogramms mu"s dieselbe Sorte
haben, die in der ersten {\tt STRUC} der MAX-Spezifikation
in eckigen Klammern angegeben wird.



%%%%%%%%%%%%%%%%%%%%%%%%%%%%%%%%%%%%%%%%%%%%%%%%%%%%%%%%%%%%%%%%

\section{Quellprogrammeingabe in der konkreten Syntax}
\label{konkrsyn}

Bei der konkreten Syntax werden die Sorten der Teilb"aume nicht im
Quellprogramm eingegeben. Stattdessen wird die Struktur des Quellprogramms
durch eingestreute Token festgelegt.

Die Beschreibung der abstrakten Syntax wird durch Token und diverse
Kontrollbefehle (mit \# beginnend) realisiert, deren Bedeutung
im folgenden beschrieben wird.


\subsection{Zugeordnete abstrakte Syntax einer MAX-Spezifikation}

Die abstrakte Semantik der MAX-Spezifikation erh"alt man, indem
alle Token und Kontrollbefehle ignoriert werden. Sie ist nur durch die
spezifizierten Sorten definiert.

\medskip\noindent{\bf Ausnahme :}

\noindent
Die Befehle {\tt \#line} und {\tt \#col} zur Angabe von Positionen im
Quellprogramm werden durch den vordefinierten MAX-Typ {\tt Int} ersetzt.



\section{Erweiterungen der MAX-Syntax}


Nun werden die Erweiterungen mit Beispielen vorgestellt. Dabei wird
jeweils die zugeordnete konkrete Syntax durch Produktionen einer
kontextfreien Grammatik definiert. Bei der Generierung des Parsers
werden die Grammatikproduktionen in eine entsprechende
yacc-Spezifikation umgesetzt.


\subsection{Listenproduktionen}

Motivation :
H"aufig werden z.B.\ Listen von Bezeichnern in verschiedenem Kontext
in einer Sprache verwendet. Daher soll es m"oglich sein, f"ur
Listenproduktionen verschiedene Alternativen der konkreten Syntax
anzugeben.

F"ur jede Alternative soll dabei angegeben werden,
welches Trennzeichen zwischen je zwei Objekten der Untersorte steht und
ob die Liste leer sein darf.

\begin{table}
\begin{center}
\caption{Spezifikationsm"oglichkeiten der konkreten Syntax}
\medskip
\begin{tabular}{ll}
in der MAX-Spez. & Bedeutung \\
\hline
{\tt Alist * Anont *\verb@"@Sep1\verb@"@}& leere Liste {\tt Alist()} erlaubt\\
{\tt Blist * Bnont +\verb@"@Sep2\verb@"@}& leere Liste {\tt Blist()} nicht erlaubt\\
\hline
\end{tabular}
\end{center}
\end{table}

Steht keine Angabe "uber eine konkrete Syntax, so wird als Defaultwert
{\tt *\verb@"@\verb@"@} eingesetzt (leere Liste erlaubt, kein Trennsymbol).
Mehrere Alternativen k"onnen einfach hintereinandergeschrieben werden.

\subsubsection*{Beispiel :}

{\tt Clist * Cnont *\verb@"@Sep1\verb@"@ +\verb@"@Sep2\verb@"@}

\subsubsection*{Referenz auf eine Alternative der Listenproduktion}

Mit dem einfachen Sortennamen wird stets die erste Alternative angesprochen.
Die weiteren Alternativen erreicht man durch Angabe ihrer Nummer, die sich
auf die Reihenfolge der Aufschreibung bezieht.

\begin{table}
\begin{center}
\caption{Beispiele f"ur Referenzen auf Listenproduktionen}\medskip
\begin{tabular}{ll}
\hline
{\tt Clist}    : & Bezug auf {\tt Clist * Cnont *\verb@"@Sep1\verb@"@} \\
{\tt Clist\#1} : & Bezug auf {\tt Clist * Cnont *\verb@"@Sep1\verb@"@} \\
{\tt Clist\#2} : & Bezug auf {\tt Clist * Cnont +\verb@"@Sep2\verb@"@} \\
\hline
\end{tabular}
\end{center}
\end{table}

\noindent Eine Nummernangabe darf bis jetzt nur auf der
rechten Seite einer Tupelproduktion stehen.

\medskip\noindent{\bf
Zugeordnete konkrete Syntax von {\tt Alist * Anont *\verb@"@Sep1\verb@"@}
}

Ein Hilfsnichtterminal {\tt Aseq} wird dabei eingef"uhrt.

{\tt
\medskip
\begin{tabular}{lll}
Alist & ::= & $\varepsilon$ \\
      & |   & Aseq \\
Aseq  & ::= & Aseq \verb@"@Sep1\verb@"@ Anont \\
      & |   & Anont
\end{tabular}
}

\medskip\noindent{\bf
Zugeordnete konkrete Syntax von {\tt Blist * Bnont +\verb@"@Sep2\verb@"@}
}


{\tt
\medskip
\begin{tabular}{lll}
Blist & ::= & Bseq \\
Bseq  & ::= & Bseq \verb@"@Sep2\verb@"@ Bnont \\
      & |   & Bnont
\end{tabular}
}

\medskip\noindent
Anmerkung : Namensgleichheiten, falls dieselben Sorten
mehrmals auf der rechten Seite verschiedener Listenproduktionen auftreten,
werden bei der Generierung des Parsers vermieden. Ebenso werden die Nummern
der Alternativen korrekt behandelt.


%%%%%%%%%%%%%%%%%%%%%%%%%%%%%%

\subsection{Klassenproduktionen}

Bei Klassenproduktionen d"urfen auf der rechten Seite
zus"atzlich vor und nach jedem Sortenbezeichner beliebig viele Token
angegeben werden, die direkt in die konkrete Syntax "ubernommen werden.
Dabei darf derselbe Sortenbezeichner auch mehrmals mit verschiedenen
Token angegeben werden.

\medskip\noindent{\bf MAX-Spezifikation :}

{\tt Stat = Block | While \verb@"@;\verb@"@| For \verb@"@;\verb@"@}

\medskip\noindent {\bf Zugeh"orige konkrete Syntax :}
\begin{verbatim}
Stat ::= Block
       | While ";"
       | For   ";"
\end{verbatim}

%%%%%%%%%%%%%%%%%%%%%%%%%%%%%%

\subsection{Tupelproduktionen}
%rechts String #1 #line #col #mark

Innerhalb der rechten Seite einer Tupelproduktion ist es m"oglich,
Positionen im Quelltext (Zeile und Spalte) von Terminalen des
konkreten Syntaxbaums zu verwenden.

\subsubsection*{Elemente auf der rechten Seite einer Tupelproduktion}

\begin{itemize}
\item Token.\\
Das in Anf"uhrungszeichen angegebene Token wird in die konkrete Syntax
eingebaut. Es erscheint nicht im abstrakten Syntaxbaum.

\item Markierung eines Terminals.\\
Soll bei einem Terminal die Position in Quelltext (Zeile und Spalte)
bekannt sein, mu"s dieses Terminal mit {\tt \#mark} markiert werden.
Der generierte Scanner merkt sich dann die Position.
Als Terminale k"onnen auch Token verwendet werden, obwohl diese nicht mehr
in den abstrakten Syntaxbaum aufgenommen werden.

\item Positionen im Quelltext\\
Bei {\tt \#line} bzw.\ {\tt \#col} wird die Zeile bzw.\ Spalte
des zuletzt markierten Terminals in den abstrakten Syntaxbaum eingesetzt.
Dabei wird der vordefinierte Typ {\tt Int} verwendet.
\end{itemize}

\noindent{\bf Beispiel :}
\begin{verbatim}
MAX-Spezifikation: NodeSortId  ( Ident#mark '@' #line #col )
abstrakte Syntax : NodeSortId  ( Ident Int Int )

konkrete Syntax :  Bezeichner mit Suffix '@'
\end{verbatim}



\subsection{Benutzerdefinierte yacc-Regeln}
%% #yacc #yaccadd 

Es gibt zwei M"oglichkeiten, um benutzerdefinierte Regeln in das yacc-file
einzubinden :

\begin{itemize}
\item Als Ersatz f"ur die standardm"a"sig erzeugte Regel
(Kommando {\tt \#yacc}).

\item Als zus"atzliche Regel, um eine alternative Syntax
anzugeben (Kommando {\tt \#yaccadd}).
\end{itemize}

\noindent
Beide Kommandos werden direkt nach dem neu definierten Sortenbezeichner
angegeben. Auf {\tt \#yacc} und {\tt \#yaccadd} mu"s zwischen {\tt `'}
eine Folge von Strings stehen, die in die yacc-Spezifikation kopiert wird.
Token m"ussen in diesen Strings in einfachen Hochkommas eingegeben werden.

\noindent{\bf Beispiel :}

\medskip\noindent
Bei Funktionsaufrufen mit nur einem Parameter soll f"ur "`{\tt f(N)}"' als
alternative Schreibweise "`{\tt N.f}"' erlaubt sein.

\begin{verbatim}
FuncAppl #yaccadd `
  "| ParamNt '.' SortIdNt "
  "  { $$ = FuncAppl( $3, appback(Parlist(),$1) ); } "
'
           ( SortId "(" ParList ")" )  // abstrakte Syntax
\end{verbatim}

\noindent{\bf Erzeugte Yacc-Regel :}

\begin{verbatim}
FuncApplNt :
   SortIdNt '(' ParListNt ')'
   { $$ = FuncAppl( $1, $3 ); }
|  ParamNt '.' SortIdNt
   { $$ = FuncAppl( $3, appback(Parlist(),$1) ); }
;
\end{verbatim}

\noindent Bemerkung :
Alle Sortenbezeichner (auch die vordefinierten
Terminalen) bekommen in der yacc-Spezifikation das Suffix "`Nt"'.


\subsection{M"ogliche Sorten im konkreten Syntaxbaum}
%% #ignore #aux

Normalerweise wird durch eine Analyse der Produktionen automatisch erkannt,
welche Sorten im Syntaxbaum auftreten d"urfen. Bei Verwendung von
benutzerdefinierten Regeln ({\tt \#yacc} bzw. {\tt \#yaccadd}) k"onnen sich
Abweichungen ergeben.

\begin{itemize}
\item
Steht {\tt \#ignore} nach einem Produktionsbezeichner, so wird
f"ur die Produktion (und von ihr abh"angige) keine Parserregel erzeugt.
\item
Steht {\tt \#aux} nach einem Produktionsbezeichner, so wird eine
entsprechende Parserregel erzeugt.
\end{itemize}

\noindent{\bf Beispiel :}

\medskip\noindent
In der konkreten Syntax soll eine formale Parameterliste einer Prozedur
als {\tt ( SortId Name , SortId Name , ...)} angegeben werden.

In der abstrakten Syntax jedoch sollen die Typen und Namen auseinandersortiert
werden, d.h.\ es soll eine {\tt SortIdList} und eine {\tt NameList} 
im abstrakten Syntaxbaum stehen.

\noindent M"ogliche Spezifikation :

\begin{verbatim}
Function #yacc`
" 'FCT' DefIdNt '(' ParTupListNt ')' SortIdNt ':' ExprNt"
" { $$ = Function( $2, subt1($4), subt2($4), $6 ,$8 );}"
'  ( DefId  SortIdList  NameList  SortId  Expr  )

SortIdList #ignore * SortId
NameList   #ignore * Name

ParTupList #aux    *   ParTup  *","
ParTup             (   SortId  Name  )
\end{verbatim}

\noindent
Dabei sind vom Benutzer zwei Hilfsfunktionen\\
{\tt subt1: ParTupList -> SortIdList} \enspace und\\
{\tt subt2: ParTupList -> NameList} \quad zu schreiben, die die ersten
bzw.\ zweiten Subterme der {\tt ParTupList} aufsammeln.



%%%%%%%%%%%%%%%%%%%%%%%%%%%%%%%%%%%%%%%%%%

\subsection{Priorit"aten bei arithmetischen Ausdr"ucken}

In Sprachen kommen h"aufig arithmetische Ausdr"ucke vor. Daher soll auch
hier ein unterst"utzendes Sprachkonstrukt bereitgestellt werden.
Anhand eines kleinen Beispiels sollen Syntax und Semantik des
Konstruktes erl"autert werden.

\begin{verbatim}
Expr #prio = Sum | Prod | Fact | "(" Expr ")"

Sum        = Add | Sub
Add  #left ( Expr "+" Expr )
Sub  #left ( Expr "-" Expr )

Prod       = Mult | Div
Mult #left ( Expr "*" Expr )
Div  #left ( Expr "/" Expr )

Factor     = Neg | Ident | Int
Neg  #left ( "-" Expr )
\end{verbatim}

\noindent {\bf Erzeugte Grammatik :}

\begin{verbatim}
Expr ::= Sum
Sum  ::= Add | Sub | Prod
Add  ::= Sum "+" Prod
Sub  ::= Sum "-" Prod
Prod ::= Mult | Div | Fact
Mult ::= Prod "*" Fact
Div  ::= Prod "/" Fact

Fact ::= Neg | Ident | Int | "(" Expression ")"
Neg  ::= "-" Fact
\end{verbatim}


\noindent{\bf Erl"auterungen :}

\begin{itemize}
\item
Die Reihenfolge der Sorten bei der mit {\tt \#prio} versehenen Klassenproduktion
gibt die Priorit"atsreihenfolge an, beginnend bei den niedrigen Priorit"aten.
Als letzte Alternative mu"s die eingeklammerte Verwendung der linken Seite stehen.

\item
Mit {\tt \#left} bzw.\ {\tt \#right} kann man bei einer Sorte, die Untersorte einer
Priorit"atsstufensorte ist, angeben, ob sie linksrekursiv oder rechtsrekursiv verwendet
werden soll. Im Beispiel sind alle Untersorten linksrekursiv.
Das bedeutet, da"s beim ersten Auftreten von {\tt Expr} von links bzw.\ rechts
dieselbe Priorit"atsstufe steht und beim zweiten Auftreten die n"achsth"ohere.

In der abstrakten Syntax bleibt jeweils {\tt Expr} stehen.

\item Kommt {\tt Expr} nur einmal auf der rechten Seite einer Tupelproduktion
vor, so werden {\tt \#left} bzw.\ {\tt \#right} ignoriert.

\item
F"ur die {\tt \#prio}-Klassenproduktion wird eine Kette von Produktionen\\
{\tt Expr ::= Sum}\\ {\tt Sum ::= Prod}\\ {\tt Prod ::= Fact} \quad und \quad
{\tt Fact ::= '(' Expr ')'} \quad generiert.


\end{itemize}

\section{Einbau von vordefinierten Objekten in den Syntaxbaum}

Bei der Generierung des Parsers wird bei der Wurzelproduktion eine Funktion
{\tt add\_predeclarations} eingeh"angt, die der Benutzer verwenden kann, um
vordefinierte Objekte (z.B.\ Definitionen von Sorten oder Funktionen)
in den Syntaxbaum einzuh"angen.

Im mit der Option {\tt -f} erzeugten Rahmenprogramm wird diese Funktion
erzeugt. Sie reicht jedoch nur den Syntaxbaum unver"andert weiter.

\medskip\noindent {\bf Beispiel aus der yacc-Spezifikation :}

\begin{verbatim}
Specification :
	Program
	{ $$ = add_predeclarations( $1 ); }
;
\end{verbatim}


\end{document}
