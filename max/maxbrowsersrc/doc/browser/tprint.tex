\section{Funktionsbeschreibungen des Moduls {\tt tprint.c}}

\begin{description}

\item[\tt BackgroundMapProcedure]{\bf :\\}
BackgroundMapProcedure() ist eine Xt Hintergrundfunktion. Sie ist f"ur das Mappen des Viewports zust"andig. \\
IN:
\begin{itemize}
   \item XtPointer client\_data ist ein Pointer auf das zu mappende Widget.
\end{itemize}
OUT:
\begin{itemize}
   \item Boolean BackgroundMapProcedure() hat als R"uckgabewert immer TRUE, um Xt mitzuteilen, da"s die Funktion aus derBackground Procedure Liste wieder entfernt werden soll.
\end{itemize}

\item[\tt CreateXCONTEXT]{\bf :\\}
CreateXCONTEXT() stellt einen neuen X-Kontext bereit. Der Inhalt der Struktur wird so weit wie m"oglich belegt. \\
IN:
\begin{itemize}
   \item int argc ist der Argument Counter der Procedur main().
   \item char **argv ist der Argument Vektor der Procedur main().
\end{itemize}
OUT:
\begin{itemize}
   \item XCONTEXT *CreateNewXCONTEXT() liefert als Ergebnis einen Pointer auf den neu allocierten und initialisiertenKontext.
\end{itemize}

\item[\tt CopyXCONTEXT]{\bf :\\}
CopyXCONTEXT() kopiert einen X-Kontext, wobei auch der ben"otigte Speicherbereich reserviert wird. Der Inhalt der neuen Struktur wird so weit wie m"oglich belegt. \\
IN:
\begin{itemize}
   \item XCONTEXT *xcontext ist ein Pointer auf den zu kopierenden Kontext.
\end{itemize}
OUT:
\begin{itemize}
   \item XCONTEXT *CopyXCONTEXT() liefert als Ergebnis einen Pointer auf den neu allocierten und initialisiertenKontext.
\end{itemize}

\item[\tt DestroyXCONTEXT]{\bf :\\}
DestroyXCONTEXT() gibt den durch einen X-Kontext belegten Speicherbereich frei. \\
IN:
\begin{itemize}
   \item XCONTEXT *xcontext ist ein Pointer auf den zu zerst"orenden Kontext.
\end{itemize}
OUT:

\item[\tt UpdateMaxTree]{\bf :\\}
UpdateMaxTree() verbindet alle MAX Knoten eines Baumes mit einem vorgegebenen Kontext. \\
IN:
\begin{itemize}
   \item MAX\_NODE *max\_node ist ein Pointer auf die Wurzel des MAX-Baumes, der an den Kontext gebunden werden soll.
   \item Context *context ist ein Pointer auf den zu setzenden Kontext.
\end{itemize}
OUT:

\item[\tt CopyContext]{\bf :\\}
CopyContext() kopiert einen Kontext, wobei Speicher f"ur die Kopie allociert und der Inhalt der Struktur belegt wird. Der zu diesem Kontext geh"orende MAX-Baum mu"s mit angegeben werden. Der Kontext wird au"serdem in die Kontexthierarchie eingeh"angt. \\
IN:
\begin{itemize}
   \item Context *source ist ein Pointer auf die Quelle.
   \item MAX\_NODE *max\_node ist ein Pointer auf die Wurzel des MAX-Baumes, der an den Kontext gebunden werden soll.
\end{itemize}
OUT:
\begin{itemize}
   \item Context *CopyContext() gibt einen Pointer auf den neu allocierten und initialisierten Kontext zur"uck.
\end{itemize}

\item[\tt DestroyContext]{\bf :\\}
DestroyContext() zerst"ort einen Baum von Kontexten, einschlie"slich aller rechten Nachbarn. \\
IN:
\begin{itemize}
   \item Context *context ist ein Pointer die Wurzel des zu zerst"orenden Kontextbaumes.
\end{itemize}
OUT:

\item[\tt InsertNodeWidget]{\bf :\\}
InsertNodeWidget() f"ugt abh"angig von einem MAX-Node ein NodeWidget in ein TreeWidget ein. \\
IN:
\begin{itemize}
   \item Widget tree ist das TreeWidget, in welches der neue Knoten eingef"ugt werden soll.
   \item Widget parent ist das Parent Widget, unter welchem der neue Knoten im Baum eingeh"angt werden soll.
   \item MAX\_NODE *node ist ein Pointer auf einen MAX-Knoten, welcher f"ur den Inhalt des neuen NodeWidgets verantwortlich sein soll.
\end{itemize}
OUT:
\begin{itemize}
   \item Widget InsertNodeWidget() gibt das neu erzeugte Node Widget zur"uck.
\end{itemize}

\item[\tt MakeCleanTree]{\bf :\\}
MakeCleanTree() sorgt daf"ur, da"s der MAX-Baum in einem definierten Initialzustand "ubergeht. \\
IN:
\begin{itemize}
   \item MAX\_NODE *max\_tree ist ein Pointer auf die Wurzel des Baums.
\end{itemize}
OUT:

\item[\tt CreateTree]{\bf :\\}
CreateTree() ist eine rekursive Funktion zur Umwandlung eins MAX-Baumes in ein Tree Widget. \\
IN:
\begin{itemize}
   \item Context *context ist ein Pointer auf die Kontextstruktur des Baumes.
   \item MAX\_NODE *max\_tree ist ein Pointer auf die Wurzel des umzuwandelnden MAX-Baumes.
   \item Widget parent ist das Eltern Widget des zu erzeugenden Unterbaums (gegebenenfalls NULL).
\end{itemize}
OUT:

\item[\tt CreatePath]{\bf :\\}
CreatePath() "uberpr"uft, ob zu einem gegebenen MAX Knoten ein entsprechender Pfad im Widget Baum existiert und legt diesen gegebenenfalls an, d.h. zugefaltete Unterb"aume werden aufgeklappt. \\
IN:
\begin{itemize}
   \item MAX\_NODE *max\_node ist ein Pointer auf den zu bearbeitenden MAX Knoten.
\end{itemize}
OUT:

\item[\tt CreateTreeViewport]{\bf :\\}
CreateTreeViewport() generiert zu einem gegebenen Kontext einen Viewport mit dem entsprechenden Widget Baum. \\
IN:
\begin{itemize}
   \item Context *context ist ein Pointer auf den Kontext.
\end{itemize}
OUT:

\item[\tt DestroySubtree]{\bf :\\}
DestroySubtree() l"oscht einen Widget-Baum. \\
IN:
\begin{itemize}
   \item MAX\_NODE *max\_node ist ein Pointer auf den Wurzelknoten im zugeh"origen MAX-Baum.
\end{itemize}
OUT:

\item[\tt CreateSubtree]{\bf :\\}
CreateSubtree() generiert den zu einem MAX Knoten geh"orenden Unterbaum als Widget Unterbaum - es wird {\bf nur} der Unterbaum generiert. \\
IN:
\begin{itemize}
   \item MAX\_NODE *max\_node ist ein Pointer auf den MAX Knoten, dessen Unterbaum dargestellt werden soll.
\end{itemize}
OUT:

\item[\tt FindNextNodePreorderSubtree]{\bf :\\}
FindNextNodePreorderSubtree() sucht zu einem vorgegebenen Namen rekursiv nach dem n"achsten Knoten in Preorder. Es wird nur im Unterbaum gesucht. \\
IN:
\begin{itemize}
   \item MAX\_NODE *max\_node ist ein Pointer auf den ersten Knoten des Suchdurchlaufes.
   \item char *name ist ein Pointer auf eine Zeichenkette, die den Namen des gesuchten Knoten enth"alt.
\end{itemize}
OUT:
\begin{itemize}
   \item MAX\_NODE *FindNextNodePreorderSubtree() liefert den Pointer auf einen eventuell gefundenen Knoten. Bei erfolgloser Suche wird NULL zur"uckgegeben.
\end{itemize}

\item[\tt FindNextNodePreorder]{\bf :\\}
FindNextNodePreorder() sucht zu einem vorgegebenen Namen nach dem n"achsten Knoten in Preorder. Es werden auch die entsprechenden "ubergeordneten Knoten durchsucht. \\
IN:
\begin{itemize}
   \item MAX\_NODE *max\_node ist ein Pointer auf den ersten Knoten des Suchdurchlaufes.
   \item char *name ist ein Pointer auf eine Zeichenkette, die den Namen des gesuchten Knoten enth"alt.
\end{itemize}
OUT:
\begin{itemize}
   \item MAX\_NODE *FindNextNodePreorder() liefert den Pointer auf einen eventuell gefundenen Knoten. Bei erfolgloserSuche wird NULL zur"uckgegeben.
\end{itemize}

\item[\tt FindPreviousNodePreorderSubtree]{\bf :\\}
FindPreviousNodePreorderSubtree() sucht zu einem vorgegebenen Namen rekursiv nach dem n"achsten Knoten in Preorder. Es wird nur im Unterbaum gesucht. \\
IN:
\begin{itemize}
   \item MAX\_NODE *max\_node ist ein Pointer auf den ersten Knoten des Suchdurchlaufes.
   \item char *name ist ein Pointer auf eine Zeichenkette, die den Namen des gesuchten Knoten enth"alt.
\end{itemize}
OUT:
\begin{itemize}
   \item MAX\_NODE *FindPreviousNodePreorderSubtree() liefert den Pointerauf einen eventuell gefundenen Knoten. Bei erfolgloser Suche wird NULL zur"uckgegeben.
\end{itemize}

\item[\tt FindPreviousNodePreorder]{\bf :\\}
FindPreviousNodePreorder() sucht zu einem vorgegebenen Namen nach dem vorhergehen den Knoten in Preorder. Es werden auch Vorg"angerknoten ber"ucksichtigt. \\
IN:
\begin{itemize}
   \item MAX\_NODE *max\_node ist ein Pointer auf den ersten Knoten des Suchdurchlaufes.
   \item char *name ist ein Pointer auf eine Zeichenkette, die den Namen des gesuchten Knoten enth"alt.
\end{itemize}
OUT:
\begin{itemize}
   \item MAX\_NODE *FindPreviousNodePreorder() liefert den Pointer auf einen eventuell gefundenen Knoten. Bei erfolgloserSuche wird NULL zur"uckgegeben.
\end{itemize}

\item[\tt FindNextNodePostorderSubtree]{\bf :\\}
FindNextNodePostorderSubtree() sucht zu einem vorgegebenen Namen rekursiv nach dem n"achsten Knoten in Postorder. Es wird nur im Unterbaum gesucht. \\
IN:
\begin{itemize}
   \item MAX\_NODE *max\_node ist ein Pointer auf den ersten Knoten des Suchdurchlaufes.
   \item char *name ist ein Pointer auf eine Zeichenkette, die den Namen des gesuchten Knoten enth"alt.
\end{itemize}
OUT:
\begin{itemize}
   \item MAX\_NODE *FindNextNodePostorderSubtree() liefert den Pointer auf einen eventuell gefundenen Knoten. Bei erfolgloser Suche wird NULL zur"uckgegeben.
\end{itemize}

\item[\tt FindNextNodePostorder]{\bf :\\}
FindNextNodePostorder() sucht zu einem vorgegebenen Namen nach dem n"achsten Knoten in Postorder. Es werden auch Vorg"angerknoten ber"ucksichtigt. \\
IN:
\begin{itemize}
   \item MAX\_NODE *max\_node ist ein Pointer auf den ersten Knoten des Suchdurchlaufes.
   \item char *name ist ein Pointer auf eine Zeichenkette, die den Namen des gesuchten Knoten enth"alt.
\end{itemize}
OUT:
\begin{itemize}
   \item MAX\_NODE *FindNextNodePostorder() liefert den Pointer auf einen eventuell gefundenen Knoten. Bei erfolgloser Suche wird NULL zur"uckgegeben.
\end{itemize}

\item[\tt FindPreviousNodePostorderSubtree]{\bf :\\}
FindPreviousNodePostorderSubtree() sucht zu einem vorgegebenen Namen rekursiv nach dem vorgehenden Knoten in Postorder. Es wird nur im Unterbaum gesucht. \\
IN:
\begin{itemize}
   \item MAX\_NODE *max\_node ist ein Pointer auf den ersten Knoten des Suchdurchlaufes.
   \item char *name ist ein Pointer auf eine Zeichenkette, die den Namen des gesuchten Knoten enth"alt.
\end{itemize}
OUT:
\begin{itemize}
   \item MAX\_NODE *FindPreviousNodePostorderSubtree() liefert den Pointer auf einen eventuell gefundenen Knoten.Bei erfolgloser Suche wird NULL zur"uckgegeben.
\end{itemize}

\item[\tt FindPreviousNodePostorder]{\bf :\\}
FindPreviousNodePostorder() sucht zu einem vorgegebenen Namen nach dem vorhergehenden Knoten in Postorder. Es werden auch "ubergeordnete Knoten ber"ucksichtigt. \\
IN:
\begin{itemize}
   \item MAX\_NODE *max\_node ist ein Pointer auf den ersten Knoten des Suchdurchlaufes.
   \item char *name ist ein Pointer auf eine Zeichenkette, die den Namen des gesuchten Knoten enth"alt.
\end{itemize}
OUT:
\begin{itemize}
   \item MAX\_NODE *FindPreviousNodePostorder() liefert den Pointer auf einen eventuell gefundenen Knoten. Bei erfolgloser Suche wird NULL zur"uckgegeben.
\end{itemize}

\item[\tt NodeCallback]{\bf :\\}
NodeCallback() ist als Callback-Procedure zu jedem Baumknoten eingeh"angt. Sie "ubernimmt die Reaktion auf die mittlere und die rechte Maustaste. \\
IN:
\begin{itemize}
   \item Widget node ist das angeklickte WidgetXtPointer client\_data ist ein Pointer auf den zugeh"origen Max-Knoten.
   \item XtPointer call\_data gibt den Mausbutton an.
\end{itemize}
OUT:

\item[\tt CreateX\_INFO]{\bf :\\}
CreateX\_INFO() generiert eine neue X\_INFO Struktur und belegt sie mit Defaultwerten. \\
IN: \\
OUT:
\begin{itemize}
   \item X\_INFO *CreateX\_INFO() liefert einen Pointer auf die neu bereitgestellte X\_INFO Struktur.
\end{itemize}

\item[\tt CopyX\_INFO]{\bf :\\}
CopyX\_INFO() kopiert den Inhalt einer X\_INFO Struktur und allokiert den ben"otigten Speicherbereich. \\
IN:
\begin{itemize}
   \item X\_INFO *xinfo ist ein Pointer auf die zu kopierende Struktur.
\end{itemize}
OUT:
\begin{itemize}
   \item X\_INFO *CopyX\_INFO() liefert einen Pointer auf die neu bereitgestellte X\_INFO Struktur.
\end{itemize}

\item[\tt DestroyX\_INFO]{\bf :\\}
DestroyX\_INFO() zerst"ort eine X\_INFO Struktur. \\
IN:
\begin{itemize}
   \item X\_INFO *xinfo ist ein Pointer auf die zu zerst"orende Struktur.
\end{itemize}
OUT:

\item[\tt GetActualAttrMask]{\bf :\\}
GetActualAttrMask() gibt die zu einem Kontext, der durch einen MAX Knoten angegeben wird, geh"orende Attributmaske zur"uck. \\
IN:
\begin{itemize}
   \item MAX\_NODE *max\_node ist ein Pointer auf einen MAX Knoten im gew"unschten Kontext.
\end{itemize}
OUT:
\begin{itemize}
   \item AttrMask *GetActualAttrMask() liefert einen Pointer auf die Attributmaske.
\end{itemize}

\item[\tt UnmanageViewport]{\bf :\\}
UnmanageViewport() sorgt daf"ur, da"s der Viewport nicht mehr gemanaged wird. \\
IN:
\begin{itemize}
   \item MAX\_NODE *max\_node ist ein Pointer auf einen MAX Knoten im Viewport.
\end{itemize}
OUT:

\item[\tt ManageViewport]{\bf :\\}
ManageViewport() sorgt daf"ur, da"s der Viewport wieder gemanaged wird. \\
IN:
\begin{itemize}
   \item MAX\_NODE *max\_node ist ein Pointer auf einen MAX Knoten im Viewport.
\end{itemize}
OUT:

\item[\tt UnmarkAllNodesFromRoot]{\bf :\\}
UnmarkAllNodesFromRoot() entfernt eventuell vorhandene Markierungen (ab Root des MAX\_NODE) von allen MAX Knoten. \\
IN:
\begin{itemize}
   \item MAX\_NODE *max\_node ist ein Pointer auf einen beliebigen MAX Knoten des Baumes.
\end{itemize}
OUT:

\item[\tt RaiseWindow]{\bf :\\}
RaiseWindow() legt das Window, das zu dem angegebenen Knoten geh"ort oben auf den Windowstack. \\
IN:
\begin{itemize}
   \item MAX\_NODE *max\_node ist ein Pointer auf einen beliebigen MAX Knoten innerhalb des Fensters.
\end{itemize}
OUT:

\item[\tt CenterNode]{\bf :\\}
CenterNode() zentriert einen MAX Knoten in seinem Fenster. \\
IN:
\begin{itemize}
   \item MAX\_NODE *max\_node ist ein Pointer auf den zu zentrierenden MAX Knoten.
\end{itemize}
OUT:

\item[\tt ShowAttributes]{\bf :\\}
ShowAttributes() bewirkt das Anzeigen der Attribute eines MAX Knotens. \\
IN:
\begin{itemize}
   \item MAX\_NODE *max\_node ist ein Pointer auf den zu bearbeitenden MAX Knoten.
\end{itemize}
OUT:

\item[\tt HideAttributes]{\bf :\\}
HideAttributes() bewirkt das Verstecken der Attribute eines MAX Knotens. \\
IN:
\begin{itemize}
   \item MAX\_NODE *max\_node ist ein Pointer auf den zu bearbeitenden MAX Knoten.
\end{itemize}
OUT:

\item[\tt MarkNode]{\bf :\\}
MarkNode() bewirkt das Markieren eines MAX Knotens. \\
IN:
\begin{itemize}
   \item MAX\_NODE *max\_node ist ein Pointer auf den zu bearbeitenden MAX Knoten.
\end{itemize}
OUT:

\item[\tt MarkSubtree]{\bf :\\}
MarkSubtree() bewirkt das Markieren eines MAX Unterbaumes. \\
IN:
\begin{itemize}
   \item MAX\_NODE *max\_node ist ein Pointer auf die Wurzel des zu bearbeitenden MAX Unterbaums.
\end{itemize}
OUT:

\item[\tt MarkShowedSubtree]{\bf :\\}
MarkShowedSubtree() bewirkt das Markieren des sichtbaren Bereichs eines MAX Unterbaumes. \\
IN:
\begin{itemize}
   \item MAX\_NODE *max\_node ist ein Pointer auf die Wurzel des zu bearbeitenden MAX Unterbaum.
\end{itemize}
OUT:

\item[\tt UnmarkNode]{\bf :\\}
UnmarkNode() bewirkt das Entfernen der Markierung eines MAX Knotens. \\
IN:
\begin{itemize}
   \item MAX\_NODE *max\_node ist ein Pointer auf den zu bearbeitenden MAX Knoten.
\end{itemize}
OUT:

\item[\tt UnmarkSubtree]{\bf :\\}
UnmarkSubtree() bewirkt das Entfernen der Markierungen in einem MAX Unterbaum. \\
IN:
\begin{itemize}
   \item MAX\_NODE *max\_node ist ein Pointer auf die Wurzel des zu bearbeitenden MAX Unterbaums.
\end{itemize}
OUT:

\item[\tt UnmarkShowedSubtree]{\bf :\\}
UnmarkShowedSubtree() bewirkt das Entfernen der Markierungen im sichtbaren Bereich eines MAX Unterbaums. \\
IN:
\begin{itemize}
   \item MAX\_NODE *max\_node ist ein Pointer auf den zu bearbeitenden MAX Baum.
\end{itemize}
OUT:

\item[\tt ShowSubtree]{\bf :\\}
ShowSubtree() bewirkt das Auffalten des zu einem MAX Knoten geh"orenden Unterbaums. \\
IN:
\begin{itemize}
   \item MAX\_NODE *max\_node ist ein Pointer auf den zu bearbeitenden MAX Knoten.
\end{itemize}
OUT:

\item[\tt HideSubtree]{\bf :\\}
HideSubtree() bewirkt das Verschatten des zu einem MAX Knoten geh"orenden Unterbaums. \\
IN:
\begin{itemize}
   \item MAX\_NODE *max\_node ist ein Pointer auf den zu bearbeitenden MAX Knoten.
\end{itemize}
OUT:

\item[\tt MarkPath]{\bf :\\}
MarkPath() bewirkt das Markieren aller Knoten zwischen der Wurzel des MAX Baumes und dem angegebenen MAX Knoten. \\
IN:
\begin{itemize}
   \item MAX\_NODE *max\_node ist ein Pointer auf den zu bearbeitenden MAX Knoten.
\end{itemize}
OUT:

\item[\tt UnmarkPath]{\bf :\\}
UnmarkPath() bewirkt das L"oschen der Markierungen an allen MAX Knoten zwischen der Wurzel des MAX Baumes und dem angegebenen MAX Knoten. \\
IN:
\begin{itemize}
   \item MAX\_NODE *max\_node ist ein Pointer auf den zu bearbeitenden MAX Knoten.
\end{itemize}
OUT:

\item[\tt CopySubtree]{\bf :\\}
CopySubtree() kopiert den zu einem angegebenen MAX Knoten geh"orenden Unterbaum und Kontext, und zeigt diesen in einem neuen Fenster an. \\
IN:
\begin{itemize}
   \item MAX\_NODE *max\_node ist ein Pointer auf die Wurzel des zu kopierenden MAX Knoten.
\end{itemize}
OUT:

\item[\tt GotoFirstNodePreorder]{\bf :\\}
GotoFirstNodePreorder() bewirkt einen Sprung zum ersten MAX Knoten mit gleichem Namen in Preorder. Dieser Knoten wird in der Mitte des Fensters zentriert und markiert. \\
IN:
\begin{itemize}
   \item MAX\_NODE *max\_node ist ein Pointer auf den urspr"unglichen MAX Knoten.
\end{itemize}
OUT:

\item[\tt GotoLastNodePreorder]{\bf :\\}
GotoLastNodePreorder() bewirkt einen Sprung zum letzten MAX Knoten mit gleichem Namen in Preorder. Dieser Knoten wird in der Mitte des Fensters zentriert und markiert. \\
IN:
\begin{itemize}
   \item MAX\_NODE *max\_node ist ein Pointer auf den urspr"unglichen MAX Knoten.
\end{itemize}
OUT:

\item[\tt GotoNextNodePreorder]{\bf :\\}
GotoNextNodePreorder() bewirkt einen Sprung zum n"achsten MAX Knoten mit gleichem Namen in Preorder. Dieser Knoten wird in der Mitte des Fensters zentriert und markiert. \\
IN:
\begin{itemize}
   \item MAX\_NODE *max\_node ist ein Pointer auf den urspr"unglichen MAX Knoten.
\end{itemize}
OUT:

\item[\tt GotoPreviousNodePreorder]{\bf :\\}
GotoPreviousNodePreorder() bewirkt einen Sprung zum vorhergehenden MAX Knoten mit gleichem Namen in Preorder. Dieser Knoten wird in der Mitte des Fensters zentriert und markiert. \\
IN:
\begin{itemize}
   \item MAX\_NODE *max\_node ist ein Pointer auf den urspr"unglichen MAX Knoten.
\end{itemize}
OUT:

\item[\tt GotoFirstNodePostorder]{\bf :\\}
GotoFirstNodePostorder() bewirkt einen Sprung zum ersten MAX Knoten mit gleichem Namen in Postorder. Dieser Knoten wird in der Mitte des Fensters zentriert und markiert. \\
IN:
\begin{itemize}
   \item MAX\_NODE *max\_node ist ein Pointer auf den urspr"unglichen MAX Knoten.
\end{itemize}
OUT:

\item[\tt GotoLastNodePostorder]{\bf :\\}
GotoLastNodePostorder() bewirkt einen Sprung zum letzten MAX Knoten mit gleichem Namen in Postorder. Dieser Knoten wird in der Mitte des Fensters zentriert und markiert. \\
IN:
\begin{itemize}
   \item MAX\_NODE *max\_node ist ein Pointer auf den urspr"unglichen MAX Knoten.
\end{itemize}
OUT:

\item[\tt GotoNextNodePostorder]{\bf :\\}
GotoNextNodePostorder() bewirkt einen Sprung zum n"achsten MAX Knoten mit gleichem Namen in Postorder. Dieser Knoten wird in der Mitte des Fensters zentriert und markiert. \\
IN:
\begin{itemize}
   \item MAX\_NODE *max\_node ist ein Pointer auf den urspr"unglichen MAX Knoten.
\end{itemize}
OUT:

\item[\tt GotoPreviousNodePostorder]{\bf :\\}
GotoPreviousNodePostorder() bewirkt einen Sprung zum vorhergehenden MAX Knoten mit gleichem Namen in Postorder. Dieser Knoten wird in der Mitte des Fensters zentriert und markiert. \\
IN:
\begin{itemize}
   \item MAX\_NODE *max\_node ist ein Pointer auf den urspr"unglichen MAX Knoten.
\end{itemize}
OUT:

\item[\tt HideSubtreeMarkedNodes]{\bf :\\}
HideSubtreeMarkedNodes() bewirkt das Zusammenfalten aller an markierten MAX Knoten h"angenden Unterb"aume. Es werden auch Unterb"aume in Unterb"aumen zusammengefaltet. \\
IN:
\begin{itemize}
   \item MAX\_NODE *max\_node ist ein Pointer auf die Wurzel des zu bearbeitenden MAX Baum.
\end{itemize}
OUT:

\item[\tt ShowSubtreeMarkedNodes]{\bf :\\}
ShowSubtreeMarkedNodes() bewirkt das Anzeigen aller an markierten MAX Knoten h"angenden Unterb"aume. Die Funktion wirkt auch auf markierte aber eventuell nicht sichtbare Knoten. \\
IN:
\begin{itemize}
   \item MAX\_NODE *max\_node ist ein Pointer auf die Wurzel des zu bearbeitenden MAX Unterbaums.
\end{itemize}
OUT:

\item[\tt HideSubtreeNamedNodes]{\bf :\\}
HideSubtreeNamedNodes() bewirkt das Zusammenfalten aller an MAX Knoten mit bestimmten Namen h"angenden Unterb"aume. Es werden auch Unterb"aume in Unterb"aumen zusammengefaltet. \\
IN:
\begin{itemize}
   \item MAX\_NODE *max\_node ist ein Pointer auf die Wurzel des zu bearbeitenden MAX Unterbaums.
   \item char *name ist ein Pointer auf die Zeichenkette mit dem MAX Knotennamen.
\end{itemize}
OUT:

\item[\tt ShowSubtreeNamedNodes]{\bf :\\}
ShowSubtreeNamedNodes() bewirkt das Auffalten aller an MAX Knoten mit bestimmten Namen h"angenden Unterb"aume. Es werden auch Unterb"aume in Unterb"aumen aufgefaltet. \\
IN:
\begin{itemize}
   \item MAX\_NODE *max\_node ist ein Pointer auf die Wurzel des zu bearbeitenden MAX Knoten.
   \item char *name ist ein Pointer auf die Zeichenkette mit dem MAX Knotennamen.
\end{itemize}
OUT:

\item[\tt MarkAllNodes]{\bf :\\}
MarkAllNodes() markiert alle Knoten in einem MAX Unterbaum. \\
IN:
\begin{itemize}
   \item MAX\_NODE *max\_node ist ein Pointer auf die Wurzel des zu bearbeitenden MAX Unterbaumes.
\end{itemize}
OUT:

\item[\tt MarkShowedNodes]{\bf :\\}
MarkShowedNodes() markiert alle auf dem Bildschirm sichtbaren MAX Knoten. \\
IN:
\begin{itemize}
   \item MAX\_NODE *max\_node ist ein Pointer auf die Wurzel des zu bearbeitenden MAX Unterbaumes.
\end{itemize}
OUT:

\item[\tt MarkNamedNodes]{\bf :\\}
MarkNamedNodes() markiert alle mit einem bestimmten Namen benannten MAX Knoten. \\
IN:
\begin{itemize}
   \item MAX\_NODE *max\_node ist ein Pointer auf die Wurzel des zu bearbeitenden MAX Unterbaumes.
   \item char *name ist ein Pointer auf die Zeichenkette mit dem MAX Knotennamen.
\end{itemize}
OUT:

\item[\tt UnmarkAllNodes]{\bf :\\}
UnmarkAllNodes() entfernt eventuell vorhandene Markierungen von allen MAX Knoten. \\
IN:
\begin{itemize}
   \item MAX\_NODE *max\_node ist ein Pointer auf die Wurzel des zu bearbeitenden MAX Unterbaumes.
\end{itemize}
OUT:

\item[\tt UnmarkShowedNodes]{\bf :\\}
UnmarkShowedNodes() entfernt alle vorhandene Markierungen im sichtbaren Bereich des Baumes. von allen MAX Knoten. \\
IN:
\begin{itemize}
   \item MAX\_NODE *max\_node ist ein Pointer auf die Wurzel des zu bearbeitenden MAX Unterbaumes.
\end{itemize}
OUT:

\item[\tt UnmarkNamedNodes]{\bf :\\}
UnmarkNamedNodes() entfernt Markierungen an allen mit einem bestimmten Namen benannten MAX Knoten. \\
IN:
\begin{itemize}
   \item MAX\_NODE *max\_node ist ein Pointer auf die Wurzel des zu bearbeitenden MAX Unterbaumes.
   \item char *name ist ein Pointer auf die Zeichenkette mit dem MAX Knotennamen.
\end{itemize}
OUT:

\item[\tt HideAttributesNamedNodes]{\bf :\\}
HideAttributesNamedNodes() bewirkt das Verstecken der Attribute an allen mit einem bestimmten Namen bezeichneten MAX Knoten. \\
IN:
\begin{itemize}
   \item MAX\_NODE *max\_node ist ein Pointer auf die Wurzel des zu bearbeitenden MAX Unterbaumes.
   \item char *name ist ein Pointer auf die Zeichenkette mit dem MAX Knotennamen.
\end{itemize}
OUT:

\item[\tt HideAttributesMarkedNodes]{\bf :\\}
HideAttributesMarkedNodes() bewirkt das Verstecken der Attribute an allen markierten MAX Knoten. \\
IN:
\begin{itemize}
   \item MAX\_NODE *max\_node ist ein Pointer auf die Wurzel des zu bearbeitenden MAX Unterbaumes.
\end{itemize}
OUT:

\item[\tt HideAttributesAllNodes]{\bf :\\}
HideAttributesAllNodes() bewirkt das Verstecken aller Attribute. \\
IN:
\begin{itemize}
   \item MAX\_NODE *max\_node ist ein Pointer auf die Wurzel des zu bearbeitenden MAX Unterbaumes.
\end{itemize}
OUT:

\item[\tt ShowAttributesNamedNodes]{\bf :\\}
ShowAttributesNamedNodes() bewirkt das Anzeigen der Attribute an allen mit einem bestimmten Namen bezeichneten MAX Knoten. \\
IN:
\begin{itemize}
   \item MAX\_NODE *max\_node ist ein Pointer auf die Wurzel des zu bearbeitenden MAX Unterbaumes.
   \item char *name ist ein Pointer auf die Zeichenkette mit dem MAX Knotennamen.
\end{itemize}
OUT:

\item[\tt ShowAttributesMarkedNodes]{\bf :\\}
ShowAttributesMarkedNodes() bewirkt das Anzeigen der Attribute an allen markierten MAX Knoten. \\
IN:
\begin{itemize}
   \item MAX\_NODE *max\_node ist ein Pointer auf die Wurzel des zu bearbeitenden MAX Unterbaumes.
\end{itemize}
OUT:

\item[\tt ShowAttributesAllNodes]{\bf :\\}
ShowAttributesAllNodes() bewirkt das Anzeigen aller Attribute in einem MAX Baum. \\
IN:
\begin{itemize}
   \item MAX\_NODE *max\_node ist ein Pointer auf die Wurzel des zu bearbeitenden MAX Unterbaumes.
\end{itemize}
OUT:

\item[\tt GotoFirstNamedNodePreorder]{\bf :\\}
GotoFirstNamedNodePreorder() bewirkt einen Sprung zum ersten MAX Knoten mit bestimmtem Namen in Preorder. Dieser Knoten wird in der Mitte des Fensters zentriert und markiert. \\
IN:
\begin{itemize}
   \item MAX\_NODE *max\_node ist ein Pointer auf die Wurzel des zu bearbeitenden MAX Unterbaumes.
   \item char *name ist ein Pointer auf die Zeichenkette mit dem MAX Knotennamen.
\end{itemize}
OUT:

\item[\tt GotoLastNamedNodePreorder]{\bf :\\}
GotoLastNamedNodePreorder() bewirkt einen Sprung zum letzten MAX Knoten mit bestimmtem Namen in Preorder. Dieser Knoten wird in der Mitte des Fensters zentriert und markiert. \\
IN:
\begin{itemize}
   \item MAX\_NODE *max\_node ist ein Pointer auf die Wurzel des zu bearbeitenden MAX Unterbaumes.
   \item char *name ist ein Pointer auf die Zeichenkette mit dem MAX Knotennamen.
\end{itemize}
OUT:

\item[\tt GotoFirstNamedNodePostorder]{\bf :\\}
GotoFirstNamedNodePostorder() bewirkt einen Sprung zum ersten MAX Knoten mit bestimmtem Namen in Postorder. Dieser Knoten wird in der Mitte des Fensters zentriert und markiert. \\
IN:
\begin{itemize}
   \item MAX\_NODE *max\_node ist ein Pointer auf die Wurzel des zu bearbeitenden MAX Unterbaumes.
   \item char *name ist ein Pointer auf die Zeichenkette mit dem MAX Knotennamen.
\end{itemize}
OUT:

\item[\tt GotoLastNamedNodePostorder]{\bf :\\}
GotoLastNamedNodePostorder() bewirkt einen Sprung zum letzten MAX Knoten mit bestimmtem Namen in Postorder. Dieser Knoten wird in der Mitte des Fensters zentriert und markiert. \\
IN:
\begin{itemize}
   \item MAX\_NODE *max\_node ist ein Pointer auf die Wurzel des zu bearbeitenden MAX Unterbaumes.
   \item char *name ist ein Pointer auf die Zeichenkette mit dem MAX Knotennamen.
\end{itemize}
OUT:

\item[\tt CopyTree]{\bf :\\}
CopyTree() kopiert den zu einem angegebenen MAX Knoten geh"orenden Gesamtbaum und Kontext, und zeigt diesen in einem neuen Fenster an. \\
IN:
\begin{itemize}
   \item MAX\_NODE *max\_node ist ein Pointer auf die Wurzel des zu bearbeitenden MAX Unterbaumes.
\end{itemize}
OUT:

\item[\tt CloseWindow]{\bf :\\}
CloseWindow() schlie"st das zu einem MAX Knoten geh"orende Fenster. \\
IN:
\begin{itemize}
   \item MAX\_NODE *max\_node ist ein Pointer auf einen beliebigen MAX Knoten innerhalb des zu schlie"senden Fensters.
\end{itemize}
OUT:

\item[\tt CreateMainShell]{\bf :\\}
  \\
IN:
\begin{itemize}
   \item MAX\_NODE *max\_tree ist ein Pointer auf den darzustellenden MAX Baum.
   \item int argc ist der von main() bereitgestellte Argumentcounter.
   \item char **argv ist der von main() bereitgestellte Argumentvektor.
\end{itemize}
OUT:

\item[\tt CreateNewShell]{\bf :\\}
CreateNewShell "offnet eine ne"u Shell f"ur einen MAX-Tree. Dieser MAX-Tree mu"s keinen eigenen Kontext besitzen. \\
IN:
\begin{itemize}
   \item MAX\_NODE *max\_tree ist ein Pointer auf die Wurzel des darzustellenden MAX Baumes.
   \item MAX\_NODE *max\_node ist ein Pointer auf den MAX-Knoten, dessen Kontext kopiert werden soll.
\end{itemize}
OUT:
\end{description}
