
\section{Interne Strukturen des Moduls {\tt tprint.c}}

Im folgenden Abschnitt werden die zur Fenster- und Baumverwaltung ben"otigten
Strukturen beschrieben.

\subsection{Die Strukturen {\tt Context} und {\tt XCONTEXT}}

Die Struktur Context wird f"ur die Verwaltung der Daten eines Fensters
ben"otigt. Jedes Fenster besitzt seinen eingenen Kontext, der dynamisch
beim "Offnen des Fensters erzeugt wird. Alle Kontexte eines Programmes
sind hierarchisch angeordnet. Wenn ein Kontext zerst"ort wird, so 
m"ussen auch seine Kinder zerst"ort werden.

Die Struktur XCONTEXT ist eine Unterstruktur des Kontextes. Sie enth"alt
Informationen, die f"ur die Darstellung durch Xt ben"otigt werden.


\begin{verbatim}
typedef struct 
   {
   Display      *display;   
   Widget       toplevel;  
   XtAppContext app_context;
   int          argc;      
   char         **argv;   
   Widget       form;   
   Widget       global_menu; 
   Widget       viewport;   
   Widget       h_scrollbar; 
   Widget       v_scrollbar; 
   Widget       tree;   
   AttrMask     *attr_mask; 
   } XCONTEXT;

typedef struct 
   {
   void         *max_tree;   
   XCONTEXT     *xcontext;  
   int          incarnation;
   int          number; 
   struct _Context_ *children; 
   struct _Context_ *next; 
   struct _Context_ *prev;
   struct _Context_ *parent; 
   } Context;
\end{verbatim}

\begin{description}
\item[\tt display]
   ist ein Verweis auf das von der Applikation verwendeten Display.
   Er wird mittels {\tt XtOpenDisplayi()} bei der Erzeugung der ersten
   {\tt XCONTEXT} Struktur ermittelt.
\item[\tt toplevel]
   ist das Widget der Shell des entprechenden Fensters.
   Es wird mittels {\tt XtVaAppCreateShell()} bei der Erzeugung eines
   Fensters ermittelt.
\item[\tt app\_context]
   ist der Xt-Kontext der Applikation.
   Er wird mittels {\tt XtCreateApplicationContext()} bei der Erzeugung
   der ersten {\tt XCONTEXT} Struktur ermittelt.
\item[\tt argc]
   gibt die Gr"o"se des Feldes {\tt argv} an. Der Wert wird von {\tt main()}
   durchgereicht.
\item[\tt argv]
   ist ein Verweis auf eine Liste von Argumenten. Er wird von {\tt main()}
   durchgereicht.
\item[\tt form]
   ist das Widget, welches den Viewport und die Men"uleiste enthaelt.
   Es wird mittels {\tt XtVaCreateWidget} beim Erzeugen eines Fensters
   ermittelt.
\item[\tt global\_menu]
   ist das Widget, welches die Men"uleiste enth"alt.
   Es wird mittels {\tt CreateGlobalMenu} beim Erzeugen eines Fensters
   ermittelt.
\item[\tt viewport]
   ist das Widget, welches den Viewport enth"alt.
   Es wird mittels {\tt XtVaCreateWidget} beim Erzeugen eines Fensters
   ermittelt.
\item[\tt h\_scrollbar]
   ist das Widget, welches den horizontalen Scrollbar des Viewports
   darstellt. Es wird mittels {\tt XtNameToWidget} beim Erzeugen eines 
   Fensters ermittelt.
\item[\tt v\_scrollbar]
   ist das Widget, welches den vertikalen Scrollbar des Viewports
   darstellt. Es wird mittels {\tt XtNameToWidget} beim Erzeugen eines 
   Fensters ermittelt.
\item[\tt tree]
   ist das Widget, welches den Baum darstellt.
   Es wird mittels {\tt XtVaCreateWidget} beim Erzeugen eines Fensters
   ermittelt.
\item[\tt attr\_mask]
   ist ein Verweis auf die zu einem Fenster geh"orende Attributmaske.
   Er wird von den Funktionen {\tt GetAttrMask()} oder 
   {\tt CopyAttrMask()} erzeugt.
\item[\tt max\_tree]
   ist ein Verweis auf den darzustellenden MAX-Baum. Er wird beim
   Aufruf der Funktion {\tt CreateMainShell()} geliefert.
\item[\tt xcontext]
   ist ein Verweis auf die zum Kontext geh"orende XCONTEXT Struktur.
   Er wird von den Funktionen {\tt CreateXCONTEXT()} oder
   {\tt CopyXCONTEXT()} erzeugt.
\item[\tt incarnation]
   gibt die Position des Kontextes in der hierarchischen Ordnung der
   Kontexte an. Der Kontext des ersten Fensters hat den Wert 0. Alle
   Anderen haben Werte gr"osser als 0.
\item[\tt number]
   gibt die Nummer des Kontextes in einer hierarchischen Ebene an.
   Die Nummerrierung erfolgt fortlaufend.
\item[\tt children]
   ist ein Verweis auf den Kontext des ersten Kind-Fensters.
\item[\tt next]
   ist ein Verweis auf den Kontext eines Fensters der gleichen
   hierarchischen Ebene, dessen Nummer gr"o"ser ist.
\item[\tt prev]
   ist ein Verweis auf den Kontext eines Fensters der gleichen
   hierarchischen Ebene, dessen Nummer kleiner ist.
\item[\tt parent]
   ist ein Verweis auf den Kontext eines Fensters, aus welchem heraus
   dieses Fenster erzeugt wurde.
   
\end{description}

\subsection{Die Struktur {\tt X\_INFO}}

Die Struktur {\tt X\_INFO} wird f"ur die Verwaltung der Knoten des
MAX-Baumes ben"otigt. Jeder Knoten enth"alt einen Verweis auf eine
eigene Instanz. Sie gibt den Zustand des Knoten an, aus dem das 
aktuelle Darstellungformat ausgelesen werden kann.

\begin{verbatim}
typedef struct _XIS_ {
   Context      *context; 
   Widget       widget; 
   Boolean      hidden_tree; 
   Boolean      show_attributes; 
   Boolean      marked; 
   Widget       attr_widget; 
   AttrMask     *attr_mask;
} X_INFO;
\end{verbatim}

\begin{description}
\item[\tt context]
   ist ein Verweis auf den zu diesem Knoten geh"orenden Kontext.
\item[\tt widget]
   ist das Widget, welches den Knoten auf dem Bildschirm darstellt.
\item[\tt hidden\_tree]
   ist ein Flag, das angibt, ob ein eventuell vorhandener Unterbaum des Knotens
   zusammengefaltet ist.
\item[\tt show\_attributes]
   ist ein Flag, das angibt, ob die Attribute des Knotens angezeigt werden.
\item[\tt marked]
   ist ein Flag, das angibt, ob der Knoten markiert dargestellt wird.
\item[\tt attr\_widget]
   ist das Widget, welches die Attribute auf dem Bildschirm darstellt.
   Dieses Widget existiert nur dann, wenn das Flag {\tt show\_attributes}
   auf {\tt True} gesetzt ist.
\item[\tt attr\_mask]
   ist ein Verweis auf die Attributmaske die zur generierung des 
   Attributwidgets benutzt wurde. Die Maske ist nur dann gespeichert,
   wenn das Flag {\tt show\_attributes} auf {\tt True} gesetzt ist.
\end{description}
