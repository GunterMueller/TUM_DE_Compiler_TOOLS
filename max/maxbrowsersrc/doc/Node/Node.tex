
\section{Das Node Widget}

\begin{description}

\item[Name] 
   {\bf:\\}
   Node Widget --- Knoten des Attributbaumes der mit allen drei Maustasten 
   bedient werden kann.

\item[Synopsis]
   {\bf:\\}
   \begin{tabular}{ll}
   {\bf Public Headers:} & {\it $<$Node.h$>$} \\
   {\bf Private Header:} & {\it $<$NodeP.h$>$} \\
   {\bf Class Name:} & Node \\
   {\bf Class Pointer:} & {\tt nodeWidgetClass} \\
   {\bf Instantiation:} & 
		 {\tt widget = XtCreateWidget(name,nodeWidget,Class,...)}
   \end{tabular}

\item[Klassenhierarchie]
   {\bf:\\}
   Core$\rightarrow$ Composite$\rightarrow$ Node

\item[Beschreibung]
   {\bf:\\}
   Das Node Widget ist ein reines Layout Widget und wurde zur 
   einfacheren Handhabung der attributierten 
   Knoten in einem Syntaxbaum entwickelt. Es unterst"utzt das Darstellen
   von verschatteten Unterb"aumen und markierten Knoten,
   sowie das Anzeigen bzw. Unterdr"ucken
   der zum Knoten geh"orenden Attribute. Das Node Widget erwartet vom
   Anwender ein Unterwidget, welches die darzustellenden Attribute enth"alt.
   F"ur die Benutzerkommunikation wurde das Command Widget benutzt, wobei
   die Translation Tabelle f"ur die Unterst"utzung aller drei Maustasten
   erweitert wurde.

\item[Resourcen]
   {\bf:\\}
   \begin{tabular}{l|l|l|p{3cm}}
   Name({\tt XtN...}) & Typ & Default & Beschreibung \\ \hline
   {\tt XtNcallback} & {\tt XtCallbackList} & {\tt NULL} & 
   Callback f"ur Anklicken des Buttons. \\
   {\tt XtNcommandButton} & {\tt Widget} & Widget ID &
   Widget ID des Command Buttons. \\
   {\tt XtNforeground} & {\tt Pixel} & {\tt XtDefaultForeground} &
   Farbe des Vordergrundes. \\
   {\tt XtNhiddenTree} & {\tt Boolean} & {\tt FALSE} & 
   Gibt an ob der Knoten Wurzel eines verschatteten Unterbaums ist. \\
   {\tt XtNhighlightWidth} & {\tt Dimension} & {\tt 2} &
   Linienst"arke der Umrandung beim Hervorheben. \\
   {\tt XtNlabel} & {\tt String} & ''Node`` &
   Text im Button. \\
   {\tt XtNlineWidth} & {\tt Dimension} & {\tt 1} &
   Linienst"arke der Umrandung. \\
   {\tt XtNmarked} & {\tt Boolean} & {\tt FALSE} &
   Gibt an ob der Knoten als markiert dargestellt werden soll. \\
   {\tt XtNshowAttributes} & {\tt Boolean} & {\tt FALSE} &
   Gibt an ob das Attributwidget des Knotens angezeigt werden soll.
   \end{tabular}

\item[Layout]
   {\bf:\\}
   Ein Knoten wird immer durch einen Button dargestellt, der mit
   allen drei Maustasten bedient werden kann.
   Das Layout des Node Widgets kann aber auch durch die Werte der Resourcen 
   {\tt XtNhiddenTree}, {\tt XtNmarked} und {\tt XtNshow\-Attributes}
   ver"andert werden.
   Wenn ein Wert mittels {\tt XtSetValue()} gesetzt wurde, 
   wird automatisch ein Update der Widgetdarstellung durchgef"uhrt.

   Die Belegungen haben folgende Auswirkungen auf die Darstellung:
   \begin{description}
   \item[{\tt XtNhiddenTree}:]
      Falls {\tt XtNhiddenTree} mit dem {\tt TRUE} belegt wurde, 
      wird der Knoten unten um ein Dreieck erweitert.
   \item[{\tt XtNmarked}:]
      Falls {\tt XtNmarked} mit dem Wert {\tt TRUE} belegt wurde,
      wird der Button markiert -- d.h. Fordergrund- und Hintergrundfarbe
      wird vertauscht -- dargestellt.
   \item[{\tt XtNshowAttributes}:]
      Falls {\tt XtNshowAttributes} mit dem Wert {\tt TRUE} belegt wurde,
      wird oberhalb des Buttons das vom Anwender bereitgestellte
      Attributwidget angezeigt. 
      {\bf Achtung:} {\tt XtNshowAttributes} darf nicht mit {\tt TRUE} 
      belegt werden, wenn noch kein Attributwidget als ''Child``
      des Node Widgets generiert wurde.
   \end{description}

\item[Translations und Actions]
   {\bf:\\}
   Das Node Widget selber besitzt keine eigene Translation Table.

   Die folgende Tabelle beschreibt die Translation Table des Command Buttons:

   \begin{tabular}{ll} 
   {\tt $<$Btn1Down$>$} & {\tt : set()} \\
   {\tt $<$Btn1Up$>$} & {\tt : notifyNode(1) unset()} \\
   {\tt $<$Btn2Down$>$} & {\tt : set()} \\
   {\tt $<$Btn2Up$>$} & {\tt : notifyNode(2) unset()} \\
   {\tt $<$Btn3Down$>$} & {\tt : set()} \\
   {\tt $<$Btn3Up$>$} & {\tt : notifyNode(3) unset()} \\
   {\tt $<$EnterWindow$>$} & {\tt : highlightNode() highlight()} \\
   {\tt $<$LeaveWindow$>$} & {\tt : unhighlightNode() reset()}
   \end{tabular}

   Um die Translation Table zu ver"andern mu"s mittels {\tt XtGetValue()}
   die Widget ID des Command Buttons aus der Resource {\tt XtNcommandButton}
   ausgelesen werden. Danach kann dann die Translation Table in "ublicher
   Art und Weise mit der ID des Command Buttons ver"andert werden.
   Die Funktionen {\tt set()}, {\tt unset()}, {\tt highlight()} und 
   {\tt reset()} wurden vom Command Widget "ubernommen. 
   Hinzugekommen sind die Funktionen {\tt notifyNode()}, 
   {\tt highlightNode()} und {\tt unhighlightNode()}.
   Folgende Tabelle beschreibt ihre Wirkung:

   \begin{description}
   \item[{\tt notifyNode()}:]
      F"uhrt die in der Callback Liste {\tt XtNcallback} eingeh"angten
      Funktionen aus, wobei der Wert des Arguments {\tt call\_data}
      der Integerwert des Parameters der Funktion {\tt notifyNode()}
      ist.
   \item[{\tt highlightNode()}:]
      Sorgt f"ur ein Neuzeichnen des Node Widgets, wobei die Linienst"arke
      durch den Wert in {\tt XtNhighlightWidth} bestimmt wird.
   \item[{\tt unhighlightNode()}:]
      Sorgt f"ur ein Neuzeichnen des Node Widgets, wobei die Linienst"arke
      durch den Wert in {\tt XtNlineWidth} bestimmt wird.
   \end{description}

\item[Callback Parameter]
   {\bf:\\}
   Das Node Widget unterst"utzt nur die unter {\tt XtNcallback} 
   eingehaengten Callback Funktionen. Bei unver"anderter Translation Tabelle
   wird dem {\tt call\_data} Argument die Nummer der jeweiligen Maustaste
   als Integerwert "ubergeben.
   Auf den Wert kann mittels

   {\tt int botton\_number = (int) call\_data;}

   zugegriffen werden.
\end{description}

