\section{Zusammenspiel von Browser und MAX}
Zur Laufzeit des Browsers werden Informationen zur Visualisierung 
ben"otigt, die im generierten Compiler nicht mehr explizit zur
Verf"ugung stehen, z.B. die Sortennamen, wie sie in der Grammatik
definiert wurden. Deshalb m"ussen diese Informationen gerettet
werden und dem Browser zur Verf"ugung gestellt werden.\\
Die Informationen werden von MAX in eine sprachspezifische Datei mit
der Endung {\tt .g} ausgegeben. Hier st"unden sie theoretisch dem
Zielsystem zur Verf"ugung. Es w"are aber die Folge, da\3 dem Compiler
auch immer die {\tt .g} Datei vorliegen m"usste. Um dies zu vermeiden
wird aus der {\tt .g} Datei C-Quelltext erzeugt, der "ubersetzt und
zu dem Zielcompiler hinzugebunden wird. Die Erzeugung des Quelltexts
"ubernimmt das Programm {\tt bim}.\\
Der sprachunabh"angige Teil des Browsers liegt in einem Archiv vor
(Endung {\tt .a}) und wird einfach zum Zielcompiler hinzugebunden.
Die Verarbeitung der Eingabedateien im Vergleich MAX mit/ohne Browser
sieht also wie folgt aus:
	\psfig{figure=bim/bim_use.eps}


\section{Algorithmen und Datenstrukturen des MAX-Browser Interfaces}
Dieses Dokument beschreibt die Algorithmen und Datenstrukturen,
die f"ur das Zusammenspiel zwischen MAX und dem Browser von Bedeutung 
sind.\\
Die Komponenten des Interfaces sind:\\
\begin{itemize}
	\item nodeinfo - generelle Definitionen f"ur den Baum - und
	\item bim\_util - Operationen auf dem Baum,
	
\end{itemize}

\subsection{nodeinfo.h}
Diese Include-Datei definiert die Datenstruktur f"ur den dargestellten
Baum. Die darin definierten Strukturen sind:
\begin{itemize}
	\item ATTR\_INFO - Informationen "uber die Attribute eines Knotens
	\item NODE\_INFO - Beziehung zur Grammatik
	\item MAX\_NODE - Verwaltet die Baumstruktur.
\end{itemize}
\begin{verbatim}
typedef char * ATTR_TYPE;

typedef struct _AIS_ {

   char * name;            /* Name of the attribute */
   char * defined_by_sort; /* Sort that defined this attribute */
   ATTR_TYPE type;         /* Type of the attribute */
   long attr_value;        /* Value of the attribute, has to be interpreted 
                           // Depending on the type
                           */
   
}ATTR_INFO;

\end{verbatim}
\begin{itemize}
	\item name: Enth"alt den Namen des Attributs, wie er
	in der Grammatik definiert wurde.
	\item defined\_by\_sort: Enth"alt den Namen der Sorte, f"ur die 
	dieses Attribut in der Grammatik definiert wurde. Dies ist nicht
	notwendigerweise die Sorte des Knotens, f"ur den dieses Attribut
	berechnet wurde, wenn es sich z.B. um ein Attribut handelt, das
	ererbt wurde.
	\item type: Definiert den Typ des Attributs. Dies ist einer der
	vom Browser darstellbaren Typen Int, String, Bool, Point, MAX\_NODE
	(Verweis auf einen anderen Knoten)
	\item attr\_value: Enth"alt den eigentlichen Wert des Attributs.
	Dieses Feld mu\3 je nach Typ unterschiedlich interpretiert werden.
	Als Verbesserung k"onnte man hier vielleicht mal eine union
	draus machen, l"auft aber auch so.
\end{itemize}
\begin{verbatim}
typedef struct _NIS_ {
   char * name;            /* Sort of this Node as C-String */
   char ** parent_sorts;   /* NULL-pointer terminated array of parent sorts */
   int no_of_attributes;   /* Number of attributes that are defined for this 
                           // node
                           */
   ATTR_INFO * attributes; /* Array of attribute descriptions */
}NODE_INFO;

\end{verbatim}
\begin{itemize}
	\item name: Name der Knotensorte
	\item parent\_sort: Array der Namen der Sorten, von denen diese
	Sorte abgeleitet ist, NULL-Pointer terminiert.
	\item no\_of\_attributes: Anzahl der Attribute, die f"ur diese
	Sorte definiert wurden.
	\item attributes: Zeiger auf das erste Attribut dieses Knotens,
	die weiteren Attribute k"onnen "uber Array-Indizierung angesprochen
	werden (attributes[0]...attributes[ no\_of\_attributes - 1 ]). Hat
	dieser Knoten keine Attribute, so ist der Inhalt dieses Felds
	nicht definiert.
\end{itemize}
\begin{verbatim}
typedef struct _NOD_ {
   struct _NOD_ * parent;     /* Parent node */
   struct _NOD_ * children;   /* First child node */
   struct _NOD_ * next, * prev; /* Linked list of children */
   NODE_INFO * node_info;     /* Information about the node */
   X_INFO      * x_info;        /* Information for Xt handling */
   void * origin_node;
} MAX_NODE;

\end{verbatim}
\begin{itemize}
	\item parent: Zeiger auf Vaterknoten
	\item children: Zeiger auf ersten Sohn, die Kinder sind mit
	\item prev, next: in einer doppelt verketteten Liste verbunden.
	\item node\_info: Zeigt auf die Sorteninformation dieses Knotens.
	\item x\_info: Zeiger auf die grafische Information.
	\item origin\_node: Wird beim Kopieren eines Baumes ben"otigt,
	siehe auch bim\_util.c
\end{itemize}

\subsection{bimutil.c}
\begin{verbatim}
extern struct MAX_BIM_RELATION {
   long max_node;
   MAX_NODE * bim_node;
} * max_bim_relation; 
\end{verbatim}
Diese Struktur stellt die Beziehung zwischen den Knoten, wie sie
von MAX geliefert werden und den Browserknoten her. Beim Baumaufbau
wird ein Feld mit diesen Informationen erstellt und sp"ater
beim Kopieren des Baumes und beim Verzweigen durch Terme wieder
ben"otigt.\\

Beim ersten Erzeugen eines Baumes m"ussen die Knotenattribute voom Tyo
MAX\_NODE (Knotenverweis) auf die richtigen Werte gesetzt werden. Nach Aufbau des Baumes enthalten sie noch die Knoten, wie sie von MAX geliefert werden
(long-Werte), der Browser ben"otigt jedoch Zeiger auf MAX\_NODE. Deswegen
wird der Baum einmal durchlaufen und f"ur jedes MAX\_NODE-Attribut mittels
der Tabelle max\_bim\_relation der richtige Wert ermittelt.

Bim\_CopyMaxTree kopiert einen Baum. Dazu werden im zu kopierenden
Baum in allen Knoten das Feld origin\_node auf NULL gesetzt. Dann
wird mit Bim\_CopyMaxTree\_Do rekursiv eine Kopie des Baums erzeugt.
Wichtig ist hierbei, das jeweils in dem Quellknoten das Feld origin\_node
so gesetzt wird, da\3 es auf die Kopie zeigt. Da jetzt im neuen Baum die
Attribute, die Verweise auf Knoten sind, noch in den Quellbaum zeigen, ist
ein korrigieren dieser Verweise n"otig. Dazu l"auft Bim\_UpdateAttributesOnCopy
einmal "uber den neuen Baum. F"ur jeden Knotenverweis "uberpr"uft es, ob in dem
referenzierten Knoten das Feld origin\_node gesetzt ist. Falls ja, zeigt origin\_node auf den Knoten, auf den das Attribut zeigen soll. Das Attribut kann
somit leicht umgesetzt werden.

\subsection{bim\_std.c}
Diese Datei enth"alt die Routinen, die zum Aufbau des Baumes ben"otigt
werden und unabh"angig vom Zielsystem sind. Diese Quelldatei wird in
die spezifische Datei einkopiert, und stellt somit ihre Routinen f"ur
das Zielsystem zur Verf"ugung.\par
Die Routine {\tt bim\_tree} baut aus einem Baum,
wie er von MAX geliefert wird den vom Browser ben"otigten Baum auf.
Die Signatur dieser Funktion ist 
{\tt MAX\_NODE * bim\_tree( ELEMENT n, MAX\_NODE * parent, MAX\_NODE * broth )}.\par
Argument {\tt n} ist der Baum, der visualisiert werden soll. Diese Funktion
arbeitet rekursiv und ben"otigt die beiden letzten Argumente f"ur ihre 
rekursiven Aufrufe. Beim ersten Aufruf m"ussen diese Werte auf {\tt NULL}
gesetzt werden.\par
In dieser Funktion wird zuerst versucht, Speicher f"ur den neuen Knoten zu
belegen, dann werden einige Werte initialisiert und die {\tt x\_info}-Struktur
erstellt. Nachdem die Eltern-Kind-Verweise gesetzt wurden, werden die
Sorten- und Attributinformationen ausgef"ullt und die Beziehung MAX-Knoten
zu Browserknoten in eine Tabelle eingetragen. Dann wird rekursiv der Rest
des Baumes aufgebaut.\par

Um einen MAX-Baum zu visualisieren, ben"otigt man die Routine {\tt 
Bim\_Viewer}. Sie ben"otigt als Argumente den MAX-Baum und die Argumente
{\tt argc} und {\tt argv} wie sie der Routine {\tt main()} "ubergeben werden.
Die Kommandozeilenargumente werden ben"otigt, um bei der Darstellung des
Baumes die Schalter, die von X akzeptiert werden, richtig setzen zu k"onnen,
also z.B. Wahl des Zeichensatzes, der Hintergrundfarbe etc.\par
In dieser Routine wird zuerst die Anzahl der ben"otigten Knoten ermittelt.
Dann wird Speicherplatz f"ur die Beziehungstabelle MAX-Knoten zu Browserknoten
belegt. Nachdem die Liste aller Sorten aufgebaut wurde, wird der Browserbaum
mit {\tt bim\_tree} erstellt. Als MAX-Baum wird immer {\tt root()} verwendet.
Nachdem die Attribute aktualisiert wurden, wird der Baum dargestellt. Nach
Beendigung der Visualisierung, wird ben"otigter Speicherplatz wieder freigegeben.\par

Da MAX-Attribute auch Terme sein k"onnen, m"ussen diese Terme zur 
Visualisierung in Browserb"aume umgewandelt werden. Die zentrale Funktion
zur Erstellung eines Baums aus einem Term ist {\tt MAX\_NODE * parse\_max\_term( long term, long left\_side )}. Hierbei ist {\tt term} der umzuwandelnde
Term und {\tt left\_side} die Sorte, nach der geparst werden soll. Zu 
Anfang wird in der Parsetabelle gesucht, ob sich Information "uber diese
Sorte findet, falls nicht, wird ein Knoten erzeugt und zur"uckgegeben. 
Andernfalls wird abh"angig von der Produktion der Term weitergeparst mit den
Funktionen {\tt parse\_class\_prod}, {\tt parse\_list\_prod} und
{\tt parse\_tuple\_prod}. Diese bauen jeweils rekursiv die ben"otigten
Unterb"aume auf.

\subsection{42.c}
Die f"ur jedes Zielsystem spezifisch generierte Datei {\tt 42.c} enth"alt
grammatikabh"angige Datenstrukturen und Funktionen.\\ Die Datei enth"alt in
der Reihenfolge ihres Auftretens:\\
\begin{itemize}
\item Die Tabelle tpt, die Informationen "uber die Sortenproduktionen in
der Grammatik enth"alt. Diese Tabelle wird zur Erstellung der B"aume
von Termattributen ben"otigt.
\item Die Liste der Sortennamen. Ein {\tt NULL}-Pointer terminiertes Array
von Stringpointern.
\item Die Liste der Attributnamen. Ein {\tt NULL}-Pointer terminiertes Array
von Stringpointern. Das vorletzte Element des Feldes ist der String
 ``SEMVAL``, der f"ur semantische Werte ben"otigt wird.
\item Information "uber die Sortenvererbung. F"ur jede Sorte exisitiert ein
Feld von Strings, das angibt, von welchen Sorten diese Sorte abgeleitet wurde.
\item Attributfunktionen. F"ur jede Sorte exisitiert eine Funktion {\tt
attributes\_\sl SORTE}, die die Attribute eines Knotens und deren Anzahl
ermittelt.
\item Routinen zur Ermittlung von Sortennamen, Anzahl der Attribute, 
Sortenabh"angigkeiten und Attributen. 
\item den einkopierten Inhalt der Datei {\tt bim\_std.c}.

\end{itemize} 















