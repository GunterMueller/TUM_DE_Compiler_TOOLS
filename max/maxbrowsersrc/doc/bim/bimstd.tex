\section{Funktionsbeschreibung des Moduls {\tt bimstd.c}}

Dieses Modul stellt die Funktionen zur Verf�gung, die in jedes {\tt 42.c}
eingebunden werden, unabh"angig von der Umgebung.

\begin{description}
\item[\tt enter\_max\_bim\_relation
]{\bf :\\}
Erzeugt einen Querverweis von einem Knoten des MAX-Baums
 auf einen Knoten des Browserbaums. Dies wird benoetigt, um nach
 Erstellen des Browserbaums die Attribute, die Knotenverweise sind
 richtig zu setzen. Dazu wird in einer Tabelle ein Eintrag mit
 der Nummer des MAX-Knotens und des Browserknotens erzeugt.
  \\
IN:
\begin{itemize}
   \item long max\_node: Nummer des Knotens im MAX-Baums
\item MAX\_NODE * bim\_node: Zugeh"origer Browserknoten


\end{itemize}
\item[\tt lookup\_max\_relation
]{\bf :\\}
Findet zu einem MAX-Knoten den zugeh"origen Browserknoten.
 Der Knoten mu\3 vorher mit {\tt enter\_max\_bim\_relation} in die
 Querverweisliste eingetragen worden sein.
  \\
IN:
\begin{itemize}
   \item long max\_node: MAX-Knoten zu dem der Browserknoten gefunden
werden soll.

\end{itemize}
OUT:
\begin{itemize}
   \item MAX\_NODE *: NULL: falls kein Browserknoten f"ur den
MAX-Knoten existiert.
Verweis auf den Knoten sonst.

\end{itemize}

\item[\tt alloc\_relation\_table
]{\bf :\\}
Belegt Speicher f"ur die Beziehungstabelle <MAX-Knoten>
 zu Browserknoten.
  \\
\item[\tt free\_relation\_table
]{\bf :\\}
Beziehungstabelle freigeben.
 \\
\item[\tt Bim\_Viewer
]{\bf :\\}
Die Funktion die aufgerufen werden muss, um den Baum
 am Bildschirm darzustellen. Dazu wird zuerst die Anzahl der Knoten
 im Baum ermittelt. Dann wird die Beziehungstabelle belegt.
 Ferner wird die Liste der Sorten alphabetisch sortiert. Dann wird
 der Browserbaum aus dem MAX-Baum aufgebaut.
  \\
IN:
\begin{itemize}
   \item long n: Der MAX-Baum
\item int argc, char *argv[]: Kommandozeilenparameter wie sie von
main() kommen.

\end{itemize}
\item[\tt bim\_tree
]{\bf :\\}
Baut den Browserbaum aus dem MAX-Baum auf.
 \\
\item[\tt find\_production
]{\bf :\\}
Sucht in der Parsetabelle einer Produktion nach deren
 linken Seite.
 \\
IN:
\begin{itemize}
   \item long left\_side : Linke Seite nach der gesucht werden soll.

\end{itemize}
OUT:
\begin{itemize}
   \item int: Index des Eintrags, oder -1, falls kein solcher
Eintrag exisitiert.

\end{itemize}

\item[\tt term\_attr\_node
]{\bf :\\}
Erzeugt einen Browserknoten aus einem Term.
 \\
IN:
\begin{itemize}
   \item long term: Term aus dem Knoten erzeugt wird.

\end{itemize}
OUT:
\begin{itemize}
   \item MAX\_NODE * : Zeiger auf Knoten im Erfolgsfall
NULL sonst.

\end{itemize}

\item[\tt parse\_list\_prod
]{\bf :\\}
Parst einen Term gegen eine Listenproduktion.
 \\
IN:
\begin{itemize}
   \item long term : Zu Parsender Term
\item int idx : Eintrag der Produktion

\end{itemize}
OUT:
\begin{itemize}
   \item MAX\_NODE *: Zeiger auf Baum oder NULL

\end{itemize}

\item[\tt parse\_tuple\_prod
]{\bf :\\}
Parst einen Term gegen eine Tupelproduktion.
 \\
IN:
\begin{itemize}
   \item long term : Zu Parsender Term
\item int idx : Eintrag der Produktion

\end{itemize}
OUT:
\begin{itemize}
   \item MAX\_NODE *: Zeiger auf Baum oder NULL

\end{itemize}

\item[\tt parse\_class\_prod
]{\bf :\\}
Parst einen Term gegen eine Classproduktion.
 \\
IN:
\begin{itemize}
   \item long term : Zu Parsender Term
\item int idx : Eintrag der Produktion

\end{itemize}
OUT:
\begin{itemize}
   \item MAX\_NODE *: Zeiger auf Baum oder NULL

\end{itemize}

\item[\tt parse\_max\_term
]{\bf :\\}
Erzeugt aus einem Term einen Browserbaum. Diese Funktion
 wird zur Darstellung von Termattributen benoetigt. Zum Parsen
 wird die von 42.c bereitgestellte Tabelle benoetigt.
  \\
IN:
\begin{itemize}
   \item long term: Term aus dem ein Browserbaum erzeugt werden soll.
long left\_side: Sorte nach der dieser Term geparst werden
soll.

\end{itemize}
OUT:
\begin{itemize}
   \item MAX\_NODE * : Zeiger auf Baum im Erfolgsfall
NULL		sonst

\end{itemize}
\end{description}
