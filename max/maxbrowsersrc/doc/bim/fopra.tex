\documentstyle[12pt,german,a4]{article}
\newcommand{\MAX}{{\sf MAX }}
\newcommand{\BIM}{{\sf BIM }}
\begin{document}
\section{Einf"uhrung}

Im Rahmen unseres Fortgeschrittenenpraktikums entwickelten und implementierten
wir einen grafischen Browser f"ur das \MAX-System.

\section{Problemstellung und L"osung}
F"ur das \MAX-System war ein grafischer Browser f"ur Parseb"aume zu entwickeln,
der die Kontrolle und das Debuggen des Zielcompilers vereinfachen soll. Dabei
waren folgende Punkte ausschlaggebend:\\
\begin{enumerate}
    \item Unabh"angigkeit vom Zielcompiler: Der Browser soll f"ur alle mit
	\MAX erzeugten Systeme verwendbar sein.
    \item Informationsf"ulle: Dem Benutzer sollen m"oglichst alle Informationen
	"uber den erzeugten Baum angeboten werden.
    \item Benutzerfreundlichkeit: Ein einfaches und effektives Arbeiten mit dem
	Browser soll m"oglich sein.
\end{enumerate}

Den Punkt Benutzerfreundlickeit beschreibt ein anderer Teil der Dokumentation. Hier
wird nun erl"autert, wie die Punkte Unabh"angigkeit und Informationsf"ulle realisiert
wurden.\\

Den Ansatz, einen Modul zu entwickeln, der sich f"ur alle Zielsysteme verwenden
l"a\3t, mu\3ten wir schnell verwerfen, da sich in den Zielsystemen nicht mehr
gen"ugend Informationen "uber die zu Grunde liegende Grammatik findet, als da\3
ein Parsebaum zufriedenstellend darstellbar w"are. Deswegen mu\3ten wir das Projekt
in zwei Teile aufspalten: Die Visualisierungsroutinen (Baumdarstellung, Navigation,
etc.), die f"ur jedes Zielsystem gleich sind, und die Erstellung des Baumes, f"ur
den je nach Grammatik unterschiedliche Schritte notwendig sind. Beide Teile zusammen
stellen dann den Browser dar.\\

\subsection{Modifikationen am \MAX-System}
Ein Gro\3teil der Informationen, die wir visualisieren, sind im Zielcompiler nicht
mehr vorhanden. Sie m"u\3en daher w"ahrend des \MAX-Laufs gerettet werden, soda\3
sie uns zur Verf"ugung stehen. Daher wurde von uns in Absprachen mit Arnd Poetzsch-Heffter
in \MAX die Erzeugung der Datei {\tt <Sprache>.g} eingebaut, die diese Informationen
f"ur uns enth"alt. Diese Datei enth"alt folgende Informationen:\\
\begin{itemize}
    \item Liste aller Sorten und deren Klassenabh"angigkeiten. Wird zur Benennung
	der Baumknoten ben"otigt.
    \item Definition der Klassen, Tupel und Listen. Wird ben"otigt, um Attribute, die
	Terme sind, als Baum darstellen zu k"onnen.
    \item Signaturen der Attributfunktionen. Wird zur Darstellung der Attribute eines
	Knotens ben"otigt.
\end{itemize}
Die genaue Struktur der Datei findet sich im Anhang.

\subsection{Die sprachspezifische Datei}
Aus der Datei {\tt <Sprache>.g} erzeugen wir mit Hilfe des Tools \BIM (Browser
Interface to \MAX) ein grammatikspezifisches Modul {\tt 42.c}, das die Routinen
zum Aufbau und zur Auswertung des Baumes enth"alt. Die Beschreibung von {\tt 42.c}
findet sich im Anhang.

\section{\BIM,{\tt .g} und {\tt 42.c}}
Die Datei {\tt <Sprache>.g} wird von \BIM mit Hilfe von flex und bison eingelesen,
verarbeitet und daraus C-Code in der Datei {\tt 42.c} erzeugt. Diese Datei enth"alt
vier wesentliche Funktionsgruppen:\\
\begin{itemize}
    \item Bereitstellung von Informationen "uber die Grammatik (Zahl der Sorten,
	Attribute etc.)
    \item Bereitstellung von Informationen "uber einen \MAX-Element
    \item Aufbau eines grafischen Baumes aus einem \MAX-Baum und
    \item Aufbau eines grafischen Baumes aus einem \MAX-Term.
\end{itemize}
Wiederkehrende Funktionen wurden in die Datei {\tt bim\_std.c} ausgelagert, diese
wird von \BIM automatisch einkopiert und mu\3 beim Lauf von \BIM immer im aktuellen
Verzeichnis sein. Die bereitgestelleten Funktionen werden nun genauer erl"autert:

\subsubsection{Grammatikinformationen}
    \begin{itemize}
	\item Klassenhierarchie der Sorten
	\item Liste aller Attribute
	\item Liste aller Sorten
    \end{itemize}
\subsubsection{Elementinformationen}
    \begin{itemize}
	\item Anzahl der Attribute eines \MAX-Elements.
	\item Sortenname des Elements
	\item Liste aller Attribute
    \end{itemize}
\subsubsection{Aufbau eines Baumes aus einem \MAX-Baum}
    Beim Aufbau eines Baumes aus einem \MAX-Baum wird der \MAX-Baum rekursiv
    durchlaufen. F"ur jeden \MAX-Knoten wird ein Knoten angelegt und mit Informationen
    zur Visualisierung ausgef"ullt. Alle zur Darstellung benoetigten Daten werden ermittelt,
    noch bevor das Fenster auf dem Schirm ge"offnet wird. Besondere
    Beachtung erfordern hier die Attribute, die Verweise auf andere Knoten sind.\\
    Ein Knotenattribut beschreibt nach dem Erstellen unseres Baumes immer noch einen
    \MAX-Knoten. Deswegen ist ein Durchlaufen durch den Baum n"otig, um aus den
    Verweisen auf \MAX-Knoten Verweise auf grafische Knoten zu machen. Dies wird
    mit der Routine {\tt Bim\_UpdateAttributes()} aus dem Modul {\tt bim\_util.c}
    erm"oglicht.

	Ein Knoten im grafischen Baum hat folgende Struktur:

	\begin{verbatim}
   typedef struct _NOD_ {
   struct _NOD_ * parent;     /* Parent node */
   struct _NOD_ * children;   /* First child node */
   struct _NOD_ * next, * prev; /* Linked list of children */
   NODE_INFO * node_info;     /* Information about the node */
   X_INFO      * x_info;        /* Information for Xt handling */
   void * origin_node;
	}MAX_NODE;
	\end{verbatim}

	\begin{tabular}{ll}
	\tt parent & Verweis auf den Vaterknoten\\
	\tt children & Verweis auf den ersten Sohn\\
	\tt next, prev & Verkettung aller S"ohne eines Knotens\\
	\tt node\_info & Spezifische Information. Siehe unten\\
	\tt x\_info & Grafische Information. Siehe unten\\
	\tt origin\_node & Wird zum Herstellen der Verweise ben"otigt.\\
	\end{tabular}\\

	{\tt x\_info} ist ein Teil der Grafik und wird deswegen in dem Teil der Dokumentation
	beschrieben, der die X-Programmierung behandelt.

	In dieser Struktur befindet sich keinerlei Information "uber die Attribute des Baumes.
	Der Bezug zu \MAX wird mit dem Verweis {\tt node\_info} hergestellt.

	\begin{verbatim}
	typedef struct _NIS_ {
   char * name;            /* Sort of this Node as C-String */
   char ** parent_sorts;   /* NULL-pointer terminated array of parent sorts */
   int no_of_attributes;   /* Number of attributes that are defined for this 
                           // node
                           */
   ATTR_INFO * attributes; /* Array of attribute descriptions */
	 
	}NODE_INFO
	\end{verbatim}
	\begin{tabular}{ll}
	\tt name & Name der Sorte dieses Knotens\\
	\tt parent\_sorts & Liste der Namen der Vatersorten. NULL-terminiert.\\
	\tt no\_of\_attributes & Anzahl der f"ur diesen Knoten definierten Attribute\\
   	\tt attributes & Zeiger auf ein Feld von Attribut-Beschreibungen\\
	\end{tabular}

	\begin{verbatim}
	typedef struct _AIS_ {

   char * name;            /* Name of the attribute */
   char * defined_by_sort; /* Sort that defined this aattribute */
   ATTR_TYPE type;         /* Type of the attribute */
   long attr_value;        /* Value of the attribute, has to be interpreted 
                           // Depending on the type
                           */
   
}ATTR_INFO;
	\end{verbatim}

	\begin{tabular}{ll}
	\tt name & Name des Attributs\\
	\tt defined\_by\_sort & Sorte, f"ur die dieses Attribut definiert ist.\\
	\tt type & Typ des Attributs ( Int, Bool, Knotenverweis, etc. )\\
	\tt attr\_value & Wert des Attributs.
	\end{tabular}
		
\subsubsection{Aufbau eines Baumes aus einem \MAX-Term}
    F"ur den Aufbau eines Baumes aus einem \MAX-Term wird von \BIM eine Tabelle
    angelegt, die Informationen "uber die Produktionen in der Grammatik enth"alt.
    Trifft der Browser beim Baumaufbau auf ein Termattribut, so ''hangelt'' er
    sich mit Hilfe der Tabellen durch den Term und baut einen entsprechenden
    Baum auf. Es handelt es sich hier um einen simplen recursive-descendant-Parser,
	der als Eingabe einen Term nimmt.
\section{Installation}
\subsection{Modifikationen am \MAX-System}
In das \MAX-System m"u\3en an zwei Stellen neue Zeilen eingef"ugt werden: Bei der
Ausgabe der Dateien die Routine zur Ausgabe der {\tt .g}-Datei und der Aufruf des
grafischen Browsers. Beide Aufrufe wuden schon von Roland Merk in die neueste Version
des \MAX eingebaut und stehen deswegen schon zur Verf"ugung.
\begin{appendix}
\subsection{Modifikationen am Makefile}
In das Makefile m"u\3en die neuen Module f"ur den Browser aufgenommen werden.
F"ur die Erzeugung des Moduls 42.o neben der Standard"ubersetzung mit dem Compiler
ist das noch die Erzeugung durch \BIM:\\
\begin{verbatim}
42.c: <sprache.g>
    bim <sprache>
\end{verbatim}
Beim Endlinken m"u\3en dann hinzugelinkt werden: \verb| 42.o browse.a -lX11 -lXt -lXmu|

\section{Funktionsbeschreibung des Moduls {\tt bimsem.c}}

Dieses Modul ist f"ur das Einlesen der Datei {\tt .g} zust"andig.
Es enth"alt die semantischen Aktionen, die von der Bison-Grammatik
aufgerufen werden.

\begin{description}
\item[\tt add\_production
]{\bf :\\}
F"ugt zur Liste der Sortenproduktionen eine neue
 Produktion hinzu.
 \\
IN:
\begin{itemize}
   \item int type: Gibt Art der Produktion an (Liste,Tupel,Klasse)
\item char * left: Zeiger auf linke Seite der Produktion
\item char * right: Zeiger auf rechte Seite

\end{itemize}
\item[\tt sem\_sublist
]{\bf :\\}
Hilft beim Aufbau der Subsorten einer Sorte.
 H"angt an den Buffer eine neue Sorte an, und gibt einen
 Zeiger zurueck, der hinter diese Sorte zeigt.
 \\
IN:
\begin{itemize}
   \item char * dest: Ziel
\item char * source: Ausgangssorte

\end{itemize}
OUT:
\begin{itemize}
   \item char *: Zeiger auf das Ende des Buffers, hier kann die
n"achste Sorte eingeh"angt werden.

\end{itemize}

\item[\tt add\_sort
]{\bf :\\}
F"ugt eine Sorte zur Liste der Sorten hinzu.
 Eine Sonderbehandlung f"ur Sorten vom Typ ''Node'' ist n"otig,
 da diese von nichts abgeleitet sein sollen und f"ur unsere
 Darstellung die Wurzel des Baumes sind.
 \\
IN:
\begin{itemize}
   \item char * name: Name der Sorte
\item struct SORT * parent: Zeiger auf Vatersorte. Ist dies
Null, so wird ''Node'' als Vatersorte angenommen

\end{itemize}
\item[\tt add\_attr
]{\bf :\\}
F"ugt ein Attribut zur Liste der Attribute hinzu.
 \\
IN:
\begin{itemize}
   \item char * asortname: Sorte, f"ur die dieses Attribut
definiert ist
\item char * aname: Attributname
\item char * rsortname: Ergebnissorte

\end{itemize}
\item[\tt add\_classsort
]{\bf :\\}
F"ugt eine Sorte, die eine Klasse ist, zur Liste
 der Sorten hinzu.
 \\
IN:
\begin{itemize}
   \item char * name: Sortenname

\end{itemize}
\item[\tt dump\_info
]{\bf :\\}
Gibt Informationen "uber die internen Tabellen aus.
 Wird nur zum Debuggen ben"otigt.
 \\
\item[\tt find\_sort
]{\bf :\\}
Sucht in der Tabelle der Sorteneintr"age nach einer
 Sorte mit angegebenem Namen.
 \\
IN:
\begin{itemize}
   \item char * name: Sortenname

\end{itemize}
OUT:
\begin{itemize}
   \item struct SORT *: Zeiger auf den Sorteneintrag oder NULL,
falls Eintrag nicht gefunden.

\end{itemize}
\end{description}

\section{Funktionsbeschreibung des Moduls {\tt bimcode.c}}

Dieses Modul ist f"ur die Erzeugung des Files {\tt 42.c} zust"andig.

\begin{description}

\item[\tt semval\_typ
]{\bf :\\}
Ermittelt zu einem Typnamen den dazugeh"origen Typ des semantischen
 Werts. Gibt fuer Typen, die einen semantischen Wert haben, den Typ dieses
 Wertes als C-String zurueck. Typen, denen kein semantischer Wert zugeordnet
 ist, ergeben als Resultat einen NULL-Zeiger.
 \\
IN:
\begin{itemize}
   \item char * typ :	 Name des Typen

\end{itemize}
OUT:
\begin{itemize}
   \item char * : Name des Typen des semantischen Werts

\end{itemize}

\item[\tt actual\_attributes
]{\bf :\\}
Z"ahlt durch, wieviele Attribute eine Sorte wirklich hat, dazu
 summiert die Funktion die Anzahl die Attribute dieses Knoten und aller
 seiner Vorfahren.
 \\
IN:
\begin{itemize}
   \item struct SORT * p : Zeiger auf Sortendefinition

\end{itemize}
OUT:
\begin{itemize}
   \item int : Anzahl der Attribute

\end{itemize}

\item[\tt actual\_node\_attributes
]{\bf :\\}
Z"ahlt durch, wieviele Attribute eine Sorte wirklich hat, wenn
 es sich bei der Instanz um einen Knoten handelt.
 Die Funktion sumimert die Anzahl der Knoten-Attribute dieses Knoten und
 aller seiner Vorfahren.
 \\
IN:
\begin{itemize}
   \item struct SORT * p : Zeiger auf Sortendefinition

\end{itemize}
OUT:
\begin{itemize}
   \item int : Anzahl der Attribute

\end{itemize}

\item[\tt actual\_term\_attributes
]{\bf :\\}
Z"ahlt durch, wieviele Attribute eine Sorte wirklich hat, wenn
 es sich bei der Instanz {\bf nicht} um einen Knoten handelt.
 \\
IN:
\begin{itemize}
   \item struct SORT * p : Zeiger auf Sortendefinition

\end{itemize}
OUT:
\begin{itemize}
   \item int : Anzahl der Attribute

\end{itemize}

\item[\tt bimcode
]{\bf :\\}
Gibt den zu erzeugenden Code in die Datei 42.c aus.
 \\
IN:
\begin{itemize}
   \item char * bn: Name der Sprache

\end{itemize}
\item[\tt replace\_suffix
]{\bf :\\}
Ersetzt ein Dateisuffix durch ein anderes.
 \\
\item[\tt emit\_header
]{\bf :\\}
Gibt den Header der Zieldatei aus.
 \\
\item[\tt emit\_bim\_sort\_name
]{\bf :\\}
Gibt die Funktion aus, die zu jeder Sorte deren Namen als String
 zur"uckgibt
 \\
\item[\tt emit\_sort\_dependency
]{\bf :\\}
Gibt die Liste der Sortenabh"angigkeiten aus
 \\
\item[\tt emit\_bim\_nodeinfo
]{\bf :\\}
Gibt die Routine get\_nodeinfo() aus.
 \\
\item[\tt emit\_bim\_number\_of\_attributes
]{\bf :\\}
Gibt f"ur jede Sorte die Anzahl ihrer Attribute zur"uck.
 \\
\item[\tt emit\_sort\_dependencies
]{\bf :\\}
Gibt die Sortenabh"angigkeiten aus.
 \\
\item[\tt emit\_sort\_attributes
]{\bf :\\}
Gibt die Funktionen aus, die die Sorten berechnen sollen.
 \\
\item[\tt emit\_attr\_list
]{\bf :\\}
Gibt f"ur ein MAX-Element die Liste der Knoten zur"uck
 \\
\item[\tt emit\_list\_of\_sorts\_array
]{\bf :\\}
Gibt die Liste aller vorkommenden Sorten aus.
 \\
\item[\tt emit\_term\_parse\_table
]{\bf :\\}
Gibt die Parsertabelle aus, die zum Erstellen der Darstellung von
 Termattributen n"otig ist.
 \\
\item[\tt emit\_standard\_code
]{\bf :\\}
Kopiert den Code aus {\tt bim\_std.c} her"uber
 \\
\end{description}

\section{Funktionsbeschreibung des Moduls {\tt bimstd.c}}

Dieses Modul stellt die Funktionen zur Verf�gung, die in jedes {\tt 42.c}
eingebunden werden, unabh"angig von der Umgebung.

\begin{description}
\item[\tt enter\_max\_bim\_relation
]{\bf :\\}
Erzeugt einen Querverweis von einem Knoten des MAX-Baums
 auf einen Knoten des Browserbaums. Dies wird benoetigt, um nach
 Erstellen des Browserbaums die Attribute, die Knotenverweise sind
 richtig zu setzen. Dazu wird in einer Tabelle ein Eintrag mit
 der Nummer des MAX-Knotens und des Browserknotens erzeugt.
  \\
IN:
\begin{itemize}
   \item long max\_node: Nummer des Knotens im MAX-Baums
\item MAX\_NODE * bim\_node: Zugeh"origer Browserknoten


\end{itemize}
\item[\tt lookup\_max\_relation
]{\bf :\\}
Findet zu einem MAX-Knoten den zugeh"origen Browserknoten.
 Der Knoten mu\3 vorher mit {\tt enter\_max\_bim\_relation} in die
 Querverweisliste eingetragen worden sein.
  \\
IN:
\begin{itemize}
   \item long max\_node: MAX-Knoten zu dem der Browserknoten gefunden
werden soll.

\end{itemize}
OUT:
\begin{itemize}
   \item MAX\_NODE *: NULL: falls kein Browserknoten f"ur den
MAX-Knoten existiert.
Verweis auf den Knoten sonst.

\end{itemize}

\item[\tt alloc\_relation\_table
]{\bf :\\}
Belegt Speicher f"ur die Beziehungstabelle <MAX-Knoten>
 zu Browserknoten.
  \\
\item[\tt free\_relation\_table
]{\bf :\\}
Beziehungstabelle freigeben.
 \\
\item[\tt Bim\_Viewer
]{\bf :\\}
Die Funktion die aufgerufen werden muss, um den Baum
 am Bildschirm darzustellen. Dazu wird zuerst die Anzahl der Knoten
 im Baum ermittelt. Dann wird die Beziehungstabelle belegt.
 Ferner wird die Liste der Sorten alphabetisch sortiert. Dann wird
 der Browserbaum aus dem MAX-Baum aufgebaut.
  \\
IN:
\begin{itemize}
   \item long n: Der MAX-Baum
\item int argc, char *argv[]: Kommandozeilenparameter wie sie von
main() kommen.

\end{itemize}
\item[\tt bim\_tree
]{\bf :\\}
Baut den Browserbaum aus dem MAX-Baum auf.
 \\
\item[\tt find\_production
]{\bf :\\}
Sucht in der Parsetabelle einer Produktion nach deren
 linken Seite.
 \\
IN:
\begin{itemize}
   \item long left\_side : Linke Seite nach der gesucht werden soll.

\end{itemize}
OUT:
\begin{itemize}
   \item int: Index des Eintrags, oder -1, falls kein solcher
Eintrag exisitiert.

\end{itemize}

\item[\tt term\_attr\_node
]{\bf :\\}
Erzeugt einen Browserknoten aus einem Term.
 \\
IN:
\begin{itemize}
   \item long term: Term aus dem Knoten erzeugt wird.

\end{itemize}
OUT:
\begin{itemize}
   \item MAX\_NODE * : Zeiger auf Knoten im Erfolgsfall
NULL sonst.

\end{itemize}

\item[\tt parse\_list\_prod
]{\bf :\\}
Parst einen Term gegen eine Listenproduktion.
 \\
IN:
\begin{itemize}
   \item long term : Zu Parsender Term
\item int idx : Eintrag der Produktion

\end{itemize}
OUT:
\begin{itemize}
   \item MAX\_NODE *: Zeiger auf Baum oder NULL

\end{itemize}

\item[\tt parse\_tuple\_prod
]{\bf :\\}
Parst einen Term gegen eine Tupelproduktion.
 \\
IN:
\begin{itemize}
   \item long term : Zu Parsender Term
\item int idx : Eintrag der Produktion

\end{itemize}
OUT:
\begin{itemize}
   \item MAX\_NODE *: Zeiger auf Baum oder NULL

\end{itemize}

\item[\tt parse\_class\_prod
]{\bf :\\}
Parst einen Term gegen eine Classproduktion.
 \\
IN:
\begin{itemize}
   \item long term : Zu Parsender Term
\item int idx : Eintrag der Produktion

\end{itemize}
OUT:
\begin{itemize}
   \item MAX\_NODE *: Zeiger auf Baum oder NULL

\end{itemize}

\item[\tt parse\_max\_term
]{\bf :\\}
Erzeugt aus einem Term einen Browserbaum. Diese Funktion
 wird zur Darstellung von Termattributen benoetigt. Zum Parsen
 wird die von 42.c bereitgestellte Tabelle benoetigt.
  \\
IN:
\begin{itemize}
   \item long term: Term aus dem ein Browserbaum erzeugt werden soll.
long left\_side: Sorte nach der dieser Term geparst werden
soll.

\end{itemize}
OUT:
\begin{itemize}
   \item MAX\_NODE * : Zeiger auf Baum im Erfolgsfall
NULL		sonst

\end{itemize}
\end{description}

\section{Funktionsbeschreibung des Moduls {\tt bim\_util.c}}

\begin{description}

\item[\tt Bim\_IsOfSort
]{\bf :\\}
Stellt fest, ob ein Browserknoten von einer bestimmten Sorte
 ist. Dazu wird der Sortenname des Knotens mit dem Parameter verglichen.
 Stimmen diese nicht "uberein, so werden die Sortennamen der Eltern
 verglichen.
 \\
IN:
\begin{itemize}
   \item MAX\_NODE * : Browserknoten, dessen Sorte "uberpr"uft werden soll.
char * name: Sortenname, auf den "uberpr"uft werden soll.

\end{itemize}
OUT:
\begin{itemize}
   \item int:	1 - Ist von dieser Sorte  \\
0 - Nicht von dieser Sorte

\end{itemize}
\item[\tt Bim\_CopyMaxTree\_Do
]{\bf :\\}
Kopiert einen Browserbaum.
  \\
IN:
\begin{itemize}
   \item MAX\_NODE * tree: Baum der kopiert werden soll.
\item MAX\_NODE * parent: Knoten, den der neue Baum als Vater erhalten
soll.

\end{itemize}
OUT:
\begin{itemize}
   \item MAX\_NODE * : Kopie des Baumes unter * tree, mit * parent als
Vater.

\end{itemize}

\item[\tt sort\_sort\_names
]{\bf :\\}
Sortiert eine Liste von Strings alphabetisch. Die Liste muss
 mit einem NULL-Pointer abgeschlossen sein.
  \\
IN:
\begin{itemize}
   \item char * list\_of\_sorts[ ] : Liste von Strings

\end{itemize}
\item[\tt Bim\_NumberOfSorts
]{\bf :\\}
Ermittelt die Anzahl der Sorten in der Grammatik
  \\
\item[\tt Bim\_NumberOfAttributes
]{\bf :\\}
Ermittelt die Anzahl der Attribute in der Grammatik
  \\
OUT:
\begin{itemize}
   \item int : Anzahl der Attribute

\end{itemize}

\item[\tt Bim\_ListOfAttributes
]{\bf :\\}
Gibt die sortierte Liste aller Attribute zur"uck.
  \\
OUT:
\begin{itemize}
   \item char **: Liste der Attributnamen

\end{itemize}

\item[\tt Bim\_ListOfSorts
]{\bf :\\}
Gibt die sortierte Liste aller Sortennamen zur"uck.
  \\
OUT:
\begin{itemize}
   \item char **: Liste der Sortennamen

\end{itemize}

\item[\tt Bim\_UpdateAttributes
]{\bf :\\}
Stellt beim ersten Erzeugen des Browserbaums die richtigen
 Knotenverweise her. Dazu wird die Tabelle der MAX-Knoten/Browser-Knoten
 Abbildungen herangezogen.
  \\
IN:
\begin{itemize}
   \item MAX\_NODE * tree: Zu durchlaufender Baum
\item MAX\_NODE * root: obsolet

\end{itemize}
OUT:
\begin{itemize}
   \item MAX\_NODE * : Wie Eingabe * tree.

\end{itemize}

\item[\tt ClearOriginNode
]{\bf :\\}
Setzt in einem Baum den Wert von origin auf NULL. Dies ist
 vor dem Anpassen der Attribute nach Kopieren des Baums
 notwendig.
  \\
IN:
\begin{itemize}
   \item MAX\_NODE * tree: Baum

\end{itemize}
\item[\tt Bim\_UpdateAttributesOnCopy
]{\bf :\\}
Nach dem Kopieren die Knotenverweise des Baums anpassen. Dazu
 wird das Element {\tt origin} ben"otigt, da\3 beim Kopieren gesetzt
 wurde.
  \\
IN:
\begin{itemize}
   \item MAX\_NODE * tree: Baum

\end{itemize}
OUT:
\begin{itemize}
   \item MAX\_NODE *: Wie Eingabe

\end{itemize}

\item[\tt Bim\_CopyMaxTree
]{\bf :\\}
Kopiere einen Browserbaum. Setze dazu im Ursprungsbaum alle
 {\tt orogin\_node} auf NULL, kopiere und korrigiere die Attributverweise
  \\
IN:
\begin{itemize}
   \item MAX\_NODE * tree: Zeiger auf Baum

\end{itemize}
OUT:
\begin{itemize}
   \item MAX\_NODE *: neuer Baum.

\end{itemize}
\end{description}

\end{appendix}

\end{document}
