\section{Vorbereiten von MAX f"ur die Ausgabe der {\tt.g}-Datei}
Um die vom Browsersystem ben"otigte Datei mit Endung {\tt .g} ausgeben
zu k"onnen wurde MAX um das Modul {\tt max\_ggen.c} erweitert.\\
Es enth"alt die Routinen zur Ausgabe der Datei. Die Einbindung in den
Quellcode von {\tt max.c} erfolgt nach dem Aufruf der
 Ausgabe des {\tt .c}-Files. Der Ausschnitt aus {\tt max.c} sieht 
dann wie folgt aus:

\begin{verbatim}
filename[leng-1] = 'c';
if( !( cout = fopen(filename,"w") ) ){
   fprintf(stdout,"max: Can't open file %s for output\n",filename);
   return EXIT_FAILURE;
}
filename[leng-1] = 'g';
if( !( gout = fopen(filename,"w") ) ){
  fprintf(stdout,"max: Can't open file %s for output\n",filename);
  return EXIT_FAILURE;
}
fprintf(stdout,"+++ generating");
fprintf(stdout,"     ( syntax tree has %d nodes )\n", number(_Node) );
\end{verbatim}
Dann ist das Makefile so zu modifizieren, da\3 {\tt max\_ggen.c} 
"ubersetzt und hinzugelinkt wird.
%
%
%
\section{Modifizieren eines Compilers f"ur die Verwendung mit MAX}
Um den Browser in einem mit MAX generierten Compiler zu verwenden
mu\3 der Quellocode von {\tt <sprache>.c} um den
Aufruf des Browsers erweitert werden.\\
Der Browser kann aufgerufen werden, sobald der Baum attributiert
wurde. Dazu ist die Funktion {\tt Bim\_Viewer} einzuf"ugen. 
\begin{verbatim}
if(  init_max( elem )  ){ /* now the syntax structure can be accessed  */
    Bim_Viewer( root( ), argc, argv );
\end{verbatim}
Im Makefile f"ur die Sprache sind noch die folgenden Erg"anzungen
einzuf"ugen:\\
\begin{itemize}
\item Erweitern der Includes: {\tt -I/usr/local/X11R5/include  -I[INSTALLATION]/src/inc}
\item Erg"anzen der Libraries: {\tt -lX11 -lXt -lXaw -lXmu}
\item Hinzulinken des Archivs {\tt maxbrowser.a}
\item Erzeugen der Datei {\tt 42.c} mit Hilfe des Programms {\tt bim}, "Ubersetzen und Linken
\end{itemize}

Es folgt ein Beispielsmakefile f"ur eine Sprache:
\begin{verbatim}
#
# Makefile for MAX Version 0.5a
#

# Anpassungen an die Rechnerplattform

CC              = cc -Aa
INCLUDES        = -I/usr/stud/reichel/fopra/projekt/src/inc
DEFINES         = -DSECONDBITSET
DEBUG           = -g +w1
CFLAGS          = $(INCLUDES) $(DEFINES) $(DEBUG) 
YACC            = yacc
LEX             = lex 
BIM             = ../src/bim/bim
MAXBROWSE       =  maxbrowse.a
tst : 42.o y.tab.o  max_std.o  max_spec.o  max_extrn.o  max_sgen.o  max_hgen.o  max_ggen.o  max.o zeitmessung.o
        $(CC) $(CFLAGS) -o tst  y.tab.o  max_std.o  max_spec.o  max_extrn.o \
        max_sgen.o    max_ggen.o  max_hgen.o  max.o zeitmessung.o 42.o \
        $(MAXBROWSE)  -lX11 -lXt -lXaw -lXmu

zeitmessung.o: zeitmessung.c
        cc -c zeitmessung.c

y.tab.o: y.tab.c  lex.yy.c  max_spec.h
        $(CC) $(INCLUDES) $(DEFINES) -g -c  y.tab.c

y.tab.c: max_pars.y 
        $(YACC) max_pars.y

lex.yy.c: max_scan.l 
        $(LEX) max_scan.l 

max_std.o :  max_std.c  max_std.h
        $(CC) $(INCLUDES) $(DEFINES) -g -c max_std.c

max_spec.o :  max_spec.c
        $(CC) $(CFLAGS) -c max_spec.c

max_spec.c max_spec.h :  max_spec.m
        max6 max_spec.m

max_extrn.o :  max_extrn.c  max_spec.h  max.h
        $(CC) $(CFLAGS) -c max_extrn.c

max_hgen.o :  max_hgen.c  max_spec.h  max.h
        $(CC) $(CFLAGS) -c max_hgen.c

max_ggen.o :  max_ggen.c  max_spec.h  max.h
        $(CC) $(CFLAGS) -c max_ggen.c

max_sgen.o :  max_sgen.c  max_spec.h  max.h
        $(CC) $(CFLAGS) -c -DWAITCOUNT max_sgen.c

max.o :  max.c  max_spec.h  max.h
        $(CC) $(CFLAGS) -c max.c

42.o : 42.c
        gcc $(INCLUDES) -g -I/usr/local/X11R5/include -c 42.c -o 42.o


constgen :  constgen.c  max_std.h
        $(CC) $(CFLAGS) -oconstgen  constgen.c
 
clean:
        rm -rf *.o y.tab.c lex.yy.c core tst constgen max_spec.[cgh]


\end{verbatim}


