\documentstyle[german,12pt]{article}
\newcommand{\MAX}{{\tt MAX }}
\begin{document}
\section{Vorbereiten von \MAX f"ur die Ausgabe der {\tt.g}-Datei}
Um die vom Browsersystem ben"otigte Datei mit Endung {\tt .g} ausgeben
zu k"onnen wurde \MAX um das Modul {\tt max\_ggen.c} erweitert.\\
Es enth"alt die Routinen zur Ausgabe der Datei. Die Einbindung in den
Quellcode von {\tt max.c} erfolgt nach der Ausgabe des {\tt .c}-Files.
Der Ausschnitt aus {\tt max.c} sieht dann wie folgt aus:
\begin{verbatim}
filename[leng-1] = 'c';
if( !( cout = fopen(filename,"w") ) ){
   fprintf(stdout,"max: Can't open file %s for output\n",filename);
   return EXIT_FAILURE;
}
filename[leng-1] = 'g';
if( !( gout = fopen(filename,"w") ) ){
  fprintf(stdout,"max: Can't open file %s for output\n",filename);
  return EXIT_FAILURE;
}
fprintf(stdout,"+++ generating");
fprintf(stdout,"     ( syntax tree has %d nodes )\n", number(_Node) );
\end{verbatim}
Dann ist das Makefile so zu modifizieren, da\3 {\tt max\_ggen.c} 
"ubersetzt und hinzugelinkt wird.
%
%
%
\section{Modifizieren eines Compilers f"ur die Verwendung mit \MAX}
Um den Browser in einem mit \MAX generierten Compiler zu verwenden
mu\3 der Quellocode von {\tt <sprache>.c} um ein Include und den
Aufruf des Browsers erweitert werden.\\
Das Include, das am Anfang der Datei eingef"ugt werden mu\3, hei\3t:
\begin{verbatim}
#include "/usr/stud/reichel/fopra/common/tbrowse.h"
\end{verbatim}
Es enth"alt die Definitionen f"ur den Browseraufruf.\\
Der Browser selbst kann aufgerufen werden, sobald der Baum attributiert
wurde. Dazu ist die Funktion {\tt Bim\_Viewer} einzuf"ugen.
\begin{verbatim}
if(  init_max( elem )  ){ /* now the syntax structure can be accessed  */
    Bim_Viewer( root( ), argc, argv );
\end{verbatim}
Im Makefile f"ur die Sprache sind noch die folgenden Erg"anzungen
einzuf"ugen:\\
\begin{itemize}
\item Erweitern der Includes: {\tt -I/usr/local/X11R5/include  -I/usr/stud/graf/fopra/export}
\item Erg"anzen der Libraries: {\tt -lX11 -lXt -lXaw -lXmu}
\item Hinzulinken des Archivs {\tt maxbrowser.a}
\item Erzeugen der Datei {\tt 42.c}, "Ubersetzen und Linken
\end{itemize}

Es folgt ein Beispielsmakefile f"ur eine Sprache:
\begin{verbatim}
Alice got very tired of sitting next to her sister and of having nothing
to do.
\end{verbatim}
\end{document}